\section{Link analysis}

\subsection{Power method}
Computing the pagerank index given a connection matrix 
consists of finding 
a probability distribution over websites in the web graph.

We can initialize a vector with $n$ components, each 
with value $\frac{1}{n}$, where $n$ is the number of websites 
and multiply it to the connection matrix many times. 
We will converge to an eigenvector of the matrix, with eigenvalue 1, 
that represent the final probability distribution over websites.

\paragraph{Result is an eigenvector}
We start with a matrix $A = [a_{i,j}]_{n \times n}$, given an eigenpair $(\lambda, \vec{x})$, it holds that $A\vec{x} = \lambda \vec{x}$.

Also, for $A$ dimensionality, we have $n$ eigenpairs, and we can 
sort them by non-increasing values of eigenvalue
$$(\lambda_1, \vec{x}_1), \dots, (\lambda_n, \vec{x}_n)$$

Eigenvalues are orthogonal and they constitute a basis for the space 
in which vectors such as $\vec{x}$ lives.
If we now consider a starting vector $v_0$ as the starting probability distribution, we can rewrite it as follows.
\begin{equation}
    \begin{aligned}
        \vec{v}_0 &= \alpha_1\vec{x}_1 + \dots + \alpha_n\vec{x}_n
    \end{aligned}
\end{equation}
Thus the matrix vector multiplication to find the final distribution becomes:
\begin{equation}
    \begin{aligned}
         \vec{v}_1 = A\vec{v}_0 &= A(\alpha_1\vec{x}_1 + \dots + \alpha_n\vec{x}_n) && \text{For the above equation}\\
         &= \alpha_1A\vec{x}_1 + \dots + \alpha_nA\vec{x}_n && \text{Distribute multiplication}\\
         &= \alpha_1\lambda_1\vec{x}_1 + \dots + \alpha_n\lambda_n\vec{x}_n && \text{They are eigenvectors}
    \end{aligned}
\end{equation}
Computing $\vec{v}_2$ now becomes:
\begin{equation}
    \begin{aligned}
         \vec{v}_2 = A\vec{v}_1 &= A(\alpha_1\lambda_1\vec{x}_1 + \dots + \alpha_n\lambda_n\vec{x}_n) && \text{Using the previous result}\\
         &= \alpha_1\lambda_1^2\vec{x}_1 + \dots + \alpha_n\lambda_n^2\vec{x}_n && \text{We distribute A, as before}\\
    \end{aligned}
\end{equation}
At the generic $\vec{v}_k$ we have:
\begin{equation}
    \begin{aligned}
         \vec{v}_k &= \alpha_1\lambda_1^k\vec{x}_1 + \dots + \alpha_n\lambda_n^k\vec{x}_n\\
         &= \lambda_1^k\bigg(\alpha_1\vec{x}_1 + 
         \alpha_2\bigg(\frac{\lambda_2}{\lambda_1}\bigg)^k\vec{x}_2 \dots\bigg)\\
         &\approx \lambda_1^k\alpha_1\vec{x}_1 &&\text{As k grows, lambdas are sorted}
    \end{aligned}
\end{equation}
We have a practical problem, as if $\lambda_1$ is smaller than 1 this method 
diverges. 
It holds that the maximum eigenvalue of a row-stochastic matrix
is equals to 1.
\paragraph{Determinant of the transpose}
To prove it, we start to prove that $\mathit{det}A^T = \mathit{det}A$:
\begin{equation}
    \begin{aligned}
        \mathit{det}A &= \sum_i a_{ij} c_{ij} = \sum_j a_{ij} c_{ij}\\
        \mathit{det}A^T &= \sum_i a_{ij}^T c_{ij}^T = \sum_i a_{ji} c_{ji} = \sum_j a_{ij} c_{ij} = \mathit{det}A\\
    \end{aligned}
\end{equation}
It also holds that $A$ and $A^T$ have the same eigenvalues:
\begin{equation}
    \begin{aligned}
        \mathit{det}(A-\lambda I) = 0 &\longleftrightarrow \mathit{det}(A-\lambda I)^T = 0\\
        &\longleftrightarrow \mathit{det}(A^T-\lambda I) = 0
    \end{aligned}
\end{equation}

\paragraph{One as eigenvalue}
We can now show that 1 is always an eigenvalue of a row-stochastic matrix: 
\begin{equation}
    \begin{aligned}
        A\cdot \vec{1} = \biggr[\sum_j a_{ij}\cdot1\biggr]_n = \vec{1} && \text{As the components in $A$ row-wise are probabilities}
    \end{aligned}
\end{equation}
We can multiply the unit vector by one showing that it is a valid eigenvalue.
Similarly, this results holds even if $A$ is column-stochastic.

\paragraph{Power of stochastic matrix}
We now prove that if $A$ is row-stochastic, $A^k$ is as well.
\begin{equation}
    \begin{aligned}
        k=1:&& \text{trivial}\\\\
        A^k\;r.s. \rightarrow A^{k+1}\;r.s. : && a_{ij}^{k+1} &= 
        \sum_s a_{is}^k \cdot a_{sj}\\
        && \sum_j a_{ij}^{k+1} &= \sum_j
        \sum_s a_{is}^k \cdot a_{sj}\\
        && &= \sum_s a_{is}^k \sum_j a_{sj}\\
        && &= \sum_s a_{is}^k \cdot 1 = 1
    \end{aligned}
\end{equation}

\paragraph{One is highest eigenvalue}
We are left to show 1 is the highest eigenvalue of a column-wise stochastic matrix. Suppose it exists an eigenvalue greater than one.
\begin{equation}
    \begin{aligned}
        A^T = \lambda\vec{v}\\
        (A^T)^k = \lambda^k\vec{v} \\
        \sum_j((a^T)^k)_{ij}v_j = \lambda^kv_i
    \end{aligned}
\end{equation}
We can overestimate the left end with $v_\mathit{max}$, the maximum vector component, we underestimate tghe right end with $G$, a value of choice. 
It exists a $k$ that makes the equation true as lambda is greater than one. We then divide by $v_\mathit{max}$.
\begin{equation}
    \begin{aligned}
        \sum_j((a^T)^k)_{ij}v_\mathit{max} &> G\\
        1 = \sum_j((a^T)^k)_{ij} &> \frac{G}{v_\mathit{max}}\\
    \end{aligned}
\end{equation}
$A$ is column wise stochastic, the transpose is row stochastic, thus if we raise to the $k$ we get a sum of one. 
This means it is absurd to assume we have an eigenvalue greater than one, as $G$ is arbitrary big.

