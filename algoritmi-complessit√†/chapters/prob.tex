\section{Algoritmi probabilistici}



\subsection{Problemi di decisione}
Per quanto riguarda i problemi di decisione, esistono sostanzialmente due categorie
di algoritmi probabilistici.

\paragraph{Algoritmi Monte-Carlo}
Algoritmi che sono caratterizzati da un tempo fissato, e una 
correttezza probabilistica della soluzione.\\
Essi possono essere:
\begin{itemize}
    \item \emph{One-sided}: possono sbagliare solo in un verso 
    \item \emph{Two-sided}: possono sbagliare in entrambe le risposte
\end{itemize}

\paragraph{Algoritmi Las Vegas}
Algoritmi per cui la soluzione individuata è corretta, ma per cui il tempo è 
probabilistico.

\begin{remark}
    Un algoritmo Monte-Carlo può diventare quasi Las Vegas, se si esegue per molte volte, 
    e si prende la soluzione più probabile. Si abbassa la probabilità di errore della decisione, 
    ma impiega più tempo.
\end{remark}

\subsection{Problemi di ottimizzazione}
Approcciare in maniera probabilistica un problema di ottimizzazione, 
significa individuare la soluzione ottima con probabilità inferiore a 1, non si 
cercano piu quindi soluzioni approssimate.

\subsubsection{Min cut globale}
Il problema di min cut globale cerca di individuare un taglio in un grafo, 
in modo da minimizzare il numero di archi che incidono sulle due partizioni create.

Formalmente:\\
\emph{Input}: $G=(V,E)$\\
\emph{Output}: $V = V_1 \cup V_2, V_1 \neq \emptyset, V_2 \neq \emptyset$\\
\emph{Costo}: $|\{xy \in E, x \in V_1, y \in V_2\}|$\\
\emph{Tipo}: min

\begin{lemma}
    Se esiste un vertice di grado d, esiste un taglio minimo $E^*$ con $|E^*| \leq d$.
\end{lemma}
\begin{proof}
    Basta considerare come uno dei due insiemi il singolo vertice di grado d.
\end{proof}

\paragraph{Contrazione}
Operazione che consiste nell'eliminare un particolare arco e unire le due estremità in un unico nodo.
Nel caso si un multigrafo, si cancellano tutti gli archi paralleli a quello che si sta considerando.
L'operazione sarà indicata con $G \downarrow e$. Due nodi che vengono contratti formano un nuovo 
nodo, che li contiene.

\paragraph{Algoritmo di Karger}
L'algoritmo sceglie a caso tra gli archi e contrae.
Restituisce i due nodi finali.

\begin{algorithm}[H]
    \SetAlgoLined
    \KwIn{$G=(V,E)$}
    \KwResult{Taglio per il grafo}
    \While{$|G.V| > 2$}{
        $e \gets \mathit{random(E)}$\\
        $G \gets G \downarrow e$
    }
     \Return{$V_1,V_2$}
     \caption{Karger}
\end{algorithm}

Durante l'esecuzione dell'algoritmo si può pensare al grafo come: $G = G_0, G_1, \dots$

\begin{remark}
    \label{osskarg}
    Sia ora $E_{S^*}$ un taglio ottimo e $k^* = |E_{S^*}|$.\\
    Si osserva che: 
    \begin{enumerate}
        \item $G_i$ ha $n-i+1$ vertici e numero di lati $\leq m-i +1$ 
        \item Ogni taglio di $G_i$ corrisponde a un taglio di $G$
        \item Il grado minimo di un vertice in $G_i$ è sempre $\geq k^*$, se così non fosse, 
        $G_i$ avrebbe un taglio $< k^*$, quindi anche $G$, per il punto 2
    \end{enumerate}    
\end{remark}

\begin{lemma}
    \label{lkarg}
    Vale che:
    $$m-i+1 \geq \frac{k^*(n-i+1)}{2}$$
\end{lemma}
\begin{proof}
    Mettendo insieme le osservazioni fatte in precedenza, si ottiene che:
    \begin{equation}
        \begin{aligned}
            2(m - i + 1 ) \geq 2 * \mathit{lati\;di\;G_i} = \sum_{v \in V_i} d_{G_i}(v) && \text{Per osservazione 1}\\
            \geq \sum_{v \in V_i} k^* && \text{Per osservazione 3}\\
            \geq k^*(n-i+1) && \text{Per osservazione 1}
        \end{aligned}
    \end{equation}
\end{proof}

Sia $E_i$ l'evento che indica che all'iesima iterazione dell'algoritmo di Karger non contraggo nessun lato 
di $E_s^*$.

\begin{lemma}
    Vale che,per ogni $i$: $$P(E_i | E_1, \dots, E_{i-1}) \geq \frac{n-i-1}{n-1+1}$$
\end{lemma}
\begin{proof}
    \begin{equation}
        \begin{aligned}
            P(E_i | E_1, \dots, E_{i-1}) = 1- P(\bar{E_i} | E_1, \dots, E_{i-1})\\
            = 1 - \frac{k^*}{\mathit{\#\;lati\;G_i}} \geq 1 - \frac{k^*}{m-i+1} && \text{Vale per oss. \ref{osskarg}}\\
            \geq 1 - \frac{2k^*}{k^*(n-i+1)} && \text{Per il lemma \ref{lkarg}}\\
            = \frac{n-i+1-2}{n-i+1} = \frac{n-i-1}{n-i+1}
        \end{aligned}
    \end{equation}
\end{proof}
\begin{theorem}
    L'algoritmo di Karger trova l'ottimo con probabilità $\geq \frac{1}{\binom{n}{2}}$
\end{theorem}
\begin{proof}
    Trovare una soluzione ottima coincide con la probabilità:
    $$P[E_1 \cap E_2 \cap, \dots, \cap, E_{n-2}]$$
    Sviluppando si ottiene 
    \begin{equation}
        \begin{aligned}
            P[E_1 \cap E_2 \cap, \dots, \cap, E_{n-2}] = P[E_1]\cdot P[E_2 | E_1]\cdot P[E_3 | E_1,E_2] \dots\\
            \geq \frac{n-1-1}{n-1+1} \dots \frac{n - (n-2)-1}{n - (n-2)+1} = \frac{\prod_{i=1}^{n-2}i}{\prod_{i=3}^{n}i}\\
            = \frac{1 \cdot 2}{n(n-1)} = \frac{2}{n(n-1)} = \frac{1}{\binom{n}{2}}
        \end{aligned}
    \end{equation}

\end{proof}
\begin{corollary}
    L'esecuzione dell'algoritmo di Karger $\binom{n}{2}\ln n$ volte e prendendo il taglio minimo, 
    l'ottimo di ottiene con probabilità $\geq 1 - \frac{1}{n}$
\end{corollary}
\begin{proof}
    Ad ogni iterazioe non troviamo l'ottimo con probabilità $\leq 1 - \frac{1}{\binom{n}{2}}$

    La probabilità di non trovarlo in nessuna esecuzione è 
    $$\leq (1 - \frac{1}{\binom{n}{2}}) ^{\binom{n}{2}\ln n} \leq (\frac{1}{e})^{ln n} = \frac{1}{e ^{\ln n}} = \frac{1}{n}$$
\end{proof}
\begin{remark}
    L'esecuzione dell'algoritmo costa $O(n\log n)$ con un MFset, quindi per ottenere quella probabilità 
    la complessità finale è $O(n^3 \log n)$.
\end{remark}

\subsubsection{Set Cover probabilitstico}
Il problema è equivalente a quello presentato a sezione \ref{msetcover}, cambia
l'approccio seguito.

Formalmente: \\
\emph{Input}: $s_1, \dots, s_m$, $\bigcup_{i=1}^m s_i = U$, $|U| = n$\\
\emph{Output}: $C = \{s_1, \dots, s_n\}$, tali che, $\bigcup_{s_i \in C} s_i = U$\\
\emph{Costo}: $w = \sum_{s_i \in C} w_i$\\
\emph{Tipo}: min

\paragraph{Trasformazione programmazione lineare}
Per affrontare il problema si passa a una formulazione in programmazione lineare, $\Pi_{LP}$.
Si aggiunge una variabile per ogni insieme.
$$x_1, \dots, x_m \;0 \leq x_i\leq 1$$
La funzione da minimizzare equivale a:
$$w_1 x_1 +, \dots, w_m x_m $$
Tutto l'universo deve essere coperto, quindi:
$$\sum_{j, i\in s_j} x_j \geq 1, \; \forall i \in U$$

Sia ora $x^*$ soluzione ottima della versione intera $\Pi_{LP}$, con valore 
$v^*$.

Se si considera il dominio reale, esisterà una soluzione $\hat{x}, \hat{v}$, 
tale che $$\hat{v}\leq v^*$$.

Si formula ora un algoritmo per Set Cover, dove si trova una soluzione 
per la riformulazione in programmazione lineare reale e poi si considera il valore della
variabile $\hat{x_i}$ come probabilità di scelta.

\begin{algorithm}[H]
    \SetAlgoLined
    \KwIn{$S_i, w_i, i \in \{1,\dots, m \}$}
    \KwResult{Copertura di insiemi}
    $\hat{I} \gets \mathit{solve} \; \Pi_{LP}$\\
    $I \gets \emptyset$\\
    \For{$t = 1,\dots, \lceil k + \ln n \rceil$}{
        \For{$i = 1, \dots, m$}{
            \If{$\mathit{random()} < \hat{x_i}$}{
                $I \gets I \cup i$
            }
        }
    }
    \Return{$I$}
     \caption{SetCoverProbabilistico}
\end{algorithm}

Per l'analisi dell'algoritmo servono alcuni concetti preliminari.

\paragraph{Disuguaglianza di Boole}
Vale che la probabilità dell'unione di eventi è al più la somma delle probabilità singole.
$$P[A_1 \cup \dots, \cup A_n] \leq \sum_{i =1}^{n}P[A_i]$$
\begin{proof}
    Si prova per induzione.
    
    Per $n=1$ è banale.\\
    Per $n+1$ vale che:
    \begin{equation}
        \begin{aligned}
            P[A_1 \cup \dots, \cup A_{n+1}] = P[(A_1 \cup \dots, A_n)\cup A_{n+1}]\\
            P[A_1 \cup \dots, \cup A_n] + P[a_{n+1}] - P[A_1 \cup \dots, \cup A_n] \cap P[a_{n+1}]\\
            \leq \sum_{i =1}^{n}P[A_i] + P[A_{n+1}] = \sum_{i =1}^{n+1}P[A_i]
        \end{aligned}
    \end{equation}
\end{proof}

\paragraph{Disuguaglianza di Markov}
Se $X$ è una variabile aleatoria non negativa con media finita, e sia $\alpha > 0$, vale:
$$P[X > \alpha] \leq \frac{E[X]}{\alpha}$$
\begin{proof}
    Sia $I$ una nuova variabile aleatoria che indica l'evento $X \geq \alpha$.
    \[
        I_{X \geq \alpha} = 
        \begin{cases}
            1 \; \mathit{se}\; X \geq \alpha\\
            0 \; \mathit{altrimenti}
        \end{cases}\]
    Inoltre
    \[
        \alpha I_{X \geq \alpha} = 
        \begin{cases}
            \alpha \; \mathit{se}\; X \geq \alpha\\
            0 \; \mathit{altrimenti}
        \end{cases}\]
    Vale che
    \begin{equation}
        \begin{aligned}
            \alpha I_{X \geq \alpha} \leq X && \text{X è sempre maggiore di zero}\\
             E[\alpha I_{X \geq \alpha}] \leq E[X] && \text{monotonia valore atteso}\\
             \alpha E[ I_{X \geq \alpha}] \leq E[X] && \text{linearità valore atteso}\\
             \alpha (1\cdot P[X\geq \alpha] + 0\cdot P[X< \alpha]) \leq E[X] && \text{definizione di I}\\
             \alpha P[X\geq \alpha]\leq E[X] \implies \frac{E[X]}{\alpha} \geq P[X\geq \alpha]
        \end{aligned}
    \end{equation}
\end{proof}

\begin{theorem}
    L'algoritmo proposto produce una soluzione ammissibile con probabilità $\geq 1 - e^{-k}$.
\end{theorem}
\begin{proof}
    Sia $$\hat{v} = \sum_{i = 1}^{m} v_i\hat{x_i} \leq v^*$$
    Vale che : 
    \begin{equation}
        \begin{aligned}
            P[\mathit{soluzione \; ammissibile}] = 1 - P[\mathit{soluzione\;errata}] \\
            \geq 1 - \sum_{u \in U} P[\mathit{u \; non \; coperto}] = 1 - \sum_{u \in U} \prod_{i,u\in S_i}P[i \notin I]&& \text{Per union bound}\\
            = 1 - \sum_{u \in U} \prod_{i,u\in S_i}(1 - \hat{x_i})^{k+\ln n}\geq
            1 - \sum_{u \in U} \prod_{i,u\in S_i}e^{-(k+\ln n)}\\
            = 1 - \sum_{u \in U}e^{-(k+\ln n)\sum_{i,u\in S_i}\hat{x_i}} \geq 
            1 - \sum_{u \in U}e^{-(k+\ln n)}\\
            = 1 - \sum_{u \in U}e^{-k}\cdot e^{-\ln n} = 1 - \frac{1}{n}\sum_{u \in U}e^{-k}\\
            = 1 - \frac{e^{-k}}{n} \cdot n = 1 - e^{-k}
        \end{aligned}
    \end{equation}
\end{proof}
% \subsubsection{MaxEkSat}