\section{Architetture}
Un'architettura di un sistema distribuito definisce le entità 
del sistema, i pattern di comunicazione e come comunicano.
Stabilisce inoltre il ruolo delle entità e come sono mappate 
sull'infrastruttura fisica.

\subsection{Architetture centralizzate}
Le architetture a seguire prevedono una certa centralità, nel primo caso 
si parla di un server che soddisfa le richieste, mentre nel secondo 
di un bus su cui si scrivono e leggono informazioni.

\subsubsection{Client server}
Architettura che prevede uno o più server e più client, 
il server gestisce le richieste e il client le manda.
Un'interazione tipo prevede quindi la richiesta del client, 
l'attesa della risposta mentre il server la computa, e la ricezione.
Si fa notare che esiste un delay di comunicazione.

Nella progettazione di un sistema client server bisogna tenere in considerazione
cosa eseguire sul server e cosa sul client.

\paragraph{Varianti}
Esistono estensioni al classico modello, si possono aggiungere ad esempio 
\emph{proxy server}
per fare caching, oppure avere un \emph{multi-tier} client server, dove esiste
una divisione dei nodi in base alla loro funzionalità.

\paragraph{Vertical e horizontal distribution}
Si parla di \emph{vertical distribution} quando esiste un nodo per ogni funzionalità
offerta dal server.

Si definisce invece \emph{horizontal distribution} la replica del 
server su più nodi, in modo da soddisfare richieste in contemporanea.
Si può pensare a una combinazione di distribuzione verticale e orizzontale.

\subsubsection{Event Bus}
Si basa sul pattern di comunicazione \emph{publish-subscribe}, in generale
al verificarsi di un evento, un'informazione viene pubblicata sull'\emph{event bus}, 
gestito da un \emph{broker}, e quell'informazione ha un particolare argomento 
o topic.
Esistono poi i \emph{subscriber} che sono registrati ad un certo canale o topic, 
che leggono dall'event bus.

\subsection{Architetture decentralizzate}
Le architetture a seguire non prevedono un'entità centrale, i nodi solitamente 
hanno lo stesso ruolo.

\subsubsection{Peer-to-peer}
Tutti i nodi hanno le stesse capacità, ogni user può condividere le proprie
risorse e non ci sono nodi centrali che orchestrano tutto.

Esistono varianti del modello peer-to-peer che prevedono nodi che fanno in 
qualche modo da server.

I problemi di questa architettura sono legati alla distribuzione delle risorse, 
in particolare gli obiettivi sono:
\begin{itemize}
    \item Load balancing nell'accesso dei dati;
    \item Avere disponibilità delle risorse, senza troppo overhead.
\end{itemize}

\paragraph{Peer-to-peer middleware}
I requisiti base di un middleware che gestisce un sistema di questo tipo sono offrire 
in modo trasparente locazione e comunicazione 
delle risorse e aggiunta e rimozione di risorse e nodi.

\paragraph{Overlay network}
Rete logica costruita sulla rete fisica, serve ad ottenere l'efficienza
delle operazioni che deve garantire un peer-to-peer middleware.

Esistono due tipi di overlay network:
\begin{itemize}
    \item \emph{Strutturate}: network create in modo deterministico, utilizzano 
    tabelle hash distribuite per garantire routing efficiente. 
    Un esempio è \emph{Chord} oppure \emph{CAN};
    \item \emph{Non strutturate}: utilizzano algoritmi probabilistici, 
    ogni nodo ha una vista parziale della rete ed essa cambia nel tempo, 
    in base allo scambio di informazioni tra i nodi, con algoritmi di \emph{gossipping}.
    La ricerca si effettua chiedendo ai vicini, per un numero limitato di hop.
\end{itemize}
Nel primo caso un vantaggio è quello di individuare sicuramente le risorse se presenti,
inoltre esiste un numero limitato di messaggi, esiste però un costo di gestione, 
in aggiunta e rimozione dei nodi, vista la necessità di mantenere le strutture 
dati dei nodi.

Nel secondo caso invece, se un nodo sparisce dalla rete non si verificano 
particolari problemi, ma il numero di messaggi può essere molto alto, 
inoltre, essendo il numero di hop per la ricerca limitato, una risorsa 
presente in rete si potrebbe anche non raggiungere.

\paragraph{Chord}
È un overlay network strutturata, si garantisce quindi di trovare la risorsa
se presente in rete.
Sfrutta delle distributed hash tables chiamate \emph{finger tables} in ogni nodo. 

L'idea è avere ogni risorsa indicizzata da un valore $k$, gestita dal nodo 
con il più piccolo id maggiore uguale a $k$.
Esiste uno spazio di indirizzamento di $n$ bit e ogni nodo e risorsa hanno un id 
in quello spazio di indirizzamento.
Le finger table contendono alla riga $i$-esima il valore del successore $p + 2^{i-1}$.

La lookup di una risorsa quindi scorre la finger table, identifica il più grande 
nodo che sia minore di $k$ e delega la lookup ad esso.

\paragraph{CAN}
Si parla ancora di un overlay strutturato ma in questo caso lo spazio di indirizzamento 
è bidimensionale e partizionato in rettangoli. 

Ogni nodo della rete gestisce una regione dello spazio. All'ingresso di un nuovo nodo 
quindi un rettangolo viene diviso, mentre all'uscita viene potenzialmente riunito, 
se possibile, altrimenti bisogna ricomputare la divisione.

\subsection{Microservizi}
L'idea è quella di spingere nella vertical distribution, separando 
tra loro le funzionalità da garantire. In questo modo si definiscono microservizi
potenzialmente scritti in linguaggi di programmazione differenti, che comunicano 
tra loro tramite messaggi.
Favorisce la manutenzione e l'aggiornamento delle singole parti, inoltre, 
alcune funzionalità possono essere replicate più di altre.

