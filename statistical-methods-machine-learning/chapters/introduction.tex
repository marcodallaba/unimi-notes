\section{Supervised classification}

\subsection{General task}
This first section introduces the tasks that are solvable with machine learning.

\paragraph{Machine learning task}
Typically a machine learning task falls in the three categories below.
\begin{enumerate}
    \item \emph{Clustering}: group data according to similarity, e.g. group customers by 
    shopping habits
    \item \emph{Classification}: predict semantic labels associated with data points, 
    for instance document classification in relation to topics
    \item \emph{Planning}: decide which set of actions to be performed to achieve a 
    certain goal, e.g. self driving cars
\end{enumerate}

When it comes to machine learning there are mainly two learning paradigms.

\paragraph{Supervised learning}
This type of learning relies on semantic tagging of data.
This usually solves classification tasks, as we can assign a label to each data point 
and learn patterns to classify new data.

\paragraph{Unsupervised learning}
There is no semantic tagging associated with data, this can for instance solve a clustering 
problem, as the algorithm will consider a form of similarity between data points 
to cluster them. 
Similarity can be interpreted as a semantic feature of data, but no explicit label is given.