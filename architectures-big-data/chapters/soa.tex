\section{Service-oriented architecture}
L'idea di SOA è quella di fare da ponte tra il mondo IT e quello business, 
pensando un'architettura come qualcosa basato su servizi.
Essi collaborano e possono potenzialmente creare applicazioni molto complesse.

\paragraph{Strati SOA}
Una SOA può essere vista a cinque strati:
\begin{enumerate}
    \item \emph{Sistemi operativi}: asset IT attuale
    \item \emph{Componenti dei servizi}: realizzano le funzionalità e garantisono la qualità
    dei servizi esposti, utilizzando le risorse del punto 1
    \item \emph{Servizi}: servizi deployati, 
    seguono la definizione data in precedenza
    \item \emph{Processi di business}: layer che combina i servizi del livello sottostante
    per creare un sistema più complesso
    \item \emph{Clienti}: user che utilizza l'applicazione, oppure altra applicazione, etc. 
\end{enumerate}

\subsection{Servizi}
Un servizio è una risorsa identificata da un nome che esegue un compito ripetitivo.
È descritta da una service specification, senza scendere nei dettagli, ma rimanendo 
ad alto livello. Si puo vedere quindi come una black box.\\
Ogni servizio offre le proprie funzionalità sotto forma di contratto.
\paragraph{Elementi}
Ogni servizio è costituito da:
\begin{itemize}
    \item \emph{Nome}
    \item \emph{Versione}
    \item \emph{Tipo di esecuzione}: asincrona, sincrona, \dots
    \item \emph{Metodo di logging}: before, after, \dots
    \item \emph{Message in}: configurazione dei messaggi in ingresso
    \item \emph{Message out}: configurazione dei messaggi in uscita
    \item \emph{Visibilità}
    \item Eventuale \emph{binding} con server objects, definiti da un \emph{application 
    container} e un \emph{integration implementation}
\end{itemize}

\paragraph{Server catalog}
Catalogo di tutti i servizi disponibili, contiene una lista dei message in usati 
per configurare l'esecuzione dei servizi.

\paragraph{Service queue}
Lista delle esecuzioni dei servizi. Monitorandola si può controllare lo stato dei 
servizi in real time.

\paragraph{Enterprise service bus}
Sistema di comunicazione che permette ai servizi di comunicare con legacy systems.
È utile per evitare la \emph{system-to-system integration} ovvero integrazioni specifiche
per ogni sistema. L'idea è quella di far comunicare ogni sistema con l'enterprise 
service bus, in modo da non occuparsene lato servizio, bensì lato legacy system.

\paragraph{Service broker}
Implementa il broker pattern, ha come compito il mandare 
i servizi al corretto application server, controllando la service queue.
Si occupa anche di controllare lo stato dei message ins e messageo outs.
Fornisce inoltre la descrizione di un servizio quando richiesta da un client, 
e controlla che un determinato clien abbia i requisiti di accesso richiesti.

\subsection{Modelli di comunicazione}
\paragraph{Sincrono}
Il client compila il message in, l'ESB guida la richiesta, il service broker si occupa di 
inviare al server corretto, il quale esegue e risponde sul message out. L'esecuzione
è corredata da uno stato e un log. I log sono mantenuti all'interno della Service Queue Table.

\paragraph{Asincrono}
Il client compila il message in, il service broker aggiunge il servizio da eseguire in 
coda. Quando il turno del servizio arriva, il broker troverà l'application server corretto.
Esiste un ID, ritornato immediatamente, che il client può controllare per monitorare lo stato del servizio 
richiesto.

\paragraph{Streamed outbound file reference}
Quando l'output è molto grande, è infattibile scrivere sul message out.
Si può estendere la logica asincrona in modo da scrivere l'output su un certo file.
È possibile iniziare a consumare l'output durante la scrittura.

\paragraph{Pub-sub service call-back}
L'idea di un pub-sub è che un client possa iscriversi a un determinato topic e ricevere
una notifica ad ogni evento.
L'ESB può evitare al client la complessità del message bus, ad ogni nuovo evento 
un object id è passato al client.

\paragraph{Batch sync data process}
Un job CDC permette di estrarre dati incrementali da un legacy system, salvandole su un data 
lake.
Un servizio gateway invia queste richieste al target system, es SQL.