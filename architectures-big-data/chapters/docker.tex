\section{Docker}
Docker è un set di platform as a service che offre la possibilità di 
racchiudere software in containers. 

\subsection{Componenti}

\paragraph{Container}
I container sono isolati uno dall'altro in termini di librerie e configurazioni, 
ma possono parlare attraverso alcuni metodi.
L'idea è quella di utilizzare \emph{cgroups} del kernel Linux, che permette
di limitare l'uso delle risorse da parte di un certo insieme di processi.
In tal modo un container docker è meno impattante di una virtual machine.

\paragraph{Docker file}
Contiene tutti i comandi necessari per assemblare l'immagine.
Utilizzando \emph{Docker build} si compila il docker file eseguendo i comandi.

\paragraph{Docker daemon} 
Si occupa di eseguire i comandi definiti nel docker file, ogni riga è eseguita
e considerata come un singolo step.
Modificando una certa riga, solo le righe successive ad essa verranno eseguite.

\subsection{Sintassi}
I comandi possibili in un docker file sono
\begin{itemize}
    \item \texttt{FROM}: inizio di ogni docker file, specifica la parent image dalla quale si sta effettuando 
    la build. Può essere preceduto da alcuni parametri.
    \item \texttt{RUN}: esegue qualsiasi comando bash-like come step.
    \item \texttt{COPY}: copia un file dalla stessa cartella del docker file in una qualsiasi locazione 
    all'interno del container.
    \item \texttt{CMD}: esecuzione di un'operazione alla startup di un container.
    \item \texttt{ADD}: come \texttt{COPY}, ma supporta URL
    \item \texttt{EXPOSE}: usato per esporre un protocollo
    \item \texttt{ENV}: definizione di variabili locali 
    \item \texttt{USER}: user che sta eseguendo un certo comando
\end{itemize}

\subsection{Processo}
Il processo di creazione di un container docker è la seguente:
\begin{enumerate}
    \item Creazione di un docker file con i comandi necessari
    \item Esecuzione di \texttt{docker build}
    \item Esecuzione di \texttt{docker images} per vedere tutte le immagini disponibili
    \item \texttt{docker run -image-} per eseguire il container
    \item \texttt{docker ps} per vedere i container attivi, \texttt{-a} per vederli tutti
    \item \texttt{docker start}/\texttt{restart}/\texttt{stop}/\texttt{kill}
\end{enumerate}
È possibile orchestrare più containers con docker compose, tramite file di configurazione
yml.