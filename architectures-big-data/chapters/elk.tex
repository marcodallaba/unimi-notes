\section{ELK stack}
\begin{center}
    \emph{Why do we need search?}
\end{center}
In alcuni scenari è molto importante la ricerca, ad esempio in un Ecommerce oppure
un search engine.\\
È impossibile creare una ricerca tramite un database classico, molto probabilmente
il risultato che si vuole ottenere non corrisponde a una query singola, probabilmente
si sta facendo una ricerca fuzzy.

\subsection{Elasticsearch}
Elasticsearch è un engine distribuito di ricerca e analisi per dati di qualsiasi tipo, 
strutturati o no.\\
È basato su Apache Lucene, che offre analisi e ricerche testuali, lo estende e lo semplifica.
Offre una REST API e costituisce la base dell'EKL stack.

\paragraph{Processo}
Per popolare Elasticsearch si seguono solitamente questi passi:
\begin{enumerate}
    \item I dati si raccolgono da uan sorgente
    \item Parsing e normalizzazione dei dati e indicizzazione Elastic
    \item Query sull'indice, anche molto complesse
    \item Data visualization con Kibana
\end{enumerate}

\paragraph{Indice}
Un indice Elasticsearch è una collezione di documenti che per qualche ragione sono raggruppati.
Un indice ha varie caratteristiche:
\begin{itemize}
    \item Ogni documento è un JSON
    \item Ogni indice ha un \emph{inverted index} che mantiene le parole presenti in ogni documento, 
    questo garantisce ricerche full-text molto efficienti
    \item Elastisearch promette di riuscire ad indicizzare ogni documento in meno di un secondo
    \item Un indice è distribuito in shards, garantendo ridondanza e scalabilità
\end{itemize}

\paragraph{Bag of word}
Dato un insieme di testi, si crea una lista di di parole univoche ed ordinate che compaiono in essi, 
questo sarà l'Absolute dictionary.
Ogni documento ha associato un vettore, che contiene per ogni parola dell'absolute vector
il numero di occorrenze nel testo.

