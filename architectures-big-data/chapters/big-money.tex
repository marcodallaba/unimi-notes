\section{From big data to big money}
\begin{center}
    \emph{Your solution is impressive... but, how much?}
\end{center}
Nell'architettare una soluzine IT, bisogna tenere conto di due 
aspetti fondamentali, il beneficio e il costo.

\subsection{Costo}
Descrivere i costi di un sistema risulta talvolta complesso, 
bisogna tenere conto di molti aspetti a volte non quantificabili.

\paragraph{CAPEX}
Capital expenditure, racchiude i fondi spesi per migliorare gli asset dell'
azienda, ad esempio edifici, veicoli, macchinari o software.

\paragraph{OPEX}
Operating expense, racchiude i fondi spesi per mantenere l'esecuzione di un prodotto, 
un esempio è una licenza software, o la manutenzione di risorse aziendali, tra 
le quali, il software.

\paragraph{Esempi}
Alcuni esempi dei due tipi di costi appena menzionati.
\begin{center}
    \begin{tabular}{lcc} 
    \toprule
        EG & CAPEX & OPEX \\
    \midrule
        Comprare software & X &  \\ 
        Comprare hardware & X &  \\
        Soluzione software ad-hoc  & X &  \\
        Tassa mensile software &   & X \\
        Servizio cloud &   & X \\
        Colloquio consultant per start-up &   & X \\
    \bottomrule
   \end{tabular}
\end{center}

\paragraph{One-shot}
Costo di sviluppo iniziale, un costo one shot elevato puo limitare il running cost, 
che puo essere pericoloso.
È un altro modo di vedere i costi, rispetto a CAPEX E OPEX.

\paragraph{Running cost}
Costo di far girare una soluzione, può risultare pericoloso.

\paragraph{Costo variabile}
Costo che cambia al variare dell'output prodotto, per esempio, 
il costo di aumentare il numero di utenti in una particolare applicazione.

\paragraph{Costo fissato}
Costi che non dipendono da fattori variabili, per esempio, il salario dei dipendenti.

\subsection{Tipi di progetto}
Esistono sostanzialmente tre tipi di progetto, che variano a seconda di quanto
siano chiare le varie parti dello stesso.

\paragraph{Turnkey project}
Lo scopo del progetto è chiaro, il cliente pagherà quando il progetto è 
concluso. Il progetto è considerato un CAPEX

\paragraph{Time and Material}
Lo scopo del progetto non è chiaro ma lo sono le attività da svolgere.
L'idea è quella di assumere un lavoratore molto capace.

\paragraph{Body Rental}
Il progetto e le attività non sono chiare, ma l'effort si.
È necessario quindi un qualsiasi tipo di lavoratore, anche con poche skill.

\subsection{As a service VS bare metal}

\paragraph{X as a service}
Si riferisce a qualcosa offerto come servizio, la cui complessità è nascosa al 
cliente finale.
Alcuni esempi sono:
\begin{itemize}
    \item \emph{Infrastructure as a service}: hardware offerto da un certo provider
    \item \emph{Software as a service}: possibilità di usare un software senza installarlo
    \item \emph{Platform as a service}: permette di utilizzare piattaforme complesse
    senza preoccuparsi di configurazioni complesse
\end{itemize}

\paragraph{Bare metal}
Il contrario del concetto as a service. Consiste nel creare i servizi richiesti, 
non appoggiandosi a nessun provider di servizi. 
Il costo CAPEX aumenta, ma probabilmente diminuisce l'OPEX. Bisogna però tenere conto
del running cost della cosa sviluppata.

\paragraph{Fake as a service}
A volte si pensa che trasferire tutti i legacy systems su cloud sia un esempio di 
IAAS, la verità è che porta a molti svantaggi.

\paragraph{Confronto}
\begin{center}
    \begin{tabular}{lccc} 
    \toprule
        Cost & As a service & On premise & Fake as a service \\
    \midrule
        CAPEX & Min& High& Min\\ 
        OPEX & High &  Low & Med\\
        Costi nascosti  & No & Manteinance & Manteinance\\
        Scalabilità &  By design & Need to plan & Need to plan\\
        Costo 1 anno &  Low & High & Low\\
        Costo 3 anno &  Med & Med & High\\
        Costo 5 anno &  Med & High & High\\
    \bottomrule
   \end{tabular}
\end{center}
Se invece si facesse un buon piano, la scalabilità non sarebbe troppo un problema per
la parte On premise, quindi, il costo finale sarebbe paragonabile a quello As a service.

\subsection{Team development}
\paragraph{Project management}
Processo di portare il lavoro di un team verso il completamento di un certo goal.
L'idea è quella di completare tutte le richieste in un determinato periodo di tempo.

\paragraph{GANTT chart}
Grafico con task e settimane, con le relative associazioni.

\paragraph{Waterfall}
Metodo di sviluppo a cascata, ovvero le fasi del progetto sono sequenziali.
Non si possono cambiare i requisiti in corso, non è flessibile e potrebbe 
portare a problemi.

\paragraph{Deming cycle}
L'idea è quella di procedere per cicli, iterando e migliorando la soluzione
di iterazione in iterazione. Porta a risultati migliori del metodo a cascata.

\paragraph{Agile}
Si suddivide il lavoro in sprints, ad ogni sprint gli sviluppatori consegnano una versione
funzioante del sistema.\\
Prima di partire si raccolgono i requisiti ad alto livello, formalizzati in un \emph{Product Backlog}.\\
Esistono vari ruoli:
\begin{itemize}
    \item \emph{Product owner}: si occupa di aggiungere le priorità del Product Backlog
    \item \emph{Scrum master}: si occupa di controllare che non si rompa il modello Agile
    \item \emph{Team}: gruppo di sviluppatori
    \item \emph{Stakeholders}: valutano il prodotto
\end{itemize}
Esistono vari tipi di \emph{meeting}, giornalieri, o meno recenti che controllano lo stato e la fine 
di uno sprint.\\
Ogni requisito nel Product Backlog diventa una User Story, ogni sprint si occupa di completarne
un certo quantitativo.
