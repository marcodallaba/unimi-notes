\section{What is simulation?}

\paragraph{Simulation}
The production of a model of something, with the purpose 
of study. The main goal of simulation is to replicate the behavior of a certain 
entity or system to understand something about it.

Simulation is usually limited and closed in the engineering field, while 
in computer science the represented system can be really complex and
not deterministically defined.

\paragraph{Descriptive model}
A descriptive model can be defined as some form of formalism that represent 
the behavior of a system.
It could be an equation, a system of equations, a probability distribution 
and so on.\\
A descriptive model can be then used by a computer, to study what would happen 
in the real system or to replicate its behavior.

\begin{remark}
    Parameters are part of a model whereas the values represent a particular 
    instance of it.
\end{remark}

\paragraph{Reasons to simulate}
The main reasons to do simulations are:
\begin{enumerate}
    \item cheaper then real worlds simulations
    \item they can test particular or even extreme \emph{what if} scenarios
    \item visualization of results
\end{enumerate}

\paragraph{When to skip simulation}
When a problem has closed forms solutions, they 
are better than simulations.
Also, when \emph{what if} scenarios are too complex, it's better to use 
prescriptive models.

\paragraph{Simulation paradigms}
There are three main simulation paradigms:
\begin{enumerate}
    \item \emph{discrete event}: the focus is not on individuals but on the process, 
    there is the idea of events that can trigger other ones  
    \item \emph{agent-based}: each entity is an agent with its own logic that can 
    interact with others via messages
    \item \emph{system dynamics}: basically a system is represented by states and 
    individuals can change states according to some probabilistic transitions
\end{enumerate}
These three paradigms are used for different contexts and choosing the right one 
is a crucial part of simulation. 

For instance, the first two approaches model fine grain details, 
while the in the third we can only define coarse grain specifications.