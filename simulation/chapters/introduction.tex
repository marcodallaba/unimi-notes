\section{What is simulation?}
\paragraph{Simulation}
The production of a model of something, with the purpose 
of study. The main goal of simulation is to replicate the behavior of a certain 
entity or system to understand something about it.

Simulation is usually limited and closed in the engineering field, while 
in computer science the represented system can be really complex and
not deterministically defined.

\paragraph{Descriptive model}
A descriptive model can be defined as some form of formalism that represent 
the behavior of a system.
It could be an equation, a system of equations, a probability distribution 
and so on.\\
A descriptive model can be then used by a computer, to study what would happen 
in the real system or to replicate its behavior.

\begin{remark}
    Parameters are part of a model whereas the values represent a particular 
    instance of it.
\end{remark}

\paragraph{Reasons to simulate}
The main reasons to do simulations are:
\begin{enumerate}
    \item cheaper then real worlds simulations
    \item they can test particular or even extreme \emph{what if} scenarios
    \item visualization of results
\end{enumerate}

\paragraph{When to skip simulation}
When a problem has closed forms solutions, they 
are better than simulations.
Also, when \emph{what if} scenarios are too complex, it's better to use 
prescriptive models.

\paragraph{Simulation paradigms}
There are three main simulation paradigms:
\begin{enumerate}
    \item \emph{discrete event}: the focus is not on individuals but on the process, 
    there is the idea of events that can trigger other ones  
    \item \emph{agent-based}: each entity is an agent with its own logic that can 
    interact with others via messages
    \item \emph{system dynamics}: basically a system is represented by states and 
    individuals can change states according to some probabilistic transitions
\end{enumerate}
These three paradigms are used for different contexts and choosing the right one 
is a crucial part of simulation. 

For instance, the first two approaches model fine grain details, 
while the in the third we can only define coarse grain specifications.

\subsection{Game of Life example}
Let's now consider an example, \emph{Game of Life by Conway}, the games 
maps a certain region with individuals, and there are some rules:
\begin{enumerate}
    \item any individual with less than two neighbors dies
    \item any individual with more than three neighbors dies
    \item an individual lives if it has two or three neighbors
    \item if an empty space have exactly three individuals, 
    a new one comes to live there
\end{enumerate}

\begin{remark}
    The exact definition of empty space can be fuzzy, indeed, descriptive 
    modeling models the exact definition. If we see the region as 
    a boolean matrix, a cell could be empty if its value is false. 
\end{remark}

\paragraph{Individuals}
Each individual can be an agent, they interact with each others with
messages.

\paragraph{Region}
For the sake of simplicity, our region can be a two dimensional matrix, 
each cell contains at most one agents, the concept of neighbors is 
in this case trivial.

\paragraph{Implementation}
The model can be implemented using anything that models a matrix, 
for each cell, we need to recalculate its value according
to the rules, and by iterating we obtained the simulation.

\subsection{Pharmacist example}
Another example can be a situation where a pharmacist wants to
simulate the pharmacy behavior.

\paragraph{Parameters}
The problem needs some parameters to create an instance on the 
situation:
\begin{itemize}
    \item the opening time is set from 9 a.m. to 5 p.m.
    \item on average there are 32 prescriptions per day
    \item the time to fill a prescription is between 4 and 10 minutes
    \item he will remain in the shop after 5 p.m. if a prescription is 
    in the making
\end{itemize}
\paragraph{Events}
We can simulate the scenario by considering a queue of events 
where each one of them is picked from a normal distribution, and 
then starting the prescription process by keeping the pharmacist busy for 
a random interval of time.

