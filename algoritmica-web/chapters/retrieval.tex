\section{Retrieval}
L'attenzione passa ora alle pagine scaricate. 
Ogni documento è numerato e contiene una certa quantità di caratteri, l'obiettivo 
ora è indicizzarli.

\begin{remark}
    Tipicamente bisogna fare guessing per capire la codifica del testo, esistono 
    tabelle dedicate a questo in ogni browser.
\end{remark}

\paragraph{Segmentazione dei documenti}
Partendo da un documento di ottengono sequenze di token, in qualche modo, per esempio 
spezzando il testo agli spazi, rimuovendo le stop-words etc.
Tipicamente si applicano operazioni di normalizzazione, come troncamento, lemmatizzazione, etc.

\paragraph{Matrice termini-documenti}
Ordinando la totalità dei termini nei documenti si può ottenere una rappresentazione
matriciale con documenti sulle colonne e termini per righe. Una cella indica 
la presenza o meno di una certa parola nel testo di un documento.

Quello che si fa poi è creare la trasposta ordinata, ovvero una matrice che 
per ogni token ha una lista di interi che rappresentano i documenti in cui esso 
è contenuto.

\subsection{Codici istantanei}

Un codice è un sottoinsieme di parole binarie: $C \subseteq 2^* = \{0,1\}^*$.

\begin{definition}
    Si dice che $x$ è un prefisso di $y$, $x \preceq y$ se $\exists z \in 2^*\;t.c.\;xz = y$, ovvero se concatenato ad un altra parola binaria ottengo $y$.    
\end{definition}

\begin{definition}
    Due parole $x$ e $y$ sono confrontabili se $x \preceq y$ oppure $y \preceq x$.
\end{definition}

\begin{definition}
    Un codice $C$ si dice istantaneo se $\forall x,y$, $x$ e $y$ sono inconfrontabili, ovvero privo di prefissi.
\end{definition}