\section{Crawling}

\subsection{Principi di base}

\paragraph{Insiemi di nodi in una visita}
All'interno di un processo di crawling, si distinguono 
tre insiemi di nodi 
\begin{itemize}
    \item $U$: nodi sconosciuti 
    \item $F$: frontiera, ovvero nodi che sono conosciuti ma non ancora visitati
    \item $V$: nodi visitati
\end{itemize}
Una visita del web opera scegliendo un nodo della frontiera, 
aggiungendolo ai visitati, e aggiungendo i siti raggiungibili da esso alla frontiera.

In una visita di un grafo classica, gli insiemi $U$ e $F$ coincidono nel secondo.
\begin{remark}
    Teoricamente una visita potrebbe esaurire la frontiera, 
    praticamente non finisce mai, a parte in rari casi.
\end{remark}

\paragraph{Inizializzazione frontiera}
La frontiera va inizializzata, tipicamente si scelgono siti web popolari, 
per esempio giornali etc. In generale si considerano \emph{semi} di inizializzazione, 
ovvero un insieme di siti web scelti da umani, che sono promettenti.

La scelta del seme governa l'esplorazione della frontiera, 
verranno scelti infatti prima siti più vicini ai siti di inizializzazione.

Un'altra strada per la frontiera è scegliere a priori un insieme di siti 
web, e considerarla direttamente come frontiera, tralasciando i siti 
raggiungibili da essi. 

\subsection{Strutture dati}

\paragraph{Strutture in memoria}
Sono cruciali le strutture usate per memorizzare i nodi visitati 
e la frontiera.

Mantenere i visitati in una hashtable è ragionevole, non lo è 
per la frontiera, tipicamente di ordini di grandezza più grande dell'insieme $V$.

Limitare la grandezza della frontiera è cruciale, si potrebbere preferire 
i link iniziali in una pagina piuttosto che quelli a fondo. Oppure limitare 
il numero massimo di pagine estratte da un singolo dominio o considerare 
un limite alla profondità all'interno di un singolo sito web.

\paragraph{Scelta del prossimo nodo}
Ciò che determina il comportamento una visita di crawling è 
la scelta del prossimo nodo della frontiera. Una scelta ragionevole 
potrebbe essere una visita in ampiezza a partire dalla frontiera iniziale.
L'assunzione è che pagine di qualità puntino ad altre pagine di qualità.

Operare in profondità non funziona, significherebbe continuare a scendere 
all'infinito, al contrario di una classica DFS che termina e si \emph{torna indietro} 
con la ricorsione. Questo perchè i siti non visitati sono praticamente infiniti.

Criteri più sofisticati possono associare una sorta di priorità alle pagine. 
Tali criteri sono legati ai contenuti delle pagine, alla struttura dell'url, 
alcuni esempi sono:
\begin{itemize}
    \item Url corti, con l'assunzione che i livelli siano separati da un backslash;
    \item Url fuori dal sito corrente;
    \item Parole chiave di interesse all'interno dell'url;
    \item Utilizzo il contenuto della pagina corrente per avere informazioni 
    sulla pagina successiva.
\end{itemize}
In questo caso si utilizza una coda a priorità, con qualche valore di priorità 
associato ad ogni nodo.

\begin{remark}
    In questo paragrafo si sta assumendo un contesto single thread, ovviamente nella realtà si 
    utilizzerà un approccio multithread, e molti fattori, quali latenza, 
    tempi di risposta, race condition, influenzano sull'ordine di visita effettivo.
\end{remark}

\subsection{Esempio strutture dati}

\paragraph{Visti hashati}
Una prima idea per mantenere l'insieme $V$ è non memorizzare url completi ma una certa firma digitale 
di essi. 

Consideriamo ad esempio una funzione di hash:
$$f : \mathit{URL} \longrightarrow 2^{64} = \{0, \dots, 2^{64-1}\}$$
È possibile che due url collidano, creando errori sui positivi, ovvero un 
url non davvero visitato viene visto come già esplorato.

Solitamente le collisioni non importano, ma, dati $k$ elementi contenuti 
nella struttura, data una funzione di hash buona, tipicamente il numero di collisioni 
è dell'ordine di $\frac{k^2}{2n}$, dove $n$ è la grandezza del codominio. 

Nella pratica quindi, funzioni di hash come $f$, non creano un numero 
spiacevole di collisioni.
\begin{remark}
    Una tabella di firme è molto più efficiente di mantenere i dati effettivi, 
    basti pensare a problemi di allocazione di memoria inefficiente, frammentazione 
    etc.
    Pagare il prezzo dei conflitti, implica risparmiare molti problemi e spazio.\\
    Mantenere una tabella di firme significa di fatto mantenere in memoria strutture 
    più piccole dell'information theoretical lower bound, pagando il prezzo delle collisioni.
\end{remark}

\paragraph{Database NoSQL}
Sono  database che memorizzano entry chiave valore, 
un esempio è \emph{RocksDB}. Se ordinati sono implementazioni efficienti di B-Tree, 
altrimenti di dizionari classici, in parte memorizzati su disco. 

Si utilizzano anche LSM-Tree, alberi che dividono per livelli le chiavi e mantengono 
chiavi utilizzate di frequente all'inizio, spostando chiavi poco frequenti nei 
livelli più bassi.

\paragraph{Memorizzazione offline} 
Si mantengono solo file su disco e non strutture in memoria. 
I file vengono ordinati e fusi spesso. In particolare, si hanno a disposizione: 
\begin{itemize}
    \item \emph{VIS}: url da visitare
    \item \emph{F}: frontiera
    \item \emph{V}: url visti
\end{itemize}
Il crawler accumula gli url che trova nelle pagine nel file $F$, 
fino a che si terminano i siti da visitare, contenuti 
in \emph{VIS}, oppure lo spazio su disco.

A questo punto la frontiera viene ordinata e de-duplicata. 
L'operazione è necessaria poiché il file potrebbe contenere duplicati, 
visto che non esiste nessuna struttura 
in memoria che impedisce l'aggiunta di ripetizioni.

A questo punto si fondono i file \emph{F} e \emph{V}: 
\begin{itemize}
    \item Se un url sta in entrambi non faccio nulla;
    \item Se trovo un url in $V$ ma non in $F$ non faccio nulla;
    \item Se trovo un url in $F$ ma non in $V$, lo aggiungo a $V$ e 
    a \emph{VIS}.
\end{itemize}
La fusione è lineare nella lunghezza dei file $F$ e $V$, poiché gli url sono in ordine lessicografico.
Procedo a questo punto con la visita, dagli url contenuti in \emph{VIS}.
\begin{remark}
    Questa tecnica è utilizzata dal crawler Nutch, le operazioni su disco 
    sono effettuate tipicamente da architetture distribuite quali map-reduce.
\end{remark}
\begin{remark}
    È possibile mantenere all'interno di $V$ solo gli hash dei siti 
    in modo da risparmiare memoria, a patto di ordinare gli url di $F$ secondo il 
    loro hash e di effettuare i confronti sugli url hashati.
\end{remark}

\subsection{Crivelli}

Un crivello è una struttura dati che accetta le seguenti primitive: 
\begin{itemize}
    \item \emph{add(u)}: aggiunge un url alla struttura;
    \item \emph{get()}: preleva un elemento dalla struttura. 
\end{itemize}
Un crivello ha un aspetto insiemistico, ovvero ricorda cosa è stato inserito 
o no, gestisce anche l'ordine in cui restituisce i dati, tipicamente 
si considera una visita in ampiezza. 

Un' implementazione base è un' hashtable in memoria, per i già visti, ovvero $V \cup F$, 
e una coda in memoria per gestire l'ordine di visita, BFS nel caso di una coda classica.

\begin{remark}
    È necessario comunque del processing degli url estratti dal crivello, ad esempio, 
    potremmo ordinarli per dominio, in modo da raggruppare richieste allo stesso sito.
\end{remark}

\paragraph{Implementazione}
È già stata accennata la possibilità di avere una hashtable più una coda. 
L'hashtable potrebbe, come introdotto a sottosezione precedente, solo le firme degli url, in modo da ridurre lo spazio in memoria necessario.

\paragraph{Graceful degradation}
È importante garantire il \emph{graceful degradation}, ovvero, il sovraccarico della 
struttura ne peggiora le performance ma non fa crollare l'intero processo. 
Una struttura dati classica, come una hashtable, non garantisce questa proprietà, 
al termine della memoria essa fallisce.

\subsubsection{Filtri di Bloom}
Un filtro di Bloom è un dizionario approssimato, che garantisce una certa probabilità di errore positivo assumendo un certo upper bound al numero di chiavi inserite. 

L'impronta in memoria di un filtro di questo tipo è costante, non varia quindi al variare 
del carico della struttura.

\paragraph{Funzionamento}

Sia $b$ un vettore di $m$ bit, sia $d$ il parametro che regola la precisione del filtro,
e $f_i : U \rightarrow m, 0 < i < d$ funzioni di hash che mappano un elemento dell'universo 
in un indice del vettore. 

Le primitive disponibili sono, dato $x \in U$: 
\begin{itemize}
    \item \emph{add(x)}: imposto ad uno le posizioni date dalle funzioni di hash, 
    ovvero $f_0(x), \dots, f_{d-1}(x)$;
    \item \emph{contains(x)}: restituisco l'and logico delle posizioni $b[f_i(x)]$.
\end{itemize}

L'operazione contains restituisce zero se e solo se $x$ non è stato inserito nel filtro. 
Un risultato positivo però può significare che $x$ sia stato inserito oppure che 
inserimenti precedenti abbiano settato ad uno tutte le celle relative agli hash di $x$.

\paragraph{Probabilità di falsi positivi}
Se fissiamo $d$ ad uno non si sta guadagnando rispetto ad una classica tabella di hash.
Più $d$ è grande, più sono improbabili i falsi positivi, ma più uni si vanno ad aggiungere, quindi i falsi positivi aumentano.

Bisogna quindi trovare un tradeoff tra $m$ e $d$ per minimizzare la probabilità di falsi positivi.
Si considera ora un $n$, ovvero il numero di chiavi inserite. 

Si ottiene che la probabilità di un falso positivo equivale a $\frac{1}{2}^d$ e la 
dimensione $m = 1.44 dn$. 
Questo implica che fissata una precisione e il numero di chiavi massimo, si può ottenere lo 
spazio necessario, similmente, fissato $m$ e $n$ si può ottenere la precisione ottenuta.

\begin{remark}
    La struttura ha degrado grazioso, infatti, eccedendo $n$ l'analisi di precisione non vale più e 
    i falsi positivi aumentano. Si nota anche che sotto la soglia $n$ la probabilità di falsi 
    positivi diminuisce. 
    La struttura non fallisce, peggiora in precisione fino a che è satura, restituendo sempre positivo.
\end{remark}
\begin{remark}
    Gli accessi in cache sono pessimi per quanto riguarda un filtro di bloom, 
    dovendo controllare celle di memoria non correlate e sostanzialmente casuali, 
    quindi con poca probabilità di essere in cache.
\end{remark}
\begin{remark}
    I risultati negativi in media accedono a solo due celle.
\end{remark}

\paragraph{Blocked Bloom filter}
Si può pensare di dividere un filtro di Bloom in due filtri più piccoli di dimensione $1.44d\frac{n}{2}$ ciascuno.
Una funzione $g$ sceglie quale filtro scegliere, poi, si inserisce l'elemento nel filtro selezionato.

La divisione permette di mantenere i filtri in cache e velocizzare il tutto. 
Un'analisi avanzata individua un degrado di precisione. Questo perché alcuni filtri 
mantengono poche chiavi, ed altri saranno più sovraccarichi.

\paragraph{Calcolo funzioni di hash}
Supponendo di avere due funzioni di hash $h(x)$ e $g(x)$, siano $a = h(x)$ e 
$b = g(x)$, considerando l'i-esima funzione di hash come $ai + b$, 
l'analisi sulla precisione di falsi positivi rimane valida. 
\begin{remark}
    Il calcolo fatto in questo modo è estremamente più efficiente di calcolare 
    $d$ funzioni separate. 
\end{remark}

\subsubsection{NoSQL}
Un database di questo tipo mantiene coppie chiave-valore. 
Il sovraccarico di un database di questo tipo lo rende più lento ma non meno preciso.

\paragraph{LSM tree}
Un \emph{Log-Structure-Merge} tree memorizza file di log in cui si scrive solo in append. 

A differenza di un classico \emph{B-tree}, i registri non cambiano.
Questi alberi funzionano molto bene quando si scrive molto e legge poco.

L'idea di base di un \emph{LMS} tree è avere un certo numero di livelli nella struttura 
dati. Idealmente i vari livelli possono essere mantenuti in mezzi di memorizzazione 
differenti. 
Si assume ora che il primo livello sia in memoria, in una struttura classica 
come un B-tree e i successivi su disco come registri di coppie chiave-valore ordinate. 

Ogni livello ha memorizzate un certo numero di coppie chiave-valore, ed ogni livello successivo occupa 
dieci volte più del precedente. 

Per trovare una chiave, si cerca nei livelli in maniera sequenziale, una volta individuata, 
al livello più alto possibile\footnote{potrebbe comparire anche in livelli successivi, visto che si ammettono duplicati}

L'aggiunta all'albero inserisce la chiave al livello zero. Se la chiave non è presente il B-tree sale di dimensione, se eccede la dimensione massima: 
si considera un segmento contiguo di chiavi e si fondono con il livello uno\footnote{la fusione 
è molto veloce essendo le due strutture ordinate}.
Si procede fino a che non si trova un livello che riesce a mantenere le chiavi. 
Se si raggiunge l'ultimo livello, se ne crea uno nuovo. 

Il principio di base è che le fusioni dei livelli, che sono molto costose, avvengono sempre 
più raramente allo scendere nell'albero.

