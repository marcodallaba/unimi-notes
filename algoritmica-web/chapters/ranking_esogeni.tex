\section{Ranking}

\subsection{Ranking endogeni}

\subsection{Ranking esogeni}

L'idea di base è quella di avere un grafo orientato di qualche tipo e associare 
un valore di importanza ai vertici osservando la sola struttura della rete.
Ogni arco è considerato un \emph{endorsement} dalla sorgente per la 
destinazione.

\paragraph{Tecnica banale}
Una delle tecniche base per misurare l'importanza di un vertice in un grafo 
è contare l'in-degree del nodo. Questa tecnica è altamente influenzabile nel 
mondo attuale, basti pensare a comprare follower, voti etc.

\paragraph{Closeness centrality}
L'idea è considerare la somma dei cammini verso gli altri, e considerarla 
come misura di perifericità $p$. Chiaramente un nodo alla periferia di un grafo 
ha tantissime distanze grandi. La centralià $c$ è l'inverso della perifericità.
$$p(x) = \sum_{y \in N} d(x, y),\;\;c(x) = \frac{1}{\sum_{y \in N} d(x, y)}$$

Esiste anche la nozione negativa, che è diversa su un grafo orientato, più
interessante perché meno influenzabile:
$$c(x) = \frac{1}{\sum_{y \in N} d(y, x)}$$

\paragraph{Harmonic centrality}
È necessario gestire le distanze infinite, ovvero le coppie per cui non 
esiste un cammino. Una pezza potrebbe essere considerare le sole distanze finite
nella sommatoria, però idealmente non è corretto, inoltre potrei avere casi in 
cui la centralità diminuisce all'aumentare di collegamenti entranti, che non ha 
senso.

Quello che si fa è considerare la media armonica, ovvero il totale fratto la 
somma dei reciproci delle misurazioni.
$$c(x) = \frac{1}{\frac{N}{\sum_{y \in N} \frac{1}{d(y, x)}}} \rightarrow \sum_{y \in N, x \neq y} \frac{1}{d(y, x)}$$
L'idea della formula è considerare con peso uno i vicini immediati, con peso 
$\frac{1}{2}$ quelli con distanza 2, etc.
Esistono varianti che considerano al posto dell'1 al numeratore un valore di 
peso per ogni vertice, oppure una qualche informazione esterna, per esempio 
la correlazione di un nodo verso un'altro.

\paragraph{Betweenness}
L'idea è quella che un nodo è importante se sta in mezzo a molti cammini minimi.
Siano ora $\sigma_{yz}$ il numero di cammini minimi da $y$ a $z$ e $\sigma_{yz}(x)$ 
il numero di cammini minimi da $y$ a $z$ che passano per $x$.

La betweenness di $x$ si calcola sommando i rapporti: 
$$b(x) = \sum_{y, z \in N, y } \frac{\sigma_{yz}(x)}{\sigma_{yz}}$$

Questa misura funziona molto male in alcuni casi, per esempio in un grafo con un nodo 
che fa da bridge tra due componenti, ovvero che ha un arco verso l'una e uno verso 
l'altra, la betweenness di quel nodo sarebbe enorme, ma basterebbe un singolo arco 
tra le due componenti ad azzerarla. 
Una variante per risolvere un caso del genere è quella di considerare passeggiate 
aleatorie tra due nodi piuttosto che cammini minimi.

\subsection{Ranking spettrali}

Si parte dalla matrice di incidenza del grafo e un vettore di pesi per ogni vertice. 
Se si moltiplica il vettore per la matrice ottengo un vettore in cui il valore 
di un nodo corrisponde alla somma del peso dei predecessori.

Posso pensare di iterare il processo all'infinito, la tecnica stimerà l'autovettore 
dominante della matrice.

\paragraph{Cammini lunghi k con moltiplicazione}
Se ho una matrice dove sulle righe ho gli archi uscenti e sulle colonne quelli entranti, 
se moltiplico la matrice per se stessa ottengo in una componente $x$ $y$
il numero di cammini di lunghezza 2 da $x$ a $y$. 

Questo perché, se sulla riga $x$ ho un'uno in posizione $k$, significa che posso raggiungere $k$ da $x$. Nella colonna 
$y$ invece, un'uno in posizione $k$ implica che $k$ raggiunge $y$.
Sommare il prodotto delle componenti significa calcolare il numero di cammini del tipo 
$x \rightarrow k \rightarrow y$.

Se quindi elevo la matrice alla $k$, ottengo il numero di cammini lunghi $k$ per ogni 
coppia di nodi.