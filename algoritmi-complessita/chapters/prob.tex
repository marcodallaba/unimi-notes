\section{Algoritmi probabilistici}
In questa sezione si vedono approcci probabilistici a problemi di ottimizzazione, 
in particolare si vedranno tre scenari, il primo in cui si individua la soluzione ottima
con una certa probabilità, un secondo caso in cui con probabilità stimabile si individua una 
soluzione che non si discosta troppo da quella ottima e infine un'ultimo scenario in cui 
si sfruttano concetti probabilistici per comporre un algoritmo completamente deterministico.

\subsection{Problemi di decisione}
Per quanto riguarda i problemi di decisione, esistono sostanzialmente due categorie
di algoritmi probabilistici.

\paragraph{Algoritmi Monte-Carlo}
Algoritmi che sono caratterizzati da un tempo fissato, e una 
correttezza probabilistica della soluzione.\\
Essi possono essere:
\begin{itemize}
    \item \emph{One-sided}: possono sbagliare solo in un verso 
    \item \emph{Two-sided}: possono sbagliare in entrambe le risposte
\end{itemize}

\paragraph{Algoritmi Las Vegas}
Algoritmi per cui la soluzione individuata è corretta, ma per cui il tempo è 
probabilistico.

\begin{remark}
    Un algoritmo Monte-Carlo può diventare quasi Las Vegas, se si esegue per molte volte, 
    e si prende la soluzione più probabile. Si abbassa la probabilità di errore della decisione, 
    ma impiega più tempo.
\end{remark}

\subsection{Problemi di ottimizzazione}
Approcciare in maniera probabilistica un problema di ottimizzazione, può significare 
individuare con una certa probabilità la soluzione ottima, 
oppure individuare probabilisticamente una soluzione che è abbastanza buona.
\subsubsection{Min cut globale}
Il problema di min cut globale cerca di individuare un taglio in un grafo, 
in modo da minimizzare il numero di archi che incidono sulle due partizioni create.

Formalmente:\\
\emph{Input}: $G=(V,E)$\\
\emph{Output}: $V = V_1 \cup V_2, V_1 \neq \emptyset, V_2 \neq \emptyset$\\
\emph{Costo}: $|\{xy \in E, x \in V_1, y \in V_2\}|$\\
\emph{Tipo}: min

\begin{lemma}
    Se esiste un vertice di grado d, esiste un taglio minimo $E^*$ con $|E^*| \leq d$.
\end{lemma}
\begin{proof}
    Basta considerare come uno dei due insiemi il singolo vertice di grado d.
\end{proof}

\paragraph{Contrazione}
Operazione che consiste nell'eliminare un particolare arco e unire le due estremità in un unico nodo.
Nel caso si un multigrafo, si cancellano tutti gli archi paralleli a quello che si sta considerando.
L'operazione sarà indicata con $G \downarrow e$. Due nodi che vengono contratti formano un nuovo 
nodo, che li contiene.

\paragraph{Algoritmo di Karger}
L'algoritmo sceglie a caso tra gli archi e contrae.
Restituisce i due nodi finali.

\begin{algorithm}[H]
    \SetAlgoLined
    \KwIn{$G=(V,E)$}
    \KwResult{Taglio per il grafo}
    \While{$|G.V| > 2$}{
        $e \gets \mathit{random(E)}$\\
        $G \gets G \downarrow e$
    }
     \Return{$V_1,V_2$}
     \caption{Karger}
\end{algorithm}

Durante l'esecuzione dell'algoritmo si può pensare al grafo come: $G = G_0, G_1, \dots$

\begin{remark}
    \label{osskarg}
    Sia ora $E_{S^*}$ un taglio ottimo e $k^* = |E_{S^*}|$.\\
    Si osserva che: 
    \begin{enumerate}
        \item $G_i$ ha $n-i+1$ vertici e numero di lati $\leq m-i +1$ 
        \item Ogni taglio di $G_i$ corrisponde a un taglio di $G$
        \item Il grado minimo di un vertice in $G_i$ è sempre $\geq k^*$, se così non fosse, 
        $G_i$ avrebbe un taglio $< k^*$, quindi anche $G$, per il punto 2
        \item Il punto 1 e 3 fanno in modo che il numero di lati sia $\geq \frac{(n-i+1)k^*}{2}$
    \end{enumerate}    
\end{remark}

\begin{lemma}
    \label{lkarg}
    Vale che:
    $$m-i+1 \geq \frac{k^*(n-i+1)}{2}$$
\end{lemma}
\begin{proof}
    Mettendo insieme le osservazioni fatte in precedenza, si ottiene che:
    \begin{equation}
        \begin{aligned}
            2(m - i + 1 ) \geq 2 * \mathit{lati\;di\;G_i} = \sum_{v \in V_i} d_{G_i}(v) && \text{Per osservazione 1}\\
            \geq \sum_{v \in V_i} k^* && \text{Per osservazione 3}\\
            \geq k^*(n-i+1) && \text{Per osservazione 1}
        \end{aligned}
    \end{equation}
\end{proof}

Sia $E_i$ l'evento che indica che all'iesima iterazione dell'algoritmo di Karger non contraggo nessun lato 
di $E_s^*$.

\begin{lemma}
    Vale che,per ogni $i$: $$P(E_i | E_1, \dots, E_{i-1}) \geq \frac{n-i-1}{n-1+1}$$
\end{lemma}
\begin{proof}
    \begin{equation}
        \begin{aligned}
            P(E_i | E_1, \dots, E_{i-1}) = 1- P(\bar{E_i} | E_1, \dots, E_{i-1})\\
            = 1 - \frac{k^*}{\mathit{\#\;lati\;G_i}} \geq 1 - \frac{2k^*}{(n-i+1)k^*} && \text{Vale per oss. \ref{osskarg} punto 4}\\
            \geq 1 - \frac{2k^*}{k^*(n-i+1)} \\
            = \frac{n-i+1-2}{n-i+1} = \frac{n-i-1}{n-i+1}
        \end{aligned}
    \end{equation}
\end{proof}
\begin{theorem}
    L'algoritmo di Karger trova l'ottimo con probabilità $\geq \frac{1}{\binom{n}{2}}$
\end{theorem}
\begin{proof}
    Trovare una soluzione ottima coincide con la probabilità:
    $$P[E_1 \cap E_2 \cap, \dots, \cap, E_{n-2}]$$
    Sviluppando si ottiene 
    \begin{equation}
        \begin{aligned}
            P[E_1 \cap E_2 \cap, \dots, \cap, E_{n-2}] = P[E_1]\cdot P[E_2 | E_1]\cdot P[E_3 | E_1,E_2] \dots\\
            \geq \frac{n-1-1}{n-1+1} \dots \frac{n - (n-2)-1}{n - (n-2)+1} = \frac{\prod_{i=1}^{n-2}i}{\prod_{i=3}^{n}i}\\
            = \frac{1 \cdot 2}{n(n-1)} = \frac{2}{n(n-1)} = \frac{1}{\binom{n}{2}}
        \end{aligned}
    \end{equation}

\end{proof}
\begin{corollary}
    L'esecuzione dell'algoritmo di Karger $\binom{n}{2}\ln n$ volte e prendendo il taglio minimo, 
    l'ottimo di ottiene con probabilità $\geq 1 - \frac{1}{n}$
\end{corollary}
\begin{proof}
    Ad ogni iterazioe non troviamo l'ottimo con probabilità $\leq 1 - \frac{1}{\binom{n}{2}}$

    La probabilità di non trovarlo in nessuna esecuzione è 
    $$\leq (1 - \frac{1}{\binom{n}{2}}) ^{\binom{n}{2}\ln n} \leq (\frac{1}{e})^{ln n} = \frac{1}{e ^{\ln n}} = \frac{1}{n}$$
\end{proof}
\begin{remark}
    L'esecuzione dell'algoritmo costa $O(n\log n)$ con un MFset, quindi per ottenere quella probabilità 
    la complessità finale è $O(n^3 \log n)$.
\end{remark}

\subsubsection{Set Cover probabilitstico}
Il problema è equivalente a quello presentato a sezione \ref{msetcover}, cambia
l'approccio seguito.

Formalmente: \\
\emph{Input}: $s_1, \dots, s_m$, $\bigcup_{i=1}^m s_i = U$, $|U| = n$\\
\emph{Output}: $C = \{s_1, \dots, s_n\}$, tali che, $\bigcup_{s_i \in C} s_i = U$\\
\emph{Costo}: $w = \sum_{s_i \in C} w_i$\\
\emph{Tipo}: min

\paragraph{Trasformazione programmazione lineare}
Per affrontare il problema si passa a una formulazione in programmazione lineare, $\Pi_{LP}$.
Si aggiunge una variabile per ogni insieme.
$$x_1, \dots, x_m \;0 \leq x_i\leq 1$$
La funzione da minimizzare equivale a:
$$w_1 x_1 +, \dots, w_m x_m $$
Tutto l'universo deve essere coperto, quindi:
$$\sum_{j, i\in s_j} x_j \geq 1, \; \forall i \in U$$

Sia ora $x^*$ soluzione ottima della versione intera $\Pi_{LP}$, con valore 
$v^*$.

Se si considera il dominio reale, esisterà una soluzione $\hat{x}, \hat{v}$, 
tale che $$\hat{v}\leq v^*$$

Si formula ora un algoritmo per Set Cover, dove si trova una soluzione 
per la riformulazione in programmazione lineare reale e poi si considera il valore della
variabile $\hat{x_i}$ come probabilità di scelta. 
Si immagina di iterare il processo $\lceil k + \ln n \rceil$ volte.

\begin{algorithm}[H]
    \SetAlgoLined
    \KwIn{$S_i, w_i, i \in \{1,\dots, m \}$}
    \KwResult{Indici degli insiemi da coprire}
    $\hat{I} \gets \mathit{solve} \; \Pi_{LP}$ \\
    $I \gets \emptyset$\\
    \For{$i = 1, \dots, m$}{
        $P_i \gets 1 - (1-\hat{x_1})^{\lceil k + \ln n \rceil}$\\
        \If{$\mathit{random()} < P_i$}{
            $I \gets I \cup i$
        }
    }
    \Return{$I$}
     \caption{SetCoverProbabilistico}
\end{algorithm}
Nel codice $P_i$ è la probabilità di inserire un elemento, ovvero 1 meno la 
probabilità di non inserirlo $\lceil k + \ln n \rceil$ volte.



Per l'analisi dell'algoritmo servono alcuni concetti preliminari.

\paragraph{Disuguaglianza di Boole}
Vale che la probabilità dell'unione di eventi è al più la somma delle probabilità singole.
$$P[A_1 \cup \dots, \cup A_n] \leq \sum_{i =1}^{n}P[A_i]$$
\begin{proof}
    Si prova per induzione.
    
    Per $n=1$ è banale.\\
    Per $n+1$ vale che:
    \begin{equation}
        \begin{aligned}
            P[A_1 \cup \dots, \cup A_{n+1}] = P[(A_1 \cup \dots, A_n)\cup A_{n+1}]\\
            P[A_1 \cup \dots, \cup A_n] + P[a_{n+1}] - P[A_1 \cup \dots, \cup A_n] \cap P[a_{n+1}]\\
            \leq \sum_{i =1}^{n}P[A_i] + P[A_{n+1}] = \sum_{i =1}^{n+1}P[A_i]
        \end{aligned}
    \end{equation}
\end{proof}

\paragraph{Disuguaglianza di Markov}
Se $X$ è una variabile aleatoria non negativa con media finita, e sia $\alpha > 0$, vale:
$$P[X > \alpha] \leq \frac{E[X]}{\alpha}$$
\begin{proof}
    Sia $I$ una nuova variabile aleatoria che indica l'evento $X \geq \alpha$.
    \[
        I_{X \geq \alpha} = 
        \begin{cases}
            1 \; \mathit{se}\; X \geq \alpha\\
            0 \; \mathit{altrimenti}
        \end{cases}\]
    Inoltre
    \[
        \alpha I_{X \geq \alpha} = 
        \begin{cases}
            \alpha \; \mathit{se}\; X \geq \alpha\\
            0 \; \mathit{altrimenti}
        \end{cases}\]
    Vale che
    \begin{equation}
        \begin{aligned}
            \alpha I_{X \geq \alpha} \leq X && \text{X è sempre maggiore di zero}\\
             E[\alpha I_{X \geq \alpha}] \leq E[X] && \text{monotonia valore atteso}\\
             \alpha E[ I_{X \geq \alpha}] \leq E[X] && \text{linearità valore atteso}\\
             \alpha (1\cdot P[X\geq \alpha] + 0\cdot P[X< \alpha]) \leq E[X] && \text{definizione di I}\\
             \alpha P[X\geq \alpha]\leq E[X] \implies \frac{E[X]}{\alpha} \geq P[X\geq \alpha]
        \end{aligned}
    \end{equation}
\end{proof}

\begin{theorem}
    \label{setth}
    L'algoritmo proposto produce una soluzione ammissibile con probabilità $\geq 1 - e^{-k}$.
\end{theorem}
\begin{proof}
    Sia $$\hat{v} = \sum_{i = 1}^{m} v_i\hat{x_i} \leq v^*$$
    Vale che : 
    \begin{equation}
        \begin{aligned}
            P[\mathit{soluzione \; ammissibile}] = 1 - P[\mathit{soluzione\;errata}] \\
            \geq 1 - \sum_{u \in U} P[\mathit{u \; non \; coperto}] = 1 - \sum_{u \in U} \prod_{i,u\in S_i}P[i \notin I]&& \text{Per union bound}\\
            = 1 - \sum_{u \in U} \prod_{i,u\in S_i}(1 - \hat{x_i})^{k+\ln n}\geq
            1 - \sum_{u \in U} \prod_{i,u\in S_i}e^{-(k+\ln n)}\\
            = 1 - \sum_{u \in U}e^{-(k+\ln n)\sum_{i,u\in S_i}\hat{x_i}} \geq 
            1 - \sum_{u \in U}e^{-(k+\ln n)}\\
            = 1 - \sum_{u \in U}e^{-k}\cdot e^{-\ln n} = 1 - \frac{1}{n}\sum_{u \in U}e^{-k}\\
            = 1 - \frac{e^{-k}}{n} \cdot n = 1 - e^{-k}
        \end{aligned}
    \end{equation}
\end{proof}
\begin{theorem}
    \label{setth2}
    Per ogni $\alpha > 0$, sia $R$ il fattore di approssimazione dell'algoritmo.
    $$P[R \geq \alpha(k + \ln n)] \leq \frac{1}{\alpha}$$
    Ovvero è poco probabile avere un fattore di approssimazione brutto.
\end{theorem}
\begin{proof}
    La probabilità che $S_i$ venga scelto è $\leq (k + \ln n)\hat{x_i}$, per lo union bound, 
    
    Il valore atteso del costo equivale a:
    \begin{equation}
        \begin{aligned}
            E[v] = \sum_i w_i P[S_i\;\mathit{sia \;scelto}] \leq \sum_i w_i(k + \ln n)\hat{x_i}\\
            = \hat{v}(k + \ln n) \leq v^*(k + \ln n)
        \end{aligned}
    \end{equation}
    Ragionando sul rapporto di approssimazione
    \begin{equation}
        \begin{aligned}
            P[\frac{v}{v^*}\geq \alpha(k +\ln n)]\leq \frac{E[\frac{v}{v^*}]}{\alpha(k + \ln n)} && \text{Per disuguaglianza di Markov}\\
            = \frac{E[v]}{\alpha v^* (k + \ln n)} \leq \frac{v^* (k + \ln n)}{\alpha v^* (k + \ln n)} = \frac{1}{\alpha}
        \end{aligned}
    \end{equation}
\end{proof}
\begin{corollary}
    \label{setcr}
    Fissando il parametro $k = 3$:
    \begin{enumerate}
        \item il teorema \ref{setth} dice che la probabilità di ottenere una soluzione ammissibile
        $\geq 1-e^{-k} = 95\%$
        \item il teorema \ref{setth2} invece, dice che, fissando ad esempio $\alpha=2$, con R fattore di 
        approssimazione,  $P[R \geq 2(3 + \ln n)] \leq \frac{1}{2}$, quindi la probabilità di ottenere un fattore
        di approssimazione peggiore di $2(3 + \ln n)$ è inferiore a $\frac{1}{2}$
    \end{enumerate}
\end{corollary}

\begin{corollary}
    Con $k = 3$ c'è almeno il $45\%$ di probabilità che l'algoritmo emetta una 
    soluzione ammissibile con fattore di approssimazione al più $2(3+\ln n)$
\end{corollary}
\begin{proof}
    Considerando i due punti del corollario \ref{setcr} appena scritto come eventi indipendenti, 
    si calcola ora la probabilità che si verifichino insieme, che sarebbe quello che si vuole 
    ottenere dall'algoritmo.
    \begin{equation}
        \begin{aligned}
            P[E_1 \wedge E_2] = 1 - P[\bar{E_1} \lor \bar{E_2}]\\
            \leq 1 - (P[\bar{E_1}] + P[\bar{E_2}]) && \text{Per union bound}\\
            \leq 1 - (0.05 + 0.5) && \text{Per il corollario \ref{setcr}}\\
            = 1 - 0.55 = 0.45
        \end{aligned}
    \end{equation}    
\end{proof}

\subsubsection{MaxEkSat}
\label{eksat}
Questa riformulazione del problema aggiunge un costraint per quanto riguarda il numero di 
variabili per ogni condizione di sat.

Formalmente:
\emph{Input}: Formula CNF con $t$ clausole, ognuna di $k$ letterali, ogni variabile al massimo una volta
per clausola\\
\emph{Output}: Assegnamento che rende vera la CNF\\
\emph{Funzione obiettivo}: Numero di clausole risolte\\
\emph{Tipo}: max

\begin{remark}
    Il problema è NP completo per ogni $k \geq 3$
\end{remark}
\begin{theorem}
    Un assegnamento casuale delle variabili, porta ad una soluzione che 
    soddisfa in valore atteso $\geq \frac{2^k-1}{2^k}t$ clausole, ovvero, 
    sia $T$ la variabile aleatoria che rappresenta il numero di clausole soddisfatte: 
    $$E[T] =\frac{2^k-1}{2^k}t$$
\end{theorem}
\begin{proof}
    Siano $x_1, \dots, x_m$ le variabili che compaiono nella formula e 
    $c_1, \dots ,c_t$ le clausole.

    Si definiscono poi delle variabili aleatorie:
    \begin{enumerate}
        \item $X_1, \dots, X_m\thicksim\mathit{Unif}(\{0,1\}) $,dove $X_i$ è il valore assegnato a $x_i$
        \item $C_1, \dots, C_t$ ognuna delle quali vale 1 sse la clausola $c_i$ è soddisfatta
        \item $T$ numero di clausole soffisfatte, $T = \sum_{i = 1}^{t} C_i$
    \end{enumerate} 
    Si calcola ora il valore atteso di $T$.
    
    Vale che:
    \begin{equation}
        \begin{aligned}
            E[T] = \sum_E E[T|E]\cdot P[E] \\
            = \sum_{b_1 \in 2 } \dots \sum_{b_n \in 2 } P[X_1 = b_1, \dots, X_n = b_n]\\
            E[T|X_1 = b_1, \dots, X_n = b_n]\\
            = \sum_{b_1 \in 2 } \dots \sum_{b_n \in 2 } P[X_1 = b_1] \dots P[X_n = b_n]&& \text{var indipendenti}\\
            E[C_1 + \dots + C_t | X_1 = b_1, \dots, X_n = b_n] && \text{definizione di T}\\
            = \frac{1}{2^n}\sum_{b_1 \in 2 } \dots \sum_{b_n \in 2 } \sum_{i=1}^{t} E[C_i | X_1 = b_1, \dots, X_n = b_n]&& \text{valore di X, linearità v.a.}\\
        \end{aligned}
    \end{equation}
    Ora si nota che quel valore atteso finale, vale 0 oppure 1, in base al soddisfacimento o meno della clausola $C_i$.
    Ogni clausola è composta da $k$ variabili in disgiunzione, quindi, esiste un solo assegnamento delle $k$ che la rende falsa.

    Visto che si hanno $n$ variabili totali, esistono esattamente $2^{n-k}$ assegnamenti che rendono falsa la clausola, ovvero, 
    si individua un assegnamento delle $k$ variabili che rendono falsa la clausola $C_i$ e si considerano tutti i possibili assegnamenti 
    delle restanti $(n-k)$ variabili.
    
    Vale quindi che gli assegnamenti che rendono vera la clausola $C_i$ sono $2^n - 2^{n-k}$, ovvero il totale delle combinazioni 
    meno quelle che la rendono falsa.

    Proseguendo con l'equazione quindi, sostituendo quella sommatoria con quanto appena discusso:
    \begin{equation}
        \begin{aligned}
            = \frac{1}{2^n}\sum_{b_1 \in 2 } \dots \sum_{b_n \in 2 } \sum_{i=1}^{t} E[C_i | X_1 = b_1, \dots, X_n = b_n] \\
            = \frac{1}{2^n}t(2^n - 2^{n-k}) = \frac{t(2^n - 2^{n-k})}{2^n}\\
            =  \frac{t(2^n - 2^{n-k})}{2^n} * \frac{2^{k-n}}{2^{k-n}} = t\frac{2^k -1}{2^k}
        \end{aligned}
    \end{equation}
\end{proof}
\begin{theorem}
    \label{maxth}
    Dato un input per MaxEkSat con $n$ variabili, e $t$ clausole.
    \begin{equation}
        \begin{aligned}
            \forall j = 0, \dots, n,\;\exists b_1, \dots, b_j \in 2, t.c\\
            E[T|X_1 = b_1, \dots, X_j = b_j] \geq  t\frac{2^k -1}{2^k}
        \end{aligned}
    \end{equation}
\end{theorem}
\begin{proof}
    Si dimostra per induzione per $j$.\\
    $j = 0$, dimostrato dal teorema precendente.
    
    Passo induttivo:
    \begin{equation}
        \begin{aligned}
            E[T|X_1 = b_1, \dots, X_j = b_j] = \\
            E[T|X_1 = b_1, \dots, X_j = b_j, X_{j+1} = 0]\cdot P[x_{j+1} = 0]\\
            +\;E[T|X_1 = b_1, \dots, X_j = b_j, X_{j+1} = 1]\cdot P[x_{j+1} = 1] &&\text{Eventi disgiunti}\\
            = \frac{1}{2}\cdot E[T|X_1 = b_1, \dots, X_j = b_j, X_{j+1} = 0]\\
            + \; \frac{1}{2}\cdot E[T|X_1 = b_1, \dots, X_j = b_j, X_{j+1} = 1]
        \end{aligned}
    \end{equation}
    Per ipotesi induttiva, la media dei due valori attesi è almeno $t\frac{2^k -1}{2^k}$, 
    questo implica che uno dei due valori attesi è maggiore di quella quantitità, ovvero, 
    per qualche $b \in 2$:
    $$E[T|X_1 = b_1, \dots, X_j = b_j, X_{j+1} = b] \geq t\frac{2^k -1}{2^k}$$
\end{proof}
\begin{remark}
    La dimostrazione appena scritta suggerisce un approccio da seguire per l'algoritmo finale, 
    si deciderà iterativamente quale valore assegnare alla variabile $j$ in base al valore atteso di 
    $T$.
\end{remark}

\paragraph{MaxEkSat derandomizzato}
Seguendo le proprietà osservate in precedenza si formula il seguente algoritmo.

\begin{algorithm}[H]
    \SetAlgoLined
    \KwIn{$C_1 \wedge \dots \wedge C_t$, ognuna con $k$ letterali $\in x_1, \dots, x_n$}
    \KwResult{Assegnamento di variabili}
    $D \gets \emptyset$\\
    \For{$i = 1,\dots, n$}{
        $\Delta_0 \gets 0$\\
        $\Delta_1 \gets 0$\\
        $\Delta D_0 \gets \emptyset$\\
        $\Delta D_1 \gets \emptyset$\\
        \For{$j = 1, \dots, t$}{
            \If{$j \in D \lor x_i \notin C_j$}{
                $\mathit{continue}$
            }
            $h \gets |x_k \in C_j, k > i|$\\
            \If{$x_i$ positiva in $C_j$}{
                $\Delta_0 \gets \Delta_0 - \frac{1}{2^h}$\\
                $\Delta_1 \gets \Delta_1 + \frac{1}{2^h}$\\
                $\Delta D_1 \gets \Delta D_1 \cup \{j\} $\\
            }
            \Else{
                $\Delta_0 \gets \Delta_0 + \frac{1}{2^h}$\\
                $\Delta_1 \gets \Delta_1 - \frac{1}{2^h}$\\
                $\Delta D_0 \gets \Delta D_0 \cup \{j\} $\\    
            }
        }
        $b = (\Delta_0 \leq \Delta_1)$\\
        $X_i = b$\\
        $D \gets D \cup \Delta D_b$
    }
    \Return{$\mathit{Assegnamento}$}
    \caption{MaxEkSat Derandomizzato}
\end{algorithm}

L'algoritmo implementa quanto osservato nel teorema precedente.
In particolare si cerca di mantenere al passo $i$ vera questa proprietà
$$E[T|X_1 = b_1, \dots, X_{i-1} = b_{i-1}] \geq t\frac{2^k -1}{2^k}$$
Il valore atteso si può scomporre
\begin{equation}
    \begin{aligned}
        E[T|X_1 = b_1, \dots, X_{i-1} = b_{i-1}] = \\
        E[C_1|X_1 = b_1, \dots, X_{i-1} = b_{i-1}]\;+\\
        E[C_2|X_1 = b_1, \dots, X_{i-1} = b_{i-1}]\;+\\
        \dots\\
        + \; E[C_t|X_1 = b_1, \dots, X_{i-1} = b_{i-1}]  
    \end{aligned}
\end{equation}
Quello che si fa ora è confrontare i singoli valori attesi con gli stessi 
ottenuti dopo l'assegnamento della iesima variabile.

Si fa notare che, se una clausola $C_j$ appartiene a $D$, il valore atteso dopo 
l'assegnamento della iesima variabile non cambia, ovvero quella clausola è già decisa.

Il valore può cambiare per quelle che contengono la variabile $X_i$, consideriamone una
che contiene quella variabile, per esempio:
$$x_1 \lor \bar{x_4} \lor \dots \lor x_i \lor x_{i+1} \dots \lor x_n$$
Consideriamo $h$ variabili da $i$ a $n$ comprese, ci sono $\frac{2^h -1}{2^h}$ assegnamenti delle $h$ variabili che rendono vera 
la clausola.

Il valore atteso della clausola vale, prima dell'assegnamento di $x_i$
$$E[C_j] = \frac{2^{h} -1}{2^{h}}$$
dopo l'assegnamento varia in questo modo
\[
    E[C_j] = 
    \begin{cases}
        1\;\mathit{se}\; x_i = 1\\
        \frac{2^{h-1} -1}{2^{h-1}} \;\mathit{se}\; x_i = 0
    \end{cases}
\]
il cambiamento di valore atteso vale 
\[
    \Delta\;\mathit{valore}\;\mathit{atteso} = 
    \begin{cases}
        1 - \frac{2^h -1}{2^h} = \frac{2^h - 2^h + 1}{2^h} = +\frac{1}{2^h}\;\mathit{se}\; x_i = 1\\
        \frac{2^{h-1} -1}{2^{h-1}} - \frac{2^h -1}{2^h} = \frac{2^h -2 - 2^h + 1}{2^h} = -\frac{1}{2^h}\;\mathit{se}\; x_i = 0\\
    \end{cases}
\]
Nell'algoritmo non si sta facendo altro che calcolare il delta di valore atteso di $T$ in base al 
valore assegnato alla variabile $x_i$.\\
Per la dimostrazione del teorema \ref{maxth} uno degli assegnamenti di $x_i$ produrrà un valore atteso 
maggiore, quindi si sceglierà quello.

Si sta quindi procedendo nell'algoritmo accertandosi di mantenere vera la proprietà enunciata dal teorema \ref{maxth}.
\begin{theorem}
    Al termine dell'algoritmo almeno $\frac{2^k -1}{2^k}t$ clausole saranno vere.
\end{theorem}
\begin{corollary}
    L'algoritmo fornisce una $\frac{2^k -1}{2^k}$'approssimazione per MaxEkSat.
\end{corollary}
\begin{proof}
    Sia $t^*$ la soluzione ottima e sia $\bar{t}$ la soluzione prodotta dall'algoritmo.
    \begin{equation}
        \begin{aligned}
            \bar{t} \geq \frac{2^k -1}{2^k}t \implies t \leq \bar{t}\frac{2^k}{2^k -1} &&\text{Per il teorema \ref{maxth}}\\
            \frac{t^*}{\bar{t}} \leq \frac{t}{\bar{t}} \leq \frac{\bar{t}\frac{2^k}{2^k -1}}{\bar{t}} \leq \frac{2^k}{2^k -1}
        \end{aligned}
    \end{equation}
\end{proof}
\begin{corollary}
    MaxE3Sat è $\frac{7}{8}$-approssimante.
\end{corollary}