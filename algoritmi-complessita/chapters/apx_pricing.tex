\subsection{Tecniche di pricing}
L'idea è quella di attribuire ad ogni elemento da inserire in 
una soluzione un costo. I costi permettono di scegliere gli elementi 
più vantaggiosi e di analizzare il tasso di approssimazione di questi algoritmi.

\subsubsection{Minimum Set Cover}
\label{msetcover}
Nel problema di Set Cover l'obiettivo è quello di coprire l'universo $U$, 
utilizzando seubset di esso che hanno un costo. Si vuole ottenere il costo
minimo, dato come somma dei costi dei set che si sono scelti.

Formalmente: \\
\emph{Input}: $s_1, \dots, s_m$, $\bigcup_{i=1}^m s_i = U$, $|U| = n$\\
\emph{Output}: $C = \{s_1, \dots, s_n\}$, tali che, $\bigcup_{s_i \in C} s_i = U$\\
\emph{Costo}: $w = \sum_{s_i \in C} w_i$\\
\emph{Tipo}: min\\

\paragraph{Funzione armonica}
La funzione armonica è definita come: 
\begin{equation}
    \begin{aligned}
        H: \mathbb{N}^{>0} \rightarrow \mathbb{R}\\
        H(n) = \sum_{i = 1}^{n} \frac{1}{i}  
    \end{aligned}
\end{equation}
Vale la seguente proprietà: 
% \begin{equation}
%     \begin{aligned}
%         H(n) \leq 1 + \int_{1}^{n} \frac{1}{x} \,dx \leq 1 +
%         \big[ \ln n\big]_1^n = 1 + \ln n\\
%         \ln (n+1) \leq H(n) \leq 1 +\ln n
%     \end{aligned}
% \end{equation}
$$\ln (n+1) \leq H(n) \leq 1 +\ln n$$

\paragraph{Price Set Cover}
L'algoritmo effettua ad ogni iterazione una scelta greedy, 
si sceglie l'insieme che minimizza il rapporto tra prezzo e copertura 
dell'universo.

\begin{algorithm}[H]
    \SetAlgoLined
    \KwIn{$s_1, \dots, s_m$, $w_0, \dots, w_n$}
    \KwResult{Scelta di sottoinsiemi che copre l'universo}
     $R \gets U$\\
     $C \gets \emptyset$\\
     \While{$R \neq \emptyset$}{
         $S_i \gets \min([s_1, \dots, s_m], \frac{w_i}{|S_i\cap R|})$\\
         $C.add(S_i)$\\
         $R \gets R \setminus S_i$
     }
     \Return{$C$}
     \caption{PriceSetCover}
\end{algorithm}

\begin{remark}
    \label{oss1set}
    Il costo della soluzione equivale a $$w = \sum_{s\in U}c_s$$ ovvero la somma dei costi 
    degli insiemi scelti.
\end{remark}
\begin{remark}
    \label{oss2set}
    Per ogni $k$, il costo degli elementi in $s_k$, ottengo 
    $$\sum_{s \in S_k}C_s \leq H(|S_k|) \cdot  W_k$$
\end{remark}
\begin{proof}
    Sia $S_k = \{s_1, \dots, s_d\}$ un insieme tra quelli da scegliere, e siano 
    i suoi elementi elencati in ordine di copertura\footnote{Per chiarezza, l'insieme $S_k$ non verrà scelto, 
    ma i suoi elementi saranno coperti da altri insiemi che intersecano con esso.}.

    Consideriamo ora l'istante in cui si copre $s_j$ tramite un qualche insieme $S_h$.
    Si può notare che, visto che gli elementi sono in ordine di copertura: 
    $$R \supseteq \{s_j, s_{j+1} \dots, s_d \}$$
    Inoltre, visto che gli elementi di $S_k$ sono in ordine di copertura: 
    $$|S_k \cap R| \geq d-j+1$$
    Riguardo al costo dell'elemento $j$, e in generale per tutti i $j$, vale:
    \begin{equation}
        \begin{aligned}
            C_{s_j} &= \frac{W_h}{|S_h \cap R|} \leq \frac{W_k}{|S_k \cap R|} && \text{\emph{h} minimizza quel rapporto}\\
            &\leq \frac{W_k}{d-j+1} && \text{equazione precedente}\\
        \end{aligned}
    \end{equation}
    Considerando ora tutti gli elementi di $S_k$: 
    \begin{equation}
        \begin{aligned}
            \sum_{s\in S_k} C_s \leq \sum_{j=1}^{d}\frac{W_k}{d-j+1} = \frac{W_k}{d} + \frac{W_k}{d-1} \dots && \text{La relazione vale per tutti i j}\\
            = W_k(1 + \frac{1}{2}, \dots, \frac{1}{d}) = H(d)W_k = H(|S_k|)W_k && \text{Sviluppo e raccolgo, ottengo l'oss.}
        \end{aligned}
    \end{equation}
\end{proof}
\begin{theorem}
    Price Set Cover fornisce una $H(M)$-approssimazione per Set Cover, dove $M=\max_i|S_i|$
\end{theorem}
\begin{proof}
    Sia il peso della soluzione ottima $$w^* = \sum_{S_i \in C^*} w_i$$
    Per l'osservazione \ref{oss2set} vale che: 
    $$w_i \geq \frac{\sum_{s \in S_i}C_s}{H(|S_i|)} \geq \frac{\sum_{s \in S_i}C_s}{H(M)}$$
    Visto che gli $s_i \in C^*$ sono una copertura, per l'osservazione \ref{oss1set}:
    $$\sum_{S_i \in C^*}\sum_{s \in S_i}C_s \geq \sum_{s \in U} C_s = w$$
    Inoltre, vale che, sfruttando le due disequazioni appena scritte: 
    \begin{equation}
        \begin{aligned}
            w^* = \sum_{S_i \in C^*} w_i \geq \sum_{S_i \in C^*}\frac{\sum_{s \in S_i}C_s}{H(M)} \geq \frac{w}{H(M)}\\
            \implies \frac{w}{w^*} = H(M)
        \end{aligned}
    \end{equation}
\end{proof}
\begin{corollary}
    Price Set Cover fornisce una $O(\log n)$-approssimazione, non quindi costante 
    come gli algoritmi precedenti.
\end{corollary}
\begin{remark}
    L'analisi è tight, non esiste un algoritmo migliore, quindi
    Price Set Cover $\notin$ APX, bensì, $\in \log(n)$-APX, una classe in cui 
    si accetta un'approssimazione che peggiora logaritmicamente nell'input.
    Esistono varie $f$-APX.
\end{remark}
\begin{proof}
    Per dimostrare la tightness, ecco un esempio in cui Price Set Cover va male.

    Consideriamo l'insieme $S$ di set tra cui scegliere così formato:
    \begin{enumerate}
        \item Due insiemi grandi $\frac{n}{2}$ che uniti coprono tutti gli elementi, 
        di costo $1+\epsilon$
        \item Un insieme che copre $\frac{n}{4}$ elementi degli insiemi 1 e 2 al punto 1, di costo 1
        \item Un insieme che copre $\frac{n}{8}$ elementi degli insiemi 1 e 2 al punto 1, di costo 1
        \item Un insieme che copre $\frac{n}{16}$ elementi degli insiemi 1 e 2 al punto 1, di costo 1\\
        \dots
    \end{enumerate}
    Al primo passo, si sceglie l'insieme del punto 2, visto che
    il suo costo equivale a $\frac{1}{\frac{n}{2}} = \frac{2}{n}$, mentre il costo 
    di entrambi gli insimemi al punto 1, $\frac{1+\epsilon}{\frac{n}{2}} = \frac{2+2\epsilon}{n}$

    Al secondo passo, si preferisce ai primi due l'insieme al punto 3, il calcolo è simile al punto precedente.\\
    \dots

    Il costo che si ottiene è $w = \log n$
    La soluzione ottima sarebbe quella di prendere al passo 1 i primi due insiemi, in modo da coprire l'intero 
    universo, ovvero $w^* = 2 + 2\epsilon$.
\end{proof}

\subsubsection{Vertex Cover}
Il problema di Vertex Cover consiste nel trovare una copertura di 
vertici in un grafo tale per cui per ogni vertice, una delle due estremità è contenuta nella copertura.

Formalmente: \\
\emph{Input}: $G=(V,E), w_i \in \mathbb{Q}^{>0}, \forall i \in V$\\
\emph{Output}: $X \subseteq V, \forall xy \in E, x \in X \vee y \in X$\\
\emph{Costo}: $w = \sum_{i \in X} w_i$\\
\emph{Tipo}: min\\

Consideriamo la versione di decisione del problema, ovvero $\hat{\mathit{VertexCover}}$, cioè, 
posso trovare una copertura che ha peso minore di $\theta$?\\

Vale che $\hat{\mathit{VertexCover}} \leq_{p} \hat{\mathit{SetCover}}$
Per effettuare il passaggio, bisogna passare dall'input del primo al secondo
$$G=(V,E), w_i \in \mathbb{Q}^{>0}, \theta \longrightarrow f \longrightarrow S_1, \dots, S_m, \bar{W_1}, \dots, \bar{W_m},\bar{\theta}$$
La funzione $f$ funziona così: 
\begin{equation}
    \begin{aligned}
        S_i = \{ e \in E, i \in e \} && \text{Per ogni vertice considero i suoi vicini}\\
        U = E, \bar{W_i} = W_i, \bar{\theta} = \theta
    \end{aligned}
\end{equation}
\begin{remark}
    La funzione $f$ può essere utilizzata per mappare anche VertexCover in SetCover di ottimizzazione, 
    visto che non stravolge il problema.\\
    La trasformazione appena discussa quindi può essere utilizzata per fornire una $\log n$-approssimazione
    per VertexCover di ottimizzazione.
\end{remark}
\paragraph{Price Vertex Cover}
Prima di definire l'algoritmo vero e proprio, sono necessari alcuni passaggi preliminari.
L'idea si basa sul fatto che gli archi sono intenzionati a comprare uno dei due estremi,
ad un certo prezzo.
Un nodo si vende, se la somma delle offerte degli archi incidenti arriva al suo $W_i$.
\begin{definition}
    Un insieme di offerte si dice equo per un vertice sse:
    $$\sum_{e\in E, i \in e}P_e \leq W_i$$
    Ovvero se le offerte $P_e$ degli archi che incidono sul vertice $i$ 
    non superano il suo costo, ovviamente devono raggiungerlo per comprare 
    il vertice.    
\end{definition}
\begin{lemma}
    \label{pscl1}
    Se $P_e$ è equo, allora
    $$\sum_{e \in E} P_e \leq w^*$$
\end{lemma}
\begin{proof}
    La definizione di equità implica che
    $$\forall i \in V \sum_{e\in E, i \in e}P_e \leq W_i$$
    Consideriamo ora la somma delle disequazioni per la soluzione ottima
\begin{equation}
    \begin{aligned}
        \sum_{i \in S^*} \sum_{e\in E, i \in e}P_e &\leq \sum_{i \in S^*}W_i\\
        \sum_{e \in E} P_e &\leq \sum_{i \in S^*} \sum_{e\in E, i \in e}P_e \leq W^*
    \end{aligned}
\end{equation}
Questo vale perchè la seconda parte della disequazione considera potenzialmente più volte alcuni lati.
\end{proof}
\begin{definition}
    $P_e$ è stretto sul vertice $i$ sse
    $$\sum_{e \in E, i \in e} P_e = W_i$$
\end{definition}
Ecco ora l'algoritmo.

\begin{algorithm}[H]
    \SetAlgoLined
    \KwIn{$G=(V,E), W_i \forall i \in V$}
    \KwResult{Cover per il grafo}
     $P_e \gets 0, \forall e \in E$\\
     \While{$\exists ij \in E, P_e$ non stretto su i o su j}{
        $\bar{e} \gets \bar{i}\bar{j}$\\
        $\Delta \gets \min(W_{\bar{i}}-\sum_{e \in E, \bar{i} \in e}P_e, W_{\bar{j}}-\sum_{e \in E, \bar{j} \in e}P_e,)$\\
        $P_e \gets P_e + \Delta$
     }
     \Return{$\{ i| P_e$ stretto su $i\}$}
     \caption{PriceVertexCover}
\end{algorithm}
L'idea su cui si basa l'algoritmo è quella che, se un arco non sta offrendo abbastanza per i vertici, allora 
aumenta la sua offerta, del minimo per soddisfare uno dei due nodi.
\begin{lemma}
    \label{pscl2}
    Price Set Cover crea un peso $$W \leq 2\sum_{e \in E}P_e$$
\end{lemma}
\begin{proof}
    Il costo finale di PSC è $W = \sum_{i \in S}W_i$, ovvero il costo 
    dei vertici in output. Vale che, per la definizione di strettezza:
    $$W = \sum_{i \in S}W_i = \sum_{i \in S} \sum_{e\in E, i \in e}P_e$$
    Nella seconda sommatoria un lato compare una o due volte, se rispettivamente 
    una o due estremità appartengono ad $S$.
\end{proof}
\begin{theorem}
    Price Set Cover è $2$-approssimante per Set Cover.
\end{theorem}
\begin{proof}
    \begin{equation}
        \begin{aligned}
            \frac{w}{w^*} &\leq \frac{2\sum_{e \in E}P_e}{w^*} && \text{Per il lemma \ref{pscl2}}\\
            \frac{w}{w^*} &\leq \frac{2\sum_{e \in E}P_e}{\sum_{e \in E} P_e } \leq 2 && \text{Per il lemma \ref{pscl1}}
        \end{aligned}
    \end{equation}
\end{proof}
\begin{remark}
    Non esistono ad oggi algoritmi migliori per Price Set Cover.
\end{remark}

\subsubsection{Disjoint Paths}
Dato un grafo orientato, esistono $K$ sorgenti e target, l'obiettivo è trovare $K$ cammini per le 
coppie di sorgenti e destinazioni. Ogni arco può essere usato al più $C$ volte.

Formalmente: \\
\emph{Input}: $G=(V,E)$ orientato, $s_1, \dots, s_k$ sorgenti, $t_1, \dots, t_k$ destinazioni $\in V$
$C \in \mathbb{N}^{>0}$\\
\emph{Output}: $I = \{1, \dots, K\}, \forall i \in I,$ cammino $\Pi_i$, tale che nessun arco è utilizzato più di $C$ volte\\
\emph{Funzione obiettivo}: $|I|$\\
\emph{Tipo}: max\\

L'idea su cui si basa l'algoritmo è quella di allungare un arco tanto quanto è utilizzato, 
in modo da scoraggiare il suo utilizzo.\\
È quindi definita una funzione di lunghezza: 
$$l : E \longrightarrow \mathbb{R}^{>0}$$
Dato un cammino come sequenza di nodi, la lunghezza del cammino sarà data dalla lunghezza su ogni coppia di consecutivi.

\paragraph{Greedy Paths with capacity}
L'algoritmo considera come input, oltre a quello per Disjoint Paths, un parametro
$\beta > 0$, che indica il fattore di allungamento degli archi.

\begin{algorithm}[H]
    \SetAlgoLined
    \KwIn{$G=(V,E),s_1, \dots, s_k, t_1, \dots, t_k, \beta > 0$}
    \KwResult{Cammini da sorgenti a destinazioni}
     $I \gets \emptyset$\\
     $P \gets 0$\\
     $l(xy) \gets 1, \forall xy \in E$\\
     \While{$\Pi \gets$ shortest path from $s_i$ to $t_i$, $i \notin I$}{
         $I \gets I \cup i$\\
         $P \gets P \cup \Pi$\\
         \For{$e\in \Pi$}{
            $l(e) \gets l(e)*\beta$
            \If{$e$ used by C paths}{
                $\mathit{remove}(e)$
            }
         }
     }
     \Return{$I,P$}
     \caption{GreedyDisjointPaths}
\end{algorithm}

Consideriamo ora l'evolversi dei cammini nell'algoritmo, in particolare un cammino
può essere:
\begin{itemize}
    \item Inutile, se unisce una sorgente e destinazione già collegate
    \item Utile ma ostruito, ovvero collega una sorgente e destinazione sconnesse, 
    ma passa per un arco utlizzato da più di $C$ cammini
\end{itemize}

\begin{definition}
    Un cammino $\Pi$ è corto rispetto a una certa lunghezza $l$ sse
    $l(\Pi) < \beta^c$, lungo altrimenti.
\end{definition}
\begin{remark}
    \label{ossdisjoint}
    Se sono presenti cammini utili non ostruiti e corti, l'algoritmo sceglie un cammino
    di questa natura, inoltre, nelle prime fasi dell'algoritmo, ovvero quando sono presenti cammini
    di questo tipo, si può evitare di eliminare archi.
\end{remark}
\begin{proof}
    Se esite un cammino corto, la sua lunghezza è per forza minore di $\beta^c$ quindi nessun arco 
    è stato utilizzato $C$ volte.
\end{proof}

Tenendo presente quanto detto nell'osservazione \ref{ossdisjoint}, teniamo ora in considerazione 
la funzione di lunghezza $\bar{l}$ nell'esatto istante in cui finiscono i cammini corti.
Inoltre, $\bar{I}, \bar{P}$ saranno le sorgenti, destinazioni collegate e i rispettivi cammini 
in quell'istante.

\begin{lemma}
    \label{ldis1}
    Se $i \in I^* \setminus I$, allora $\bar{l}(\Pi_i^*) \geq \beta^c$
\end{lemma}
\begin{proof}
    Se valesse $\bar{l}(\Pi_i^*) < \beta^c$, il cammino $\Pi_i^*$ sarebbe corto e
    utile, in quanto collega una coppia ancora non collegata, dato che non appartiene a $I$.
    Quindi, non esisterebbe motivo per non selezionarlo.
\end{proof}

\begin{lemma}
    \label{ldis2}
    Vale che $\sum_{e \in E}\bar{l}(e) \leq \beta^{c+1}|\bar{I}| + m$.
\end{lemma}
\begin{proof}
    All'inizio dell'algoritmo, $\sum_{e \in E} l(e) = m$.
    Consideriamo ora le due funzioni di lunghezza dopo la selezione di un cammino
    $$l_1 \longrightarrow \Pi \longrightarrow l_2$$
    \[ 
        l_2(e) = 
        \begin{cases}
        \beta l_1(e) & \mathit{se}\;e \in \Pi\\
        l_1(e) & \mathit{se}\;e \notin \Pi
     \end{cases}
    \]
    Consideriamo ora la differenza delle lunghezze
    \begin{equation}
        \begin{aligned}
            \sum_{e \in E} l_1(e) - \sum_{e \in E} l_2(e)&= \sum_{e \in E} (l_1(e) - l_2(e))\\
            &= \sum_{e \in \Pi} (l_1(e) - l_2(e)) = \sum_{e \in \Pi} (\beta l_1(e) - l_2(e))\\
            &= \sum_{e \in \Pi} (\beta -1)l_1(e) = (\beta -1) \sum_{e \in \Pi} l_1(e)\\
            &= (\beta -1)l(\Pi) \leq (\beta -1)\beta^c \leq \beta^{c+1}
        \end{aligned}
    \end{equation}
    Ovvero, le uniche lunghezze che cambiano sono quelle relative al cammino $\Pi$ poi, 
    il cammino selezionato è corto, quindi la sommatoria delle lunghezze sul cammino è minore di 
    $\beta^{c}$.
    Considerando l'inizio della dimostrazione, visto che ad ogni passo la lunghezza aumenta di al 
    massimo $\beta^{c+1}$, il lemma vale.
\end{proof}
\begin{remark}
    \label{oss3dis}
    Valgono:
    \begin{equation}
        \begin{aligned}
            \sum_{i \in I^* \setminus I} \bar{l}(\Pi_i^*) \geq \beta^c|I^* \setminus I| && \text{dal lemma \ref{ldis1}}\\
            \sum_{i \in I^* \setminus I} \bar{l}(\Pi_i^*) \leq c\sum_{e \in E} \bar{l}(e) \leq c(\beta ^{c+1}|\bar{I}| + m)
            && \text{per il lemma \ref{ldis2}}
        \end{aligned}
    \end{equation}
\end{remark}

Mettendo tutto insieme: 
\begin{equation}
    \begin{aligned}
        \beta^c|I^*| &= \beta^c|I^* \setminus I| + \beta^c|I^* \cap I|\\
        &\leq \sum_{i \in I^* \setminus I} \bar{\Pi_i^*} + \beta^c && \text{per il punto 1 di oss. \ref{oss3dis}, e maggioro}\\
        &\leq c(\beta^{c+1}|\bar{I}| + m) + \beta^c |I|&& \text{per il punto 2 di oss. \ref{oss3dis}}\\
        &\leq c(\beta^{c+1}|I| + m) + \beta^c |I|&& \text{I è sovrainsieme per I barra}\\
        |I^*| &\leq c\beta|I| + c\beta^{-c}m + |I|&& \text{divido per beta elevato alla c}\\
        |I^*| &\leq c\beta|I| + c\beta^{-c }m|I| + |I|&& \text{moltiplico per la cardinalità quello in mezzo}\\
        \frac{|I^*|}{|I|} &\leq c\beta + c\beta^{-c}m + 1&& \text{divido per cardinalità di I}
    \end{aligned}
\end{equation}
Poniamo ora $\beta = m^{\frac{1}{c+1}}$ si ottiene:
\begin{equation}
    \begin{aligned}
        \frac{|I^*|}{|I|} &\leq c(m^{\frac{1}{c+1}} + m^{\frac{-c+c+1}{c+1}}) + 1\\
        &\leq 2cm^{\frac{1}{c+1}} +1
    \end{aligned}
\end{equation}

Al variare di $C$ varia il grado di approssimazione dell'algoritmo.
\begin{center}
    \begin{tabular}{ c c  }
     C & Bound approssimazione \\
     \hline 
     1 & $2\sqrt[2]{m} + 1$ \\  
     2 & $4\sqrt[3]{m} + 1$ \\  
     3 & $6\sqrt[4]{m} + 1$ \\  
    \dots & \dots
    \end{tabular}
\end{center}
\begin{theorem}
    Greedy Disjoint Paths con  $\beta = m^{\frac{1}{c+1}}$ fornisce una $(2cm^{\frac{1}{c+1}} +1)$-approssimazione.
\end{theorem}
\begin{remark}
    L'algoritmo non è buono, il fattore di approssimazione è pessimo, infatti, funziona decentemente solo 
    se il numero di coppie da collegare $K >> 2\sqrt[2]{m}$.\\
    Si fa notare infine che l'algoritmo funziona anche se una coppia è da collegare più volte.
\end{remark}

\subsection{Tecniche di arrotondamento}
In questa sezione si presenta il problema di programmazione lineare e si 
ricorre alla tecnica dell'arrotondamento di una soluzione reale per 
individurarne una intera.

Essa si rivela possibile soluzione per il problema di copertura dei vertici.

\subsubsection{Linear programming}
Un problema di programmazione lineare è formato da una funzione obiettivo e dei vincoli.
\paragraph{Esempio}
Funzione obiettivo: $$\mathit{min}\;\;3x_1 + |7x_2 - 4x_3|$$
Vincoli:
\[ 
    \begin{cases}
        x_1 + x_2 \leq 3\\
        3x_3 - x1 \geq 7\\
        x1 \geq 0\\
        x2 \geq 0\\
        x3 \geq 0\\
    \end{cases}
\]

Più formalmente:\\
\emph{Input}: $A \in \mathbb{Q}^{m\times n}, b \in \mathbb{Q}^m, c \in \mathbb{Q}^n$\\
\emph{Output}: $x \in \mathbb{R}^n, Ax \geq b$\\
\emph{Funzione obiettivo}: $C^T \cdot x$, valore della funzione obiettivo\\
\emph{Tipo}: max, min\\

\begin{remark}
    Il problema di  LP $\in $ PO, spesso però si utilizzano
    algoritmi worst case esponenziali.
\end{remark}

\paragraph{Integer linear programming}
Cambiando il vincolo della soluzione ai soli interi, il problema diventa NPO completo.

Ovvero:\\
\emph{Input}: $A \in \mathbb{Q}^{m\times n}, b \in \mathbb{Q}^m, c \in \mathbb{Q}^n$\\
\emph{Output}: $x \in \mathbb{Z}^n, Ax \geq b$\\
\emph{Funzione obiettivo}: $C^T \cdot x$, valore della funzione obiettivo\\
\emph{Tipo}: max, min\\

\begin{theorem}
    $\hat{\mathit{VertexCover}} \leq_p \hat{\mathit{ILP}}$ 
\end{theorem}
\begin{proof}
    Consideriamo un input per il problema di $\hat{\mathit{VertexCover}}$, ovvero $$(G=(E,V), w_i \forall i \in V, \theta)$$
    Ci si chiede se: $$\exists X \subseteq V | \forall xy \in E, x \in X \mathit{or} y \in X, \sum_{i\in X} w_i \leq \theta$$
    Costruisco ora il problema di programmazione linerare in questo modo:
    $$x_i \forall i \in V$$
    Vincoli:
    \[
        2(n+m)\;\mathit{volte}
        \begin{cases}
            x_i \geq 0 \;\forall i \in V\\
            x_i \leq 1 \;\forall i \in V\\
            x_i + x_j \geq 1 \;\forall ij \in E\\
        \end{cases}
    \]
    Obiettivo,:
    $$\min \sum_{i\in V} w_ix_i$$

    Assumiamo ora $\bar{x_i}$ soluzione ammissibile per ILP. Sia $X = \{ i | \bar{x_i} = 1\}$.
    Visto che c'è il vincolo sugli archi, $X$ è soluzione per $\hat{\mathit{VertexCover}}$.

    Quindi, risolvendo in tempo polinomiale ILP si risolverebbe VertexCover, ma 
    visto che $\hat{\mathit{VertexCover}}$ è NP completo, anche ILP è NP completo.
\end{proof}

\subsubsection{Vertex Cover con arrotondamento}
Consideriamo ora la trasformazione mostrata al teorema precedente, ma invece
che un problema di programmazione lineare intera consideriamone uno reale, $\Pi^\prime$.

Esiste il problema di fare il passo inverso. Con il problema intero infatti 
bastava considerare le variabili con valore 1, ora sono reali, bisogna quindi trovare un 
modo di capire se una variabile (nodo), fa o no parte del cover.

Si sta di fatto rilassando il problema, chiamiamo $\Pi^\prime_{\mathit{INT}}$ il problema 
intero, rispetto a quello reale si avrà che:\footnote{Si sta considerando un problema di minimo}
\begin{equation}
    \begin{aligned}
        x_i^*, w^*, \tilde{x_i^*}, \tilde{w^*} && \text{Soluzioni rispettivamente reale e intera}\\
        \tilde{w^*} \geq w^* && \text{Poichè il dominio di ricerca reale è maggiore}\\
    \end{aligned}
\end{equation}

Si definisce ora la seguente trasformazione:
\[
    r_i = 
    \begin{cases}
        1\;\mathit{se}\;x_i^* \geq \frac{1}{2}\\
        0\;\mathit{se}\;x_i^* < \frac{1}{2}
    \end{cases}
\]
\begin{remark}
    La trasformazione fornisce una soluzione ammissibile per $\Pi^\prime_{\mathit{INT}}$.
\end{remark}
\begin{proof}
    Bisogna dimostrare che:
    $$\forall ij \in E, r_i + r_j \geq 1$$
    Sapendo che:
    $$\forall ij \in E, x_i^* + x_j^* \geq 1$$
    Supponiamo per assurdo che valgano entrambi 0, ma allora: 
    $$x_i^* < \frac{1}{2}, x_j^* < \frac{1}{2}$$
    che contraddice la seconda assunzione.
\end{proof}
\begin{theorem}
    Arrotondando la soluzione reale con la trasformazione mostrata, 
    si fornisce una 2-approssimazione per $\hat{\mathit{VertexCover}}$.
\end{theorem}
\begin{proof}
    Valgono:
    \begin{equation}
        \begin{aligned}
            r_i \leq 2x_i^* && \text{Per definizione della trasformazione}\\
            \sum_{i\in V} w_ir_i \leq 2 \sum_{i\in V} w_ix_i^* = 2w^* \leq 2\tilde{w} &&\text{La soluzione arrotondata è al più il doppio}
        \end{aligned}
    \end{equation}
\end{proof}
\begin{remark}
    La soluzione non migliora Price Vertex Cover, anch'esso 2-approssimante, è però interessante l'approccio utilizzato.
\end{remark}

\subsection{Altri esempi}
In questa sezione si considera un problema che non ricade nelle techine viste in precedenza.

\subsubsection{Traveling Salesman Problem}
Prima di introdurre il problema, si introducono alcuni concetti preliminari 
tramite un secondo problema, quello dei ponti di Konisberg.
\paragraph{Problema dei ponti di Konisberg}
Il problema considera una città, costruita a cavallo di un fiume, 
e risponde alla domanda, è possibile effettuare un cammino che passi 
una e una sola volta su tutti i ponti della città?

La specifica istanza del problema si modella su un multigrafo \footnote{Grafo in cui possono esistere piu archi per la stessa coppia
sorgente destinazione} e la risposta è no.

Il problema generale consiste nell'individuare un \emph{circuito Euleriano} nel grafo.

\begin{theorem}
    \label{teulerian}
    Un multigrafo ammette un circuito Euleriano sse tutti i vertici 
    hanno grado\footnote{Numero di archi incidenti} pari.
\end{theorem}
\begin{proof}
    Si sta dimostrando l'implicazione da sinistra a destra, 
    ovvero se i nodi hanno tutti grado pari esiste un circuito Euleriano.

    Concettualmente in un cammino, non posso rimanere bloccato, 
    visto che tutti i nodi hanno grado pari. Nello specifico, in un cammino può succedere:
    \begin{enumerate}
        \item Creo un ciclo con la partenza, in questo caso riparto
        \item Incido su un nodo del cammino, posso continuare, visto che ha grado pari
    \end{enumerate}
    È facile intuire che è possibile continuare cosi finchè non si visitano tutti gli archi del multigrafo.
\end{proof}
\begin{lemma}
    \label{lstrette}
    In ogni grafo non orientato il numero di nodi con grado dispari è pari.
\end{lemma}
\begin{proof}
    Consideriamo la somma dei gradi:
    $$\sum_{x\in V} d(x) = 2m$$
    Entrambi i membri sono pari.
    Nella somma dei gradi, se considero un grado pari, la parità della somma non cambia, 
    mentre, se aggiungo un grado dispari, la somma diventa dispari.

    È facile intuire che serve necessariamente un numero pari di gradi dispari per ottenere un 
    numero pari nel primo membro dell'equazione.
\end{proof}

Terminata la digressione sul problema dei ponti, si introduce ora formalmente 
il problema del commesso viaggiatore.

\emph{Input}: $G = (V,E)$ non orientato, $\delta_e \in \mathbb{Q}^{>0} \forall e\in E$\\
\emph{Output}: ciclo che passa per tutti i vertici, ovvero un ciclo Hamiltoniano\\
\emph{Funzione obiettivo}: $\delta(\Pi) = \sum_{e\in E}\delta_e$\\
\emph{Tipo}: min

\subsubsection{Traveling Salesman Problem su cricche}
Il problema del commesso viaggiatore spesso si considera su 
una cricca\footnote{Grafo con tutti gli archi possibili $K_V = (N, \binom{n}{2} )$}.

È possibile trasformare un grafo normale in una cricca aggiungendo 
tutti gli archi che mancano con peso equivalente alla somma di tutti i vertici più uno.
La soluzione ottima non varia.

\subsubsection{Traveling Salesman Problem metrico}
Il problema in questa versione considera come input una cricca, 
inoltre vale la distanza triangolare, ovvero: 
$$\forall i, j, k \in V, \delta_{ij} \leq \delta_{ik}+\delta_{kj}$$

\begin{theorem}
    TSP metrico è NPO completo.
\end{theorem}

\paragraph{Algoritmo di Christofides}
L'algoritmo procede trovando un albero di copertura minimo e successivamente lavorando attorno 
a quello per individuare una soluzione.

\begin{algorithm}[H]
    \SetAlgoLined
    \KwIn{$G=(V,\binom{V}{2}), \delta_e \in \mathbb{Q}^{>0} \forall e\in E$ che soddisfa la triangolare}
    \KwResult{Soluzione per TSP metrico}
     $T \gets \mathit{MinimumSpanningTree(G)}$\\
     $D \gets \{v \in T, d(v)\; \mathit{dispari}\}$\\
     $M \gets \mathit{MinimumPerfectMatching(D)}$\\ 
     $H \gets T \cup M$\\
     $\Pi \gets \mathit{EulerianCircuit(H)}$\\
     $\Pi \gets \mathit{Hamiltonian(\Pi)}$\\
     \Return{$\Pi$}
     \caption{Christofides}
\end{algorithm}
\begin{remark}
    Vale che:
    \begin{enumerate}
        \item D ha cardinalità pari per il lemma \ref{lstrette}
        \item H è un multigrafo con tutti gradi pari, quindi esiste un circuito euleriano
        \item Il circuito euleriano si costruisce come nella dimostrazione del teorema \ref{teulerian}
        \item Per trovare un ciclo hamiltoniano da $\Pi$ basta procedere nel cammino saltando i nodi già visitati,
        tanto è una cricca e ci sono tutti gli archi
    \end{enumerate}
\end{remark}
\begin{lemma}
    \label{tspl1}
    Il peso dello spanning tree è al più uguale al peso della soluzione ottima. $$\delta(T) \leq \delta^*$$
\end{lemma}
\begin{proof}
    Sia $\Pi^*$ la soluzione ottima.
    Rimuovendo un qualunque lato, si ottiene un albero.
    Vale che, visto che $T$ è il minimo albero di copertura:
    $$\delta(T) \leq \delta(\Pi^* - e) \leq \delta(\Pi^*) = \delta^*$$
\end{proof}
\begin{lemma}
    \label{tspl2}
    Il costo del perfect matching è al massimo la metà della soluzione ottima:
    $$\delta(M) \leq \frac{1}{2}\delta^*$$
\end{lemma}
\begin{proof}
    Sia $\Pi^*$ la soluzione ottima.
    Siano $D$ i vertici con grado dispari considerati nell'algoritmo, che ovviamente appartengono alla soluzione
    ottima.

    Si considera ora il ciclo costruito unendo i nodi di $D$ e siano $M_1$ e $M_2$ due gruppi di lati, 
    che contengono in maniera alternata i lati del ciclo.
    Sia $M_1$ che $M_2$ sono matching.\\
    Vale che:
    \begin{equation}
        \begin{aligned}
            \delta(M) &\leq \delta(M_1)\\
            \delta(M) &\leq \delta(M_2)\\
            2\delta(M) &\leq \delta(M_1) + \delta(M_2) \leq \delta^*
        \end{aligned}
    \end{equation}
    L'ultima disequazione segue dal fatto che, collegando i nodi che sono in $D$ si sta cortocircuitando il cammino, 
    ovvero, due nodi $n_1$ e $n_2 \in D$ sono collegati direttamente nel matching ma potrebbero non esserlo 
    nella soluzione ottima, segue quindi che per la triangolare, la distanza tra i due è al piu uguale a quella nel cammino 
    ottimo.\\
    Proseguendo con la disequazione si ottiene che
    $$\delta(M) \leq \frac{1}{2}\delta^*$$
\end{proof}
\begin{theorem}
    L'algoritmo di Christofides è $\frac{3}{2}$-approssimante per TSP metrico.
\end{theorem}
\begin{proof}
    Sia $\tilde{\Pi}$ il cammino ottimo.
    Vale che:
    \begin{equation}
        \begin{aligned}
            \delta(\tilde{\Pi}) \leq \delta(\Pi) &= \sum_{e \in M}\delta_e + \sum_{e \in T}\delta_e\\
            \delta(\tilde{\Pi}) &\leq \frac{1}{2}\delta^* + \delta^* \leq \frac{3}{2}\delta^* && \text{Per i lemmi \ref{tspl1} e \ref{tspl2}}\\
        \end{aligned}
    \end{equation}
\end{proof}
\begin{theorem}
    L'analisi di Christofides è stretta.
\end{theorem}
\begin{proof}
    Si considera un grafo con $n$ nodi disposti in linea.
    I nodi sono collegati in sequenza con archi lunghi 1.
    Poi, si collegano i nodi a distanza 2, partendo dal primo, 
    andando verso destra con archi $1 + \epsilon$, stesso discorso partendo dall'ultimo tornando 
    indietro.

    Si aggiungono poi tutti gli archi per creare una cricca, tali archi hanno lunghezza
    equivalente a quella del cammino minimo che unisce sorgente e destinazione. 

    Il minimum spanning tree pesca solo gli archi lunghi 1.\\
    Il perfect matching utilizza l' arco dal nodo più a sinistra a quello più a destra, 
    lungo $$(1+\epsilon)(\frac{n}{2} -1 ) + 1 = \frac{n}{2} + \epsilon \frac{n}{2} - \epsilon$$

    Il cammino individuato da Christofides è quindi:
    $$(n-1) + \frac{n}{2} + \epsilon \frac{n}{2} - \epsilon = \frac{3}{2}n +\epsilon \frac{n}{2} - (\epsilon +1)$$

    Esiste però la soluzione ottima che consiste nel muoversi da sinistra a destra seguendo gli archi che portano 
    a distanza 2, seguire l'arco dal penultimo all'ultimo nodo lungo 1, e ripetere il procedimento 
    tornando indietro.\\
    Tale approccio ha costo: 
    $$\delta(\Pi^*) = 2((1+\epsilon)(\frac{n}{2}-1) +1) = n + \epsilon n -2\epsilon$$
    Calcoliamo il rapporto di approssimazione:
    $$\frac{\delta(\Pi)}{\delta(\Pi^*)} = \frac{\frac{3}{2}n +\epsilon \frac{n}{2} - (\epsilon +1)}{n + \epsilon n -2\epsilon}$$
    $$\lim_{n \rightarrow \infty, \epsilon \rightarrow 0}\frac{\delta(\Pi)}{\delta(\Pi^*)} = \frac{3}{2}$$
\end{proof}
\begin{theorem}
    Non esiste un $\alpha > 1$ tale che TSP sia $\alpha$-approssiamante in tempo polinomiale.
\end{theorem}
\begin{proof}
    Dato un grafo supponiamo di dover decidere se ammette un ciclo Hamiltoniano (NP completo).

    Sia il grafo $G^\prime = (V, \binom{V}{2})$ costruico come segue:
    \[
        d(x,y) =     
        \begin{cases}
            1\;\mathit{se}\;xy \in E\\ 
            \lceil {\alpha n} \rceil + 1\;\mathit{se}\;xy \notin E        

        \end{cases}
    \]
    
    Applichiamo l'algoritmo ottimo su $G^\prime$, la soluzione equivale a:
    \[
        \delta^*(\Pi) =    
        \begin{cases}
            n\;\mathit{se\;esiste\;un\;ciclo\;hamiltoniano}\\ 
            \lceil \alpha n \rceil+ 1 \;\mathit{altrimenti}
        \end{cases}
    \]

    Usando l'algoritmo $\alpha$-approssimante che si assume esistere, 
    si otterrebbe una soluzione:
    \begin{equation}
        \begin{aligned}
            n \leq \delta(\Pi) \leq \alpha n && \text{con ciclo hamiltoniano}\\
            \lceil\alpha n \rceil+ 1 \leq \delta(\Pi) \leq \alpha\lceil \alpha n \rceil+ 1 && \text{altrimenti}
        \end{aligned}
    \end{equation}
    Significa che si può decidere se il grafo ammette un ciclo hamiltoniano in tempo polinomiale.\\
    Deve perciò valere che:
    $$\alpha n \geq \lceil \alpha n \rceil+ 1$$
    \begin{equation}
        \begin{aligned}
            \alpha n \geq \lceil \alpha n \rceil+ 1\\
            \alpha \geq \alpha + \frac{1}{n}
        \end{aligned}
    \end{equation}
    Non esiste un $\alpha$ maggiore di uno che soddisfa l'equazione.
\end{proof}

\subsection{Problemi PTAS e FPTAS}
I due problemi a seguire ammettono algoritmi che appartengono alla classe PTAS, sono 
quidni approssimabili in maniera arbitraria.

\subsubsection{Minimum partition}
Il problema del minium partition equivale a suddividere un insieme in due sottoinsiemi 
tali che il massimo delle somme degli elementi dei due insiemi è minimizzato.
Si può vedere come \emph{2-load balancing}, visto a sezione \ref{lb}.

Formalmente:\\
\emph{Input}: $w_1, \dots, w_n \in \mathbb{Q}^{>0}$\\
\emph{Output}: $y_1 \subseteq \{1,\dots , n \},  y_2 = \{1,\dots , n \} \setminus y_1$\\
\emph{Funzione obiettivo}: $max(\sum_{i\in y_1}w_i, \sum_{j\in y_2}w_j)$\\
\emph{Tipo}: min

\paragraph{PTAS 2-loadbalance}
L'algoritmo è simile a Sorted Balance, assegna in modo ottimale 
alcuni task.

\begin{algorithm}[H]
    \SetAlgoLined
    \KwIn{$w_1, \dots, w_n \in \mathbb{Q}^{>0}$, $\epsilon > 0$}
    \KwResult{Soluzione per load balance}
    \If{$\epsilon \geq 1$}{$\mathit{Assegno}\;\mathit{tutto}\;\mathit{alla}\;\mathit{prima}\;\mathit{macchina}$}
     $T \gets \mathit{reverse(sorted}(w_1, \dots, w_n))$\\
     $k \gets \lceil \frac{1}{\epsilon} - 1 \rceil$\\
    $\mathit{Assegna\;}\mathit{i\;}\mathit{primi\;}\mathit{k\;}\mathit{task\;}\mathit{in\;}\mathit{modo\;}\mathit{ottimale\;}$\\
    $\mathit{GreedyBalance}(w_{k+1}, \dots, w_n)$\\
     \Return{$\max(W(y_1), W(y_2))$}
     \caption{PTAS 2-loadbalance}
\end{algorithm}

\begin{theorem}
    L'algoritmo proposto fornisce una $(1+\epsilon)$-approssimazone per 
    2-loadbalance.
\end{theorem}
\begin{proof}    
    Per ogni $X \subseteq \{1, \dots, n\}$, $w(X) = \sum_{i\in X}w_i$.

    Sia $L = \frac{w(\{1, \dots, n\})}{2}$ il peso complessivo dei task, diviso 2.

    Vale che: 
    $$w^* \geq L $$
    Se così non fosse, significa che sia la prima che la seconda macchina hanno
    task per un peso inferiore alla metà del peso complessivo delle task.
    Ma ciò è assurdo, visto che la somma dei carichi delle macchine è $2L$.

    Procediamo ora per casi rispetto ad $\epsilon$:

    Caso 1: $\epsilon \geq 1$\\
    In questo caso:
    $$w(y_1) = 2L \leq 2w^* \leq (1+\epsilon)w^*$$
    Ovvero il peso della prima macchina rispetta l'approssimazione di $1+\epsilon$.

    Caso 2: $\epsilon < 1$\\
    Supponiamo che la prima macchina abbia peso maggiore: 
    $$w(y_1) \geq w(y_2)$$
    Sia $h$ l'indice dell'ultimo task assegnato alla prima macchina:
    Caso 2A: $h \leq k$ \\
    Ho trovato la soluzione ottima, visto che il peso della 
    prima macchina è maggiore di quello della seconda.
    
    Caso 2B $h > k$\\
    Sia il peso subito prima dell'ultimo task, vale che: 
    \begin{equation}
        \begin{aligned}
            w(y_1) - w_h &\leq w(y_2)^\prime \leq w(y_2) && \text{peso della macchina 2 in quel momento}\\
            w(y_1) + w(y_1)- w_h &\leq w(y_2) + w(y_1)&& \text{sommiamo il peso della macchina 1}\\
            2w(y_1) - w_h &\leq 2L\\
            w(y_1) &\leq \frac{w_h}{2} + L\\
        \end{aligned}
    \end{equation}
    Inoltre:
    \begin{equation}
        \begin{aligned}
            2L = \sum_{i = 1}^{n} w_i = w_1 + \dots + w_h + \dots + w_n\\
            \geq w_1 + \dots + w_h \geq w_h(k+1)\\
            \implies L \geq \frac{w_h}{2}(k+1)
        \end{aligned}
    \end{equation}
    Calcoliamo ora il rapporto di approssimazione:
    \begin{equation}
        \begin{aligned}
            \frac{w(y_1)}{w^*} &\leq \frac{w(y_1)}{L}&& \text{per la proprietà vista all'inizio}\\
            &\leq \frac{\frac{w_h}{2} + L}{L} = 1 + \frac{w_h}{2L}&& \text{per la seconda proprietà}\\
            &\leq 1 + \frac{w_h}{2\frac{w_h}{2}(k+1)} = 1 + \frac{1}{k+1}&& \text{per la terza proprietà}\\
            &= \frac{1}{\lceil \frac{1}{\epsilon} -1 \rceil +1} \leq 
            \frac{1}{\frac{1}{\epsilon} -1 +1} = 1+\epsilon&& \text{per la definizione di k}
        \end{aligned}
    \end{equation}
\end{proof}
\begin{theorem}
    L'algoritmo impiega un tempo pari a 
    $$O(2^{\frac{1}{\epsilon}} + n\log n)$$
    Ovvero è polinoimiale in $n$, ma esponenziale in $\epsilon$
\end{theorem}
\begin{corollary}
    Minimum Partition $\in$ PTAS
\end{corollary}
\subsubsection{Knapsack FPTAS}
Il problmea dello zaino consiste nel selezionare un insieme 
di elementi che massimizzino il valore, il cui ingombro non superi
la capacità dello zaino.

Formalmente:\\
\emph{Input}: $(w_1, v_1), \dots, (w_n, v_n)\;w_i, v_i, W \in \mathbb{N}$ dove $W$ è la capacità dello zaino\\
\emph{Output}: $X \subseteq \{1, \dots, n\}, \sum_{i \in X} w_i \leq W$\\
\emph{Funzione obiettivo}: $\sum_{i \in X} v_i$\\
\emph{Tipo}: max

La versione di decisione del problema è NP completa.

\paragraph{Knapsack in programmazione dinamica}
È possibile risolvere il problema dello zaino in programmazione dinamica, 
si puo considerare la seguenze equazione di ricorrenza.

\[
    DP[i,w] = 
    \begin{cases}
        0\; \mathit{se}\; i = 0\;\mathit{or}\;w = 0\\
        DP[i-1,w]\;\mathit{se}\;w_i > w\\
        \max(DP[i-1,w], DP[i-1,w-w_i] + v_i)\;\mathit{altrimenti}

    \end{cases}\]

L'algoritmo di facile implementazione segue l'equazione appena introdotta, 
riempendo una matrice iterativamente o ricorsivamente.
\begin{remark}
    Iterativamente si possono mantenere solo le ultime due righe, 
    lo spazio si abbatte.
\end{remark}

\begin{theorem}
    Knapsack in programmazione dinamica ha complessità in tempo e spazio (a meno di ottimizzazioni) equivalente a 
    $O(nW)$, dove $n$ è il numero di oggetti e $W$ la capacità dello zaino.
\end{theorem}
\begin{proof}
    Per mantenere il parametro $W$ in memoria sono necessari $O(\log W)$ bit, quindi
    $W = 2 ^{|W|}$. 

    L'algoritmo si dice essere pseudopolinomiale.
\end{proof}

\paragraph{Knapsack programmazione dinamica rivisitato}
Questa volta cambia il contenuto della matrice.
In particolare, in $DP[i][v]$ è contenuto il minimo peso, che 
consente di portare a casa $\geq v$ usando i primi i oggetti.

I casi base sono:
\begin{enumerate}
    \item $v = 0$, in quel caso il valore della cella è nullo
    \item $i = 0$, in quel caso si considera $+\infty$
\end{enumerate}

L'equazione di ricorrenza diventa: 
\[
    DP[i,v] = 
    \begin{cases}
        0\; \mathit{se}\; v = 0\\
        +\infty\; \mathit{se}\; i = 0\\
        \min(DP[i-1,v], DP[i-1,v-v_i] + w_i)\;\mathit{altrimenti}
    \end{cases}\]

\begin{theorem}
    Il problema riformulato ha complessità in tempo e spazio (a meno di ottimizzazioni) equivalente a 
    $O(n\sum_i v_i) = O(n^2v_{max})$. L'algoritmo rimane pseudopolinomiale, per il secondo termine.
\end{theorem}

\begin{remark}
    La matrice si può restringere in larghezza, arrotondando i valori, in modo da introdurre approssimazione
    nel valore della soluzione ma non nella validità di essa.
\end{remark}

\paragraph{Knapsack rescaling}
Preso un input per Knapsack, si considera il valore:
$$\theta = \frac{\epsilon v_max}{2n}, 0 < \epsilon \leq 1$$
L'input del problema originale diventa:
$$\bar{\Pi} = ((w_i, \bar{v_i}), W), \bar{v_i} = \lceil\frac{v_i}{\theta}\rceil\theta$$
Si considera poi una terza versione:
$$\hat{\Pi} = ((w_i, \hat{v_i}), W), \hat{v_i} = \lceil\frac{v_i}{\theta}\rceil$$

La formulazione $\bar{\Pi}$ è utile nello studio del problema, quella versione non 
si può risolvere in programmazione dinamica, visto che i valori non sono interi.

\begin{remark}
    Le soluzioni ammissibili dei problemi sono equivalenti.
\end{remark}

\begin{lemma}
    Le soluzioni ottime $\hat{S^*}, \bar{S^*}$ di $\hat{\Pi^*}, \bar{\Pi^*}$ sono equivalenti, 
    sono semplicemente scalati per il parametro $\theta$.
\end{lemma}

\begin{lemma}
    \label{knl2}
    Sia S una soluione ammissibile di $\Pi$, allora:
    $$\sum_{i \in S} v_i \leq (1 + \epsilon) \sum_{i \in \hat{S^*}} v_i$$
\end{lemma}
\begin{proof}
    \begin{equation}
        \begin{aligned}
            \sum_{i \in S} v_i &\leq \sum_{i \in S} \bar{v_i} && \text{Per l'operazione di ceil}\\
            &\leq \sum_{i \in \bar{S^*}} \bar{v_i} && \text{S ammissibile, S barra ottima}\\
            &= \sum_{i \in \hat{S^*}} \bar{v_i} && \text{per la prima osservazione}\\
            &\leq \sum_{i \in \hat{S^*}} (v_i + \theta) = \sum_{i \in \hat{S^*}} v_i + |\hat{S^*}|\theta\\
            &\leq \sum_{i \in \hat{S^*}} v_i + n \theta = \sum_{i \in \hat{S^*}} v_i + n\frac{\epsilon v_{max}}{2n}\\
            &= \sum_{i \in \hat{S^*}} v_i + \frac{\epsilon v_{max}}{2} 
        \end{aligned}
    \end{equation}
    Siccome questo vale per ogni soluzione ammissibile, è vera anche per $S = \{i_{max}\}$, ovvero l'indice
    che dà valore massimo. Continuando con l'equazione precedente:
    \begin{equation}
        \begin{aligned}
            v_{max} \leq \sum_{i \in \hat{S^*}} v_i + \frac{\epsilon v_{max}}{2} \leq \sum_{i \in \hat{S^*}} v_i + \frac{ v_{max}}{2} &&\text{epsilon minore uguale a 1}\\
            \sum_{i \in \hat{S^*}} v_i \geq \frac{\epsilon v_{max}}{2}
        \end{aligned}
    \end{equation}

    Applicando ora l'equazione appena ottenuta alla prima equazione della dimostrazione:
    \begin{equation}
        \begin{aligned}
            \sum_{i \in S} v_i \leq \sum_{i \in \hat{S^*}} v_i + \frac{\epsilon v_{max}}{2}\\
            \sum_{i \in S} v_i \leq \sum_{i \in \hat{S^*}} v_i + \epsilon \sum_{i \in \hat{S^*}} v_i = (\sum_{i \in \hat{S^*}} v_i)(1+\epsilon)
        \end{aligned}
    \end{equation}
\end{proof}
\begin{remark}
    Vale che:
    $$\hat{v_{max}} = \lceil\frac{v_{max}}{\frac{\epsilon v_{max}}{2n}}\rceil = \lceil\frac{2n}{\epsilon}\rceil
    \leq \frac{2n}{\epsilon} + 1$$
\end{remark}
\begin{corollary}
    L'algoritmo in programmazione dinamica rivisitato, applicato a $\hat{\Pi}$, fornisce una
    $(1=\epsilon)$-approssimazione di Knapsack in tempo e spazio $O(\frac{n^3}{\epsilon})$.
\end{corollary}
\begin{proof}
    L'algoritmo individua $\hat{S^*}$, vale che:
    \begin{equation}
        \begin{aligned}
            (1-\epsilon)\sum_{i\in \hat{S^*}}v_i \geq \sum_{i \in S} v_i \geq v^* &&\text{Per il lemma \ref{knl2}}\\
            1 \leq \frac{v^*}{\sum_{i\in \hat{S^*}}v_i} \leq (1+\epsilon)
        \end{aligned}
    \end{equation}
    L'algoritmo ha complessità 
    $$O(n^2 \hat{v_{max}}) = O(n^2(\frac{2n}{\epsilon} + 1)) = O(\frac{n^3}{\epsilon})$$
\end{proof}
\begin{remark}
    Al variare di $\epsilon$ si aumenta o abbassa il tempo necessario al completamento dell'algoritmo, 
    in particolare.
\end{remark}
\begin{corollary}
    Knapsack risolto in questo modo appartiene alla classe FPTAS.
\end{corollary}