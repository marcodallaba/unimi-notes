\section{Modelli di comunicazione}
Nei sistemi distribuiti i nodi comunicano tra loro tramite qualche
mezzo, esiste una grande eterogeneità riguardo al tipo dei canali utilizzati, 
ma essi vengono utilizzati per scambiare messaggi.

\paragraph{Middleware}
Considerano la coda iso osi, si possono fondere sessione e presentazione 
in un unico livello middleware. In questo livello possono essere collocati i 
servizi di comunicazione di alto livello trattati a seguire.

Considerando ad esempio un client che effettua una richiesta ad un 
server, un middleware potrebbe gestire il messaggio mandato 
dal client e ad esempio fornire feedback sullo stato della richiesta, 
ad esempio la ricezione del server, il processing etc.
\subsection{Comunicazione persistente e transiente}
Un esempio intiuitivo di comunicazione persistente consiste nello scambio 
di lettere da post office a post office, i messaggi sono 
memorizzati nel transito, mentre un sistema transiente non 
prevede la memorizzazione dei messaggi in transito, un esempio è una chiamata 
telefonica.

Esistono molte varianti di queste comunicazioni, legate a quanto 
un client aspetta, a quando il server riceve il messaggio etc.

\paragraph{Modello persistente asincrono}
Il processo A manda un messaggio al processo B, il quale non è in esecuzione.
Ad un certo punto B parte e riceve il messaggio. Si prevede una memorizzazione 
intermedia. A non aspetta nessun riscontro da B, si parla quindi di comunicazione 
asincrona.

\paragraph{Modello persistente sincrono}
In questo caso il processo A manda una richiesta a B, aspettandosi però 
un riscontro.

\paragraph{Modello transiente asincrono}
In questo caso il messaggio inviato da A a B è ricevuto dal secondo 
solo se in esecuzione.

\paragraph{Modello transiente sincrono}
In questo caso il processo B è in esecuzione ma sta eseguendo altro.
Quello che succede è che B riceve il messaggio, manda un ACK e finsice di 
processare quello che sta facendo. Poi inizia con la richiesta di A.

\paragraph{Delivery-based transiente sincrono}
In questo caso A si ferma fino a quando B non inizia ad elaborare.

\paragraph{Response-based transiente sincrono}
In questo caso A si ferma fino a quando B non finisce di elaborare.

\subsection{Message-oriented communication}

\paragraph{Berkeley sockets}
Tool di comunicazione introdotto con Berkeley Unix. Si possono 
considerare come comunicazione transiente.\\
Fanno uso di diverse primitive di sistema per creare una connessione
tra due nodi:
\begin{itemize}
    \item \emph{Socket}: crea un endpoint. Facendo riferimento 
    a Unix, si può pensare a una socket come 
    un descrittore, simile ad esempio ai descrittori per standard input,
    output ed error. Alla creazione di una nuova socket si restituisce un descrittore
    sul quale scrivere e leggere
    \item \emph{Bind}: si collega un indirizzo locale, ovvero una porta, a una socket
    \item \emph{Listen}: segnala la disponibilità a ricevere messaggi
    \item \emph{Accept}: si blocca il caller fino a quando non arriva una connessione
    \item \emph{Connect}: tentativo di stabilire una connessione, c'è bisogno di ip 
    e porta
    \item \emph{Send}: invio di dati sulla socket
    \item \emph{Receive}: ricezione di dati sulla socket
    \item \emph{Close}: rilascio delle risorse
\end{itemize}

\paragraph{Queuing systems}
L'idea di base di questo sistema è avere delle code di messaggi, gestite 
da nodi intermedi detti router. 
Si garantisce la persistenza, ovvero, un nodo manda ad una coda, i router 
indirizzano il messaggio al destinatario e lo inseriscono nella loro coda.
Tra le primitive del sistema si potrebbero trovare:
\begin{itemize}
    \item \emph{Put}: append del messaggio a una coda
    \item \emph{Get}: prelevo il messaggio da una coda, primitiva bloccante
    \item \emph{Poll}: controlla se una coda ha messaggi e rimuove il primo, non bloccante
    \item \emph{Notify}: notifica quando un messaggio è inserito in una certa coda
\end{itemize}
In un sistema di questo tipo, il ruolo dei broker è quello di garantire la persistenza, 
ma anche quello di effettuare eventuali conversioni. Il secondo scenario accade quando 
due nodi che vogliono comunicare hanno convenzioni diverse. 

Il sistema a code è preferibile quando :
\begin{itemize}
    \item si vuole comunicazione ascincrona e persistente 
    \item quando si può aumentare la scalabilità ammettendo ritardi di gestione delle richieste, 
    \item quando i producer sono più veloci dei consumer
    \item quando si sta implementando un pattern pub-sub
\end{itemize}

\paragraph{Remote proceduce call}
L'idea è quella di nascondere la messaggistica necessaria ad effettuare una 
chiamata ad una procedura remota.

Non è possibile inviare al processo remoto il semplice indirizzo di memoria dei 
dati sul client, bensì serve mandare al processo remoto una funzione con parametri.
La logica quindi si riassume in questi punti:
\begin{enumerate}
    \item Arrivo ad un'istruzione da eseguire in remoto
    \item Serializzazione dei parametri e invio nella rete del messaggio
    \item Ricezione del server, deserializzazione ed esecuzione della funzione
    \item Invio del risultato al client
\end{enumerate}
L'implementazione di questo scambio di messaggi è offerta da uno standard, 
non bisogna "inventarsi" nulla al contrario della comunicazione con socket.
Il passaggio da dati a messaggio e viceversa è chiamato:
\begin{itemize}
    \item \emph{Marshalling}: formattazione dei parametri in messaggio, prevede
    eventuale serializzazione, bisogna anche porre attenzione all'architettura
    di partenza e arrivo, potrebbero differire in termini di codifica
    \item \emph{Unmarshalling}: procedure inversa di marshalling
\end{itemize}
Il binding tra client e server è trasparente, alcuni sistemi offrono la possiblità 
di scoprire dinamicamente quale server offre una determinata funzione.

