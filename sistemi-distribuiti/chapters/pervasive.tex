\section{Sistemi pervasivi}

\subsection{Sensori}

\paragraph{Transducer}
Apparati che trasformano una forma di energia in un'altra.

\paragraph{Funzionamento sensori}
Nella pratica si utilizza un Transducer per catturare
un fenomeno fisico, trasformandolo in un segnale elettronico.
Uno dei parametri di tale osservazione è la frequenza con 
cui si osserva il fenomeno fisico.

\paragraph{Tipi di sensori}
Si possono disinguere tra i sensori fisici tre tipologie:
\begin{itemize}
    \item \emph{Movimento}: misurano forse di accellerazione 
    e rotazione sui tre assi;
    \item \emph{Ambientali}: misurano parametri ambientali, 
    quali la temperatura, pressione, illuminazione e umidità;
    \item \emph{Posizione}: misurano la posizione fisica del 
    dispositivo.
\end{itemize}

\paragraph{Componenti si un sensore}
Tra i componenti di un sensore si individuano sicuramente:
\begin{itemize}
    \item \emph{Sensing subsystem}: parte che cattura i fenomeni
    fisici;
    \item \emph{Processing subsystem}: processing dei segnali individuati;
    \item \emph{Wireless communication}: comunicazione con altri apparati;
    \item \emph{Power source}: fonte di energia, batteria, pannello solare, etc.
\end{itemize}

Esistono poi parti opzionali, che potrebbero essere uno \emph{storage per i dati} o 
una \emph{wired interface}.

\paragraph{Sensori virtuali}
Un esempio di un sensore virtuale sono le Google Places API, 
uno use-case potrebbe essere individuare la posizione 
tramite un sensore classico, per poi utilizzare 
un servizio, come ad esempio quella API, per capire cosa c'è 
nelle vicinanze oppure che tempo fa. 

Si può pensare a quei servizi come una sorta di sensori virtuali.

\paragraph{Attuatori}
Sono particolari tipi di transducer, sono alimentati e convertono 
energia in azione, effettuano quindi movimenti o operazioni 
di switch.

Un esempio sono dispositivi che aprono porte, aprono valvole, 
etc.

\paragraph{Sensor Networking}
Le reti che si formano tra i sensori sono di tipo mesh. 
L'idea è quella di avere tutti i dispositivi collegati tra loro 
in modo da evitare problemi di disconnessione da un nodo centrale, 
ma anche per avere più affidabilità.

Per collegare i dispositivi si utilizza la tecnologia \emph{Z-Wave}
che ha un range di 10-30 metri, e consuma molto poco. Essa 
crea la rete mesh tra i sensori.

\paragraph{Discovery e pairing}
Gli smart devices possono apparire e scomparire frequentemente in rete. 
Nella pratica serve un modo di registrare un dispositivo, assegnandogli un 
indirizzo, simile a DHCP, per poi associare il device.

\subsection{Acquisizione di dati}

\paragraph{Desiderata}
I problemi di interrogazione dati in un sistema pervasivo differisce 
dai classici problemi legati ai database.

Esiste la necessità di interrogare sensori in modo continuo, l'opposto 
delle query in database, che sono solitamente one-time, 
i dati hanno inoltre una forte caratterizzazione spazio-temporale, ad esempio, 
tenendo sotto controllo una temperatura, mi basterebbe capire quando 
questa cambia, senza continuare ad interrogare i sensori.

Si fa notare che non è ragionevole ripetere una query one-time ogni secondo 
per simulare un flusso continuo di dati.

\paragraph{Metodi di acquisizione}
Esistono alcuni approcci semplici per il processing dei dati dai sensori:
\begin{itemize}
    \item \emph{Batch processing}: i sensori possono fare buffering dei dati, 
    successivamente questi 
    vengono processati offline dalla base station
    o su un server.
    Non c'è processing sul sensore, ma la comunicazione 
    costa molto, la base station riceve i dati ma con un delay;
    \item \emph{Sampling}: si campionano le misurazioni, anche se non 
    è una misurazione completa come nel primo approccio, esistono delle garanzie 
    di qualità;
    \item \emph{Sliding window}: si fornisce una risposta approssimata 
    basata su un gruppo di letture consecutive, la base station 
    riceve questa misurazione aggregata approssimata. Questa tecnica potrebbe utilizzare 
    window sovrapposte, in modo da non perdere cambi repentini delle misurazioni, in tal 
    caso si parla di \emph{overlapping sliding window}.
\end{itemize}

\paragraph{In-network query processing}
L'idea di base di questo approccio è quello 
di creare una overlay network, ad esempio uno spanning tree, 
per aggregare i dati in modo gerarchico.

Un altro modo è il \emph{Duty cycling} che permette 
di mettere a dormire e svegliare dispositivi quando serve 
o meno comunicazione, in modo da risparmiare energia.

Un'altro caso è quello in cui la base station si muove, 
in quel caso si possono sfruttare gli spostamenti per 
trasferire i dati \emph{quando conviene}.

\subsubsection{Pulizia e compressione dei dati}

L'idea di base della data compression è quella 
comprimere i dati in qualche modo in modo da 
ridurre la comunicazione necessaria. Per la pulizia 
di dati invece si punta a eliminare misurazioni 
estreme, che si verificano nel caso in cui i 
sensori non siano troppo precisi.

\paragraph{Approcci model-based}
I dati raccolti possono avere forte correlazione spazio temporale, quindi
si potrebbe sfruttare tale correlazione per ridurre il sampling necessario.

Tale correlazione è utile sia per ridurre il numero di campionamenti 
necessari, sia per pulire i dati provenienti dai sensori, tipicamente 
poco precisi, sfruttando tecniche statistiche, ovvero, sfrutto un modello allenato per capire 
se una misurazione è un outlier.

Un'altra applicazione degli approcci model based è la compressione dei dati 
provenienti dai sensori.
Queste tecniche sono solitamente leggere, di possibile utilizzo direttamente 
sul dispositivo stesso.

\paragraph{Poor Man's Compression MidRange}
È un piece-wise constant approximation method.
In questa tecnica di compressione si approssima un segmento, 
ovvero un certo numero di misurazioni nel tempo, consecutive, 
con una loro funzione aggregata, ad esempio il minimo, massimo, media etc.

\paragraph{Model-based compression}
In questo caso si applica un modello alla task di compressione, il risultato è 
una compressione potenzialmente più sofisticata, 
si può pensare ad una regressione lineare o polinomiale dei dati di un 
segmento.

\subsection{Context awareness}

\paragraph{Contesto}
Le applicazioni classiche non sono in grado capire il contesto delle richieste.
Un esempio di interrogazione ad un sistema classico 
potrebbe essere, \emph{qual è la strada più veloce da X a Y considerando 
che è l'ora di punta e non voglio prendere autostrade?}

Mi sto rivolgendo quindi in linguaggio naturale ad un servizio, specificando però 
qualsiasi cosa.

Un sistema intelligente potrebbe capire da dove sto partendo con la posizione, 
dove sto andando controllando il mio calendario, capire che è l'ora di punta controllando 
l'orario e il traffico medio, e infine sapere che non mi piacciono le autostrade 
ripescando i miei tragitti passati.

L'idea è quella di chiedere il meno possibile al sistema, affidandosi alle 
capacità di quest'ultimo per capire il contesto attuale, dove per contesto 
si intendono quelle informazioni che possono essere utilizzate per caratterizzare 
la situazione di un entità, che sia essa una persona, un luogo o un oggetto, rilevante 
nell'interazione tra l'utente e l'applicazione.

\paragraph{Adattività}
Con questo principio si intende il cambiare alcune caratteristiche del sistema 
a seconda dell'ambiente circostante. Un esempio potrebbe essere quello 
di adattare la luminosità e i colori di uno schermo, 
cambiare un'interfaccia a seconda del dispositivo, etc.

In contesti meno user-related si può pensare al cambiamento delle politiche di caching, 
computazione server-side, modificare il sampling rate, etc.

\paragraph{Ottenere contesto}
I dati \emph{low-level context} si ottengono direttamente dai sensori o da 
altre fonti, un esempio possono essere dati grezzi dai sensori, preferenze dell'utente, etc.
Questi sono dati sostanzialmente di basso livello.

I dati invece di \emph{high-level context} possono essere derivati da una combinazione di 
sensori e metodi di inferenza, possibilmente molto complessi.

\paragraph{Inferenza}
Per inferire il contesto dai dati si può applicare per esempio la logica. Questi 
sono i cosiddetti approcci simbolici e si basano sulla creazione di regole 
e statement logici.

Tra gli approcci statistici invece si individuano classificazione e clustering, 
solitamente in modalità supervised learning.

Esiste poi ovviamente la possibilità di avere approcci ibridi.

\paragraph{Activity recognition}
L'idea di base è quella di utilizzare i dati provenienti
dai sensori, che potrebbero essere stati aggregati 
da una compressione, pèer allenare un modello statistico 
e capire l'attività svolta dall'utente.

Si verifica empiricamente che un approccio statistico 
ha risultati migliori di un modello logico basato su regole.

\subsubsection{Rappresentazione del contesto}

Una Rappresentazione formale dei dati di contesto 
è necessaria per poterli processare in modo automatico. 
Inoltre, tali dati potrebbero essere condivisi tra applicazioni. 

\paragraph{Requisiti}
Un modello di contesto dovrebbe supportare dati eterogenei, 
modellare relazioni, storia del contesto, imprecisione, 
ragionamento sui dati, efficienza ed utilizzo di essi.

\paragraph{Flat models}
Un sistema molto semplice consiste in una lista di chiavi valore, 
magari arricchito con XML, supportando gerarchia di valori.

Solitamente sono adattati alla singola applicazione.

\paragraph{Modelli basati su database}
In questo approccio si sta semplicemente utilizzando un database
ER per rappresentare i dati di contesto. Il modello è stato 
adattato a supportare alcune caratteristiche come l'incertezza.

\paragraph{Modelli ontologici}
Si sta cercando di assegnare una definizione formale ai dati di contesto.
Tale formalizzazione viene fatta tramite description logic, 
che permette di definire insiemi molto complessi, definire semantiche 
e consistency checking e reasoning.

Tale modello permette ad esempio di riconoscere in modo automatico 
subsumption, ovvero concetti che sono specializzazioni di altri, 
e realization, ovvero quali sono i concetti che catturano 
un set di osservazioni.
Esistono strumenti automatizzati per generare ontologie.

% \subsection{Data privacy}

% \paragraph{Dati personali}
% Sono sostanzialmente informazioni legate ad un singolo individuo, 
% identificato o identificabile.

% \paragraph{Privacy dei dati}
% Capacità di un individuo di controllare il rilascio dei propri dati.

% \paragraph{Quasi identifier}
% Dati che non identificano da sè un individuo, ma che combinati possono 
% diventare identifier. Un esempio è l'unione dell'età e dell'indirizzo, 
% i due di per sè non identificano nessuno, ma mettendoli insieme 
% potrebbero restringere il campo di molto.

