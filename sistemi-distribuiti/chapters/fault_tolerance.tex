\section{Fault tolerance}
Uno degli obiettivi di un sistema distribuito è
la trasparenza ai guasti. I requisiti da fornire sono 
disponibilità, affidabilità, sicurezza e manutenibilità.

\paragraph{Cause di fallimento}
Un sistema potrebbe fallire per molti motivi, ad esempio
potrebbe arrestarsi, fallire sia nella ricezione che invio 
di messaggi, fallire nel tempo di risposta, rispondere 
con messaggi sbagliati.

\subsection{Ridondanza}
I fallimenti possono essere mascherati dalla ridondanza.
Si può pensare di \emph{ridondare le informazioni}, 
tramite bit extra. Esiste anche la \emph{ridondanza in termini 
temporali}, se una transazione è abortita essa può essere 
ripetuta. Infine si parla di \emph{ridondanza fisica} quando 
esistono hardware o processi addizionali usati per sistemare
guasti.

\paragraph{Resilienza dei processi}
Per aumentare la fault tolerance dei processi 
si utilizzano gruppi di processi 
equivalenti dove ognuno riceve i messaggi indirizzati al gruppo.

Servono metodi di accesso ai gruppi e la domanda che 
sorge spontanea è, quanta ridondanza serve?

\paragraph{Quanta ridondanza}
Se un processo smette di funzionare $k + 1$ processi 
offrono $k$-fault tolerance.

Se invece il processo che fallisce risponde con valori 
errati, sono necessari almeno $2k +1$ processi, il client può 
capire la risposta contando la maggioranza.

Nel caso in cui serva un consenso, ad esempio in un algoritmo 
di elezione, il problema diventa molto difficile e a volte 
impossibile.

\subsection{Problemi di accordo}
I problemi che si tentanto di risolvere sono:
\begin{itemize}
    \item \emph{Problemi di consenso}: in questo caso ogni processo 
    propone un singolo valore e i processi devono mettersi d'accordo 
    sullo stesso valore;
    \item \emph{Generali Bizantini}: un processo propone un valore $v$, 
    tutti i processi corretti devono mettersi d'accordo su $v$, se 
    il processo iniziale non è faulty;
    \item \emph{Interactive consistency}: ogni processo propone un valore, 
    e tutti devono mettersi d'accordo su un vettore di valori, 
    ovvero su tutti i valori proposti dai singoli processi.
\end{itemize}

Tutti i problemi sono complicati e un algoritmo per uno di essi 
porta ad una soluzione per gli altri in maniera facile.

\paragraph{Soluzione generali Bizantini sistema sincrono}
Nel problema dei generali bizantini sono necessari almeno 
$N = 3k + 1$ processi per compensare $k$ processi malfunzionanti.
L'esecuzione con $k = 1, N = 4$ procede come segue: 
\begin{itemize}
    \item Il comandante manda il valore ai messaggeri;
    \item Essi mandano in broadcast agli altri;
    \item Ora si distinguono alcuni casi:
    \begin{itemize}
        \item nel caso in cui uno dei messaggeri fosse malfunzionante, 
        gil altri possono applicare una funzione di maggioranza 
        per capire quale valore ha mandato il comandante;
        \item se il comandante è malfunzionante, gli altri 3 processi 
        riconoscono che il valore del comandante non è affidabile, 
        procedono come devono.
    \end{itemize}
\end{itemize}

\subsection{Sistemi sincroni e asincroni}

\paragraph{Sistemi sincroni}
In questo caso l'esecuzione di ogni nodo è limitata in velocità 
e tempo, inoltre i link di comunicazione hanno delay limitato, 
ed è limitato anche il drift del clock di ogni nodo.

\paragraph{Sistemi asincroni}
Si perdono qui le assunzioni precedenti, 
la velocità di ogni nodo è arbitraria, come lo sono 
i delay sui link di comunicazione e il drift del clock di ogni nodo.

\paragraph{Considerazioni}
Nei sistemi asincroni è tutto più difficile, solitamente si 
effettuano assunzioni sulla sincronia, si devono accettare 
il fallimento di alcune soluzioni ed eventuali deadlock.

\paragraph{Teorema FLP}
Quando il delay sui link di comunicazione è arbitrario 
non esiste una soluzione ai generali Bizantini e a totally-ordered
multicast, anche se solo un nodo è malfunzionante.
Esistono però soluzioni per sistemi parzialmente sincroni.

\paragraph{Teorema CAP}
In un sistema ideale si vogliono \emph{consistenza}, ovvero leggo sempre 
lo stesso valore da nodi differenti, \emph{disponibilità}, cioè ogni richiesta riceve
una risposta 
e \emph{tolleranza} ad un numero arbitrario di \emph{messaggi persi} o 
arrivati tardi.

Il teorema dice che in un sistema asincrono si possono ottenere solo due 
di questi requisiti in storage read e write.

\paragraph{Paxos e Raft}
Esistono due algoritmi che offrono $k$ fault tolerance su sistemi non sincroni, 
risolvo il problema di consenso sotto parziale assunzione di sincronicità.