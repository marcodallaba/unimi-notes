\section{Introduzione ai sistemi distribuiti}

\begin{quote}
    \emph{You know you have one when the crash of a computer you've
    never heard of stops you from getting any work done - Leslie Lamport} 
\end{quote}

\subsection{Definizioni}

\paragraph{Definizione classica}
Un sistema distribuito è una collezione di computer indipendenti
che appaiono all'utente finale come un singolo sistema.
Le macchine non hanno memoria condivisa e possono in qualche modo 
comunicare tra loro.


\paragraph{Interpretazione come middleware}
Da un punto di vista architetturale un sistema distribuito può essere
 visto come un middleware che si estende tra molte macchine, offrendo 
ad ognuna una qualche interfaccia.

\paragraph{Esempi}
Alcuni esempi di sistemi distribuiti possono essere:
\begin{itemize}
    \item Un set di PC che comunicano tra loro con un 
    processo che è assegnato dinamicamente
    \item Online multiplayer
    \item World Wide Web 
\end{itemize}

\subsection{Obiettivi}
Gli obiettivi di un sistema distruibuito si possono ricondurre a garantire
accesso delle risorse, trasparenza, apertura e scalabilità.

\paragraph{Trasparenza}
Esistono vari tipi di trasparenza da garantire in un contesto distribuito:
\begin{itemize}
    \item \emph{Accesso}: nasconde la rappresentazione dei dati 
    e come gli utenti vi accedono
    \item \emph{Locazione}: si nasconde dove una risorse è mantenuta
    \item \emph{Migrazione}: non si mostra il fatto che una risorsa ha cambiato 
    locazione
    \item \emph{Rilocazione}: come il precedente ma quando la risorsa è in uso
    \item \emph{Replicazione}: nasconde la replicazione delle risorse
    \item \emph{Concorrenza}: nasconde il fatto che una risorsa è condivisa
    da più utenti
    \item \emph{Fallimento}: non mostra fallimenti e relativi recovery del sistema
\end{itemize}
Solitamente non è possibile garantire tutti questi punti.

\paragraph{Apertura}
Un sistema distribuito aperto dovrebbe riuscire a fornire
\begin{itemize}
    \item Interoperabilità
    \item Portabilità
    \item Possibilità di estensione
\end{itemize}
Può essere ottenuta tramite l'uso di protocolli standard, API, \dots

\paragraph{Scalabilità}
La scalabilità è fondamentale in un constesto distribuito, 
la desiderata è che le prestazioni non peggiorino alll'estensione
del sistema, bisogna perciò evitare la centralizzazione, 
in particolare di servizi, dati e algoritmi.\\
In algoritmo decentralizzato ad esempio:
\begin{itemize}
    \item Nessuna macchina ha complete informazioni sul sistema
    \item Le decisioni sono prese guardando conoscenze locali
    \item Il fallimento di una singola macchina non pregiudica lo stato
    del sistema
    \item Non ci sono implicazioni implicite sull'esistenza di un clock globale
\end{itemize}
L'opposto di un sistema centralizzato.\\
Un esempio di scalabilità si trova nella divisione del DNS in zone, 
evitando quindi di avere un singolo name server centrale.

\subsection{Tipologie}
Esistono varie tipologie di sistemi distribuiti che variano a seconda degli 
scopi. SI definiscono i computing systems, information systems e pervasive computing.
\subsubsection{Computing systems}
\paragraph{Clusters} collezione di workstations uguali o molto simili, 
collegate tra solo da una rete locale ad alta velocità, fanno girare lo 
stesso sistema operativo.
Esistono due tipologie:
\begin{enumerate}
    \item \emph{Asimmetrico}: esiste un nodo master che coordina 
    e controlla gli altri nodi, un esempio è Google Borg
    \item \emph{Simmetrico}: tutti i nodi hanno lo stesso software
\end{enumerate}

\paragraph{Cloud computing}
Il cloud computing è un modello per fornire accesso a una rete di 
computer in modo conveniente e on demand, caratterizzato dalla 
facilità di riconfigurazione delle macchine.
Alcune caratteristiche sono che:
\begin{itemize}
    \item I nodi sono eterogenei
    \item Le connessioni sono eterogenee, in termini di capacità e di affidabilità
    \item On-demand self-service, ovvero la facile configurazione che si è discussa sopra
    \item La capacità di calcolo è accessibile tramite la rete, tramite meccanismi 
    standard
    \item Le risorse sono condivise tra molti utenti, in modo da ottimizzare il loro utilizzo
    \item Elasticità rapida, ovvero le risorse sono facilmente assegnabili
    \item Esistono sistemi di misura, ovvero metriche che indicano l'utilizzo 
    del sistema cloud
\end{itemize}
Esistono poi alcune categorie: 
\begin{itemize}
    \item Software as a Service: utilizzo di un software che gira in cloud
    \item Platform as a Service: deploy che utilizza tool del cloud provider
    \item Infrastructure as a Service: software che gira sfruttando l'infrastruttura cloud
\end{itemize}

\paragraph{Edge computing}
Nel corso degli anni alcuni nodi dei sistemi distruibuiti non sono più 
semplici computer ma possono essere sensori dispositivi che 
producono molti dati.

C'è bisogno di connetterli in tempo reale, ad esempio tramite una rete 5G, ed effettuare 
preprocessing prima di comunicare con il cloud.
Si può pensare ad un solo strato intermedio tra sensori e cloud.

\paragraph{Fog computing}
Simile all'edge computing, ma potrebbero essre presenti più livelli tra i sensori 
e il cloud vero e proprio.

\subsubsection{Information systems}
Rappresentano un'altra tipologia di sistemi distruibuiti.
\paragraph{Database distribuiti}
Database distribuiti in cloud, oppure blockchain.

\paragraph{Transaction porocessing systems}
Lo scopo è quello di garantire le proprietà ACID in un contesto distribuito.

\subsubsection{Pervasive computing}

Un sistema distribuito pervasivo ha alcune caratteristiche che 
lo discostano da un sistema classico, in particolare si possono trovare 
nodi inconvenzionali, ad esempio smart objects vari e il principio 
di adattività, ovvero la capacità di adattarte il comportamento del sistema
in base all'obiettivo del sistema.

Alcuni esempi possono essere l'uso di dispositivi smart in casa, oppure 
la guida autonoma.
