\section{Distributed Ledger Technologies e Blockchain}

\paragraph{Blockchain}
L'idea di base è quella di avere una rete nella quale 
si possono solo aggiungere informazioni e non toglierle. 
Le applicazioni sono svariate.

\paragraph{DLT}
È un sistema distribuito con controllo decentralizzato, in cui i 
nodi sono eseguiti da entità che non sono trusted tra loro 
e mantengono una copia delle transazioni.

Bisogna accordarsi sulla storia dei dati.

\paragraph{Transazione blockchain}
Una transazione è un record dati, nel mondo finanziario
equivale ad un invio di una somma.

Ogni transazione viene firmata con la chiave privata del mittente e propagata 
agli altri. 
Quando un nodo riceve la transazione la valida. Una transazione validata non 
è considerata ancora parte della blockchain.

Le transazioni vengono raggruppate in blocchi per timestamp e mantenute 
da tutti i nodi. Esistono poi collegamenti tra loro.

\paragraph{Problematiche}
Esistono alcune problematiche legate al mondo blockchain, che sostanzialmente 
sono quelle dei sistemi asincroni:
\begin{itemize}
    \item l'arrivo di una transazione potrebbe differire tra nodi;
    \item alcune transazioni potrebbero contraddirsi;
    \item nodi differenti potrebbero costruire blocchi diversi;
    \item nodi differenti potrebbero finire su diverse chain.
\end{itemize}

\paragraph{Consenso in blockchain}
Il processo per ottenere consenso in blockchain è:
\begin{itemize}
    \item hash di ogni transazione e di un blocco;
    \item hash di un block includendo il blocco precedente;
    \item si aggiunge un trick per fare in modo che la computazione 
    dell'hash del blocco sia molto costosa ma facile da verificare;
    \item i nodi competono su questa operazione e il reward è 
    il blocco stesso.
\end{itemize} 
L'idea del mining è quella di trovare l'hash del blocco precedente 
per firmare la transazione in modo corretto, il reward è la transazione valida.

L'arrivo del blocco validato a uno dei nodi corrisponde all'aggiunta 
alla blockchain, dopo una verifica, potrebbero arrivare catene che si intersecano, 
in tal caso la chain sarà un albero.

\paragraph{Proprietà blockchain}
Se la maggior parte dei nodi lavora sulla stessa catena, quello più lunga è la più
attendibile. 

Un attacco deve calcolare l'hash dei blocchi precedenti, bastano che la metà
dei nodi siano onesti per evitare interventi malevoli.

