\section{Statistical learning}

We talked so far bout data, where 
examples were obtained through \emph{independent} draws from a \emph{fixed} but unknown 
\emph{probability distribution} on the data domain cartesian the label set $X \times Y$.

Fixed means that the probability distribution do not change with time, while independent means 
that every data point is independent from the other. \\The first statement is probably not true, 
as the data distribution could change in time, for instance news or images at certain hours, 
time of year etc.\\
The second statement is probably not true either, as maybe some data points are related, 
let's consider news about a pandemic in time.

The idea is to find a trade-off between simplicity of assumptions and effectivity of algorithms.

\paragraph{Dataset and learning problem}
From now on every example will be of the form $$(X,Y) \approx D \mathit{\;over\;} X \times Y$$ 
where the twos are random variables.
A dataset, made of training and test set will be a collection of \emph{random statistical samples}
$$D = (X_1, Y_1) \dots (X_m, Y_m)$$
A learning problem will be defined as a pair between the dataset and the loss function $(D,l)$.

\subsection{Bayes Optimal Predictor}

\paragraph{Statistical risk}
Given $(D,l)$ and a predictor $h: X \rightarrow Y$, the question is how good is $h$ on $(D,l)$?
We can look at \emph{statistical risk}
$$l_D(h) = E[l(Y, h(X))]$$
The expectation is computed with respect to the random draw of $(X,Y)$ from $D$.
The goal is to find a predictor $h$ such that $l_D(h)$ is small as possible.

\paragraph{Optimal Predictor}
We want to define the predictor with the smallest statistical risk given a problem $(D,l)$, 
it is defined as
$$f^*(x) = \min_{\hat{y} \in Y}E[l(Y, \hat{y})| X =x]$$
this is the conditional expected value of $l(Y, \hat{y})$, given $X = x$. I have a label 
distribution given a value for $X$.
The predictor $f^* : X \rightarrow Y$ is called the \emph{Bayes Optimal Predictor}.

By definition: 
$$\forall h : X \rightarrow Y, E[l(y, f^*(x))|X = x] \leq E[l(y, h(x))|X = x]$$
furthermore, by averaging over $X$, the following holds NOTA 1
$$l_D(f^*) = E[l(y,f^*(x))] \leq E[l(y,h(y))] = l_D(h)$$
Can we compute $f^*$? No unless I know the distribution of $Y$ given $X$. 
\begin{remark}
    That could also mean having to learn the distribution given the dataset, 
    but that requires a lot of data.
\end{remark}
Unless labels are uniquely determined by data points, $l_D(f^*) > 0$, this is also called the 
\emph{Bayes risk}.
The quantity is greater than zero in the case that two same data points have different labels.

\paragraph{Optimal predictor for square loss}
Let's consider the square loss in a regression task.
The optimal predictor now becomes: 
\begin{equation}
    \begin{aligned}
        f^*(x) = \min_{\hat{y} \in \mathbb{R}} E[(y - \hat{y})^2 | X = x]\\
        f^*(x) = \min_{\hat{y} \in \mathbb{R}} E[(y^2 + \hat{y}^2 - 2y\hat{y}) | X = x]\\
        f^*(x) = \min_{\hat{y} \in \mathbb{R}} E[y^2| X = x] + \hat{y}^2 - 2\hat{y}E[y | X = x] && \text{Expected value linearity}\\
        f^*(x) = \min_{\hat{y} \in \mathbb{R}} \hat{y}^2 - 2\hat{y}E[y | X = x] && \text{Can ignore the first}
    \end{aligned}
\end{equation}
We now compute the derivative to find the optimal predictor
\begin{equation}
    \begin{aligned}
        F(\hat{y}) = \hat{y}^2 - 2\hat{y}\square\\
        \frac{\mathrm{d}F(\hat{y})}{\mathrm{d}\hat{y}} = 2\hat{y} - 2\square = 0\\
        \hat{y} = \square
    \end{aligned}
\end{equation}
This means that the optimal predictor can be defined as follows:
$$f^*(x) = E[Y | X = x]$$
If we substitute $\hat{y}$ with $f^*(x)$ in the previous formula, we find that
$$E[(y - f^*(x))^2|X=x] = E[(y - E[Y | x] | X = x)] = \mathit{Var}[Y | X = x]$$
This implies that 
$$l_D(f^*) = E[\mathit{Var}[Y | X]]$$

\begin{remark}
    The idea is to find the right error function that ultimately 
    defines an $f^*$ predictor.
\end{remark}