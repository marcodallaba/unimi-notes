\section{Learning paradigms}

\paragraph{Online learning}
Recalling how a perceptron works and focusing on training 
complexity, we have that an epoch in that learning algorithm is 
linear in the number of data points.
This is a good result and it makes the perceptron really competitive 
on big datasets. 

Also, perceptrons are good on data streams, as they can simply continue
the training phase without recomputing everything.

This is called online learning, so we analyze a single data point 
and adapt the model, this means performing a local minimization.

\paragraph{Batch learning}
The opposite of online learning is batch learning, where we train 
a model on an entire dataset rather than on a single point. 
This means finding a global optimization, such as in ERM.

\subsection{Sequential risk}

\paragraph{Online learning protocol}
We start with parameters, that is a class $H$ of predictors and 
a loss function $l$. 
The process is simple, a default predictor $h_1 \in H$ is given as output, 
then, for $t = 1, 2, \dots$ :
\begin{enumerate}
    \item The next example is observed $(x_t, y_t)$
    \item The loss $l(h_t(x_t), y_t)$ is computed
    \item The predictor is updated $h_t \gets h_{t+1} \in H$
\end{enumerate}
In general the algorithm does not store past examples.

\paragraph{Sequential risk}
The evaluation of the algorithm performances is made via the sequential 
risk, it somehow counts the evolution of the loss.
$$\frac{1}{T}\sum_{t=1}^T l(h_t(x_t), y_t)$$
this decreases overtime as the predictor improves.

\paragraph{Regret}
To have an analogy to the variance error, or the approximation error, 
we introduce the regret measure
$$\frac{1}{T}\sum_{t=1}^T l(h_t(x_t), y_t) - \min_{h\in H}\frac{1}{T}\sum_{t=1}^T l(h(x_t), y_t)$$

\subsection{Gradient descent}
Let's fix a stream of data, we introduce $l_t(h) = l(h_t(x_t), y_t), h\in H$.\\
The question now is, are there any other learning algorithms 
besides perceptron?

In convex optimization the goal is to find a minimum in a convex 
function $f : \mathbb{R}^d \rightarrow \mathbb{R}$
One easy way to minimize such a function is to use gradient descent.

Basically we compute the gradient at a point and move in the opposite direction, 
so towards the minimum gradient.

\paragraph{Projected OGD}
We now focus on using gradient descent in an online fashion, 
with the square loss for a linear predictor.
The main idea is to compute the gradient of the loss 
for each data point and update the model accordingly.

For $t = 1, 2, \dots, \eta > 0, U > 0$ :
\begin{enumerate}
    \item $w^\prime_{t+1} = w_t - \frac{\eta}{\sqrt{t}}\nabla l_t(w_t)$
    \item $w_{t+1} = \min_{w : ||w||\leq U} ||w- w^\prime_{t+1} || $
\end{enumerate}
Our goal is to bound the regret, so 
$$\frac{1}{T}\sum_{t=1}^T l_t(w_t) - \min_{h\in H}\frac{1}{T}\sum_{t=1}^T l_t(u^*_T)\;\;\;
\text{where}\;\;\;
u^*_T = \min_{u : ||u|| \leq U}\frac{1}{T}\sum_{t=1}^T l_t(u)$$

\dots

\emph{sad math noises}

\dots