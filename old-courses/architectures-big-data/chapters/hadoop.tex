\section{Apache Hadoop}
Apache Hadoop è un framework open source con l'intento di aumentare l'affidabilità di un sistema, basandosi sul fatto che l'hardware puo fallire.
\\Ha tre componenti: \emph{storage}, \emph{resource management} e \emph{computazione parallela}.

\subsection{Elementi}

\paragraph{Job Tracker}
Servizio che decide dove una task deve essere eseguita.

\paragraph{Task Tracker}
Nodo che accetta la task data da un job tracker. Espone slot che possono eseguire
task parallele, infatti, all'arrivo di una nuova task, uno slot libero viene utilizzato per eseguire una JVM. Manda segnali di heartbeat.

\paragraph{Name node}
Contiene informazioni su dove i dati sono scritti. Mantiene quindi una lista 
di risorse e i relativi nodi che le contengono. Rappresenta un sigle point of failure
per Hadoop. Esiste un nodo di backup pronto in caso di fallimento.

\paragraph{Data node}
Nodo che contiene i dati effettivi.

\paragraph{Master e worker node}
Un nodo master è composto da un job tracker, un task tracker, un data node e un name node. Mentre un nodo worker è composto da data node e task tracker.

\subsection{Processo}
\begin{center}
        \emph{Moving computation is cheaper than moving data}
\end{center}
Un esempio del processo seguito da Hadoop è il seguente.
\begin{enumerate}
        \item Un'applicazione manda un job asincono al \emph{Job Tracker}, aspettando in polling.
        \item Il \emph{Job Tracker} cerca i dati nel \emph{Name node} e sceglie i Task Tracker più vicini
        con slot disponibili.
        \item Il \emph{Job Tracker} manda i job ai \emph{Task Tracker}, e ne monitora 
        lo stato attraverso i segnali di heartbeat. 
        \item Nel caso di fallimento il \emph{Job Tracker}
        può scegliere di aggiungere il \emph{Task Tracker} a una \emph{blacklist}.
\end{enumerate}

\paragraph{Confronto costi}
Bisogna tenere presente che in un contesto come quello di Hadoop risulta 
più economico muovere la computazione piuttosto che i dati.
Bisogna quindi scegliere i task tracer in modo opportuno, tenendoli vicini ai 
data node interessati, piuttosto che muovere i dati in sè ed eseguire le operazioni.\\
Nel primo caso infatti si parla di muovere magari GB di informazioni, nel secondo di muovere 
righe di codice.

\subsection{HDFS}
HDFS è uno storage distribuito pensato per essere utilizzato su macchine poco
performanti.

\paragraph{Architettura}
Un HDFS ha un architettura \emph{master/slave}, dove esiste un \emph{name node} per $N$ \emph{data node}.
Il primo esegue operazioni sul file system, mentre i secondi si occupano di 
servire le richieste del primo e di gestire cancellazione, creazione e replica dei 
file.

\paragraph{Data node}
I file sono suddivisi in blocchi, le scritture non sono concorrenti sullo stesso 
blocco. Mandano segnali di heartbeat al name node.

 
\paragraph{Quorum Journal Manager}
I data node sono affidabili, viste le tecniche di replicazione, per estendere
ai name node, essi potrebbero essere multipli, uno attivo e gli altri in standby.
I data node manderebbero il battito a tutti questi nodi. Al fallimento del name node 
attivo, esso sarebbe rimpiazzato da un altro nodo.

