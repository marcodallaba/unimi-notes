\documentclass[a4paper]{article}
\usepackage[T1]{fontenc}
\usepackage[utf8]{inputenc}
\usepackage[italian]{babel}
\usepackage{amssymb}
\usepackage{hyperref}
\hypersetup{
    colorlinks=true,
    linkcolor=black,
    filecolor=black,      
    urlcolor=black,
}

\newtheorem{remark}{Osservazzione}
\newtheorem{definition}{Definizione}
\newtheorem{theorem}{Teorema}

\setcounter{secnumdepth}{2}
\setcounter{tocdepth}{2}

\title{Algoritmi evolutivi}
\author{Francesco Tomaselli}
\makeindex
\begin{document}
\maketitle
\tableofcontents
\setlength{\parindent}{0pt}
\setlength{\parskip}{0.8em}
\newpage
\section{Introduzione}
A seguire un'introduzione alle metaeuristiche, per poi passare all'idea 
e alle componenti che stanno alla base degli algoritmi evolutivi.
\subsection{Metaeuristiche}
Una metaeuristica è una tecnica che permette di risolvere problemi 
di ottimizzazione in maniera approssimata, per iterazioni.

Le metaeuristiche sono spesso applicate a problemi non risolvibili 
in modo esatto in tempo utile. Con questo approccio si individua una soluzione
ammissibile.

L'idea di base sta nell'individuare un insieme di soluzioni possibili e 
cercare di migliorarle ad ogni iterazione. Si usa una sorta di \emph{ricerca casuale guidata}
che prevede l'esplorazione dello spazio di soluzioni con un certo grado di casualità, 
ma comunque guidata da una certa metrica di qualità.

Si fa notare che le metaeuristiche possono essere interrotte in qualsiasi 
momento, non serve quindi necessariamente individuare una soluzione ottima.
È possibile infatti prevedere un numero massimo di iterazioni dell'algoritmo, 
sperando che siano state sufficienti a produrre una soluzione "buona".

\subsection{Evoluzione simulata}
Prendendo spunto dall'evoluzione, si possono applicare concetti naturali alla
ricerca di una soluzione per un certo problema.
\begin{definition}
    Un problema di ottimizzazione è definito dalla coppia $(\Omega, f)$, dove 
    il primo indica lo spazio di soluzioni, e il secondo una funzione di valutazione.
    L'obbiettivo è quello di individuare il massimo globale in $\Omega$
\end{definition}
\subsubsection{Principi}
L'evoluzione simulata si basa sui principi di quella naturale, 
in particolare tenta di generare molta diversità nella generazione iniziale, 
per favorire un'esplorazione omogenea dello spazio di ricerca.

Utilizza poi il concetto di selezione naturale e di mantenimento di tratti 
vantaggiosi, ovvero, come in natura, gli individui più forti, o adatti all'ambiente 
in cui si trovano, hanno più probabilità di riprodursi e di sopravvivere di generazione 
in generazione.

\subsubsection{Nozioni di base}
Parlando di algoritmi evolutivi, bisogna introdurre alcune nozioni di base:
\begin{enumerate}
    \item \emph{Individuo}: soluzione candidata
    \item \emph{Cromosoma}: sequenza di oggetti
    \item \emph{Gene}: oggetto, es: bit, carattere, numero, etc.
    \item \emph{Allele}: valore di un oggetto
    \item \emph{Locus}: posizione di un oggetto
    \item \emph{Fenotipo}: applicazione di una soluzione candidata
    \item \emph{Genotipo}: codifica di una soluzione candidata
    \item \emph{Popolazione}: set di cromosomi
    \item \emph{Generazione}: popolazione in un certo istante di tempo
    \item \emph{Riproduzione}: creazione di cromosomi da uno o più cromosomi
    \item \emph{Fitness}: qualità di una soluzione candidata
\end{enumerate}

\subsubsection{Elementi di un problema evolutivo}
Un problema evolutivo prevede la definizione di vari elementi quali:
\begin{enumerate}
    \item Encoding di una soluzione candidata
    \item Generazione della popolazione iniziale
    \item Funzione di Fitness
    \item Metodo di selezione
    \item Operatori genetici di modifica dei cromosomi
    \item Criterio di terminazione
    \item Valori per i relativi parametri
\end{enumerate}

\subsubsection{Schema generale}
Generalmente un algoritmo evolutivo opera seguendo questi passi:
\begin{enumerate}
    \item Generazione di una popolazione iniziale
    \item Valutazione del fitness degli individui
    \item Generazione della nuova popolazione:
    \begin{enumerate}
        \item Selezione dei genitori basata sul fitness calcolato al punto precedente
        \item Generazione di discendenti, partendo dai genitori individuati al punto precedente
        \item Mutazione randomica degli individui
        \item Selezione di $|pop|$ individui tra i padri e i figli
    \end{enumerate}
    \item Ripetizione dal punto 2, fino al raggiungimento del criterio di terminazione
\end{enumerate}
Ragionando sul processo utilizzato, è immediato capire che un algoritmo evolutivo non garantisce
la qualità della soluzione trovata ma si spera che, partendo da una popolazione iniziale 
che copre molte aree dello spazio di ricerca, migliorando iterativamente gli individui 
si riesca a trovare una soluzione buona.

\subsection{Ricerca locale}
Come nell'ottimizzazione classica, si possono applicare varie tecniche per 
migliorare le soluzioni individuate.

Una particolare branca delle tecniche di ottimizzazione consiste nella così 
detta ricerca locale. Si cerca quindi di migliorare localmente la soluzione 
individuata, assumendo quindi che punti 
vicini nello spazio di soluzioni abbiano qualità simili.

\subsubsection{Gradient Ascent e Descent}
L'applicazione di gradient ascent o descent presuppone che:
\begin{enumerate}
    \item Lo spazio di soluzioni $\Omega$ sia un sottoinsieme di $ \mathbb{R}^n$
    \item La funzione di valutazione $f$ sia derivabile in ogni punto
\end{enumerate}
Fissati i due prerequisiti, si può procedere con gradient ascent (o descent):
\begin{enumerate}
    \item Si sceglie un punto iniziale
    \item Viene calcolato il gradiente del punto corrente
    \item Si procede in direzione del gradiente o in senso opposto
    \item I punti 2 e 3 si ripetono fino a quando si individua un minimo o massimo locale
\end{enumerate}
La procedura ha il grande difetto di rimanere bloccata in un minimo locale. Bisogna quindi 
ripetere gli algoritmi cambiando punto di partenza.

\subsubsection{Hill Climbing}
Se la funzione $f$ non è differenziabile, è possibile verificare l'andamento della 
funzione nelle prossimità del punto corrente e scegliere il vicino ottimo.

L'algoritmo si può definire in 4 punti:
\begin{enumerate}
    \item Si parte da un punto casuale $s_0$ in $\Omega$
    \item Viene scelto un punto $s$ nelle vicinanze di $s_t$ 
    \item $s_{t+1}$ diventa $s$ se $f(s) > f(s_t)$, altrimenti rimane $s_t$
    \item I punti 2 e 3 vengono ripetuti fino alla raggiunta di un criterio di terminazione
\end{enumerate}
Hill Climbing ha tendenza a bloccarsi in un ottimo locale, come Gradient Ascent.

\subsubsection{Simulated Annealing}
L'idea fondamentale dell'approccio è che una soluzione migliore di quella 
attuale è sempre accettata, ma è possibile accettarne anche una peggiore 
con una certa probabilità.
Tale mossa potrebbe evitare di rimanere bloccati in un massimo locale.

L'algoritmo è definito come segue:
\begin{enumerate}
    \item Si parte da un punto casuale $s_0$ in $\Omega$
    \item Viene scelto un punto $s$ nelle vicinanze di $s_t$ 
    \item $s_{t+1}$ diventa:
        \begin{itemize}
            \item $s$ se $f(s) > f(s_t)$
            \item $s$ con probabilità $p = e^{-\frac{\Delta f}{kT}}$, dove $\Delta f$ 
            indica la riduzione di qualità della soluzione e $T$ è il parametro temperatura, che
            cala con il passare del tempo
            \item $s_t$ con probabilità $1- p$
        \end{itemize}
    \item I punti 2 e 3 vengono ripetuti fino alla raggiunta di un criterio di terminazione
\end{enumerate}
La temperatura $T$ si riduce di iterazione in iterazione, per valori
 piccoli l'approccio equivale ad hill climbing, ma per valori grandi di $T$
la probabilità di muoversi in una soluzione peggiore aumenta, riducendo il rischio di bloccarsi 
in un massimo locale.

\subsubsection{Threshold Accepting}
L'idea coincide con simulated annealing, ma le soluzioni peggiori sono accettate 
se la degradazione di qualità è inferiore ad una certa threshold, che si riduce
nel tempo.

\subsubsection{Great Deluge Algorithm}
L'idea coincide con simulated annealing, ma le soluzioni peggiori sono accettate 
se la qualità è maggiore di un certo lower bound, incrementato nel tempo.

\subsubsection{Record-to-Record Travel}
L'approccio è molto simile al precedente, questa volta però per calcolare la degradazione 
di qualità si fa riferimento alla soluzione ottima fino ad ora.

\subsection{Altre tecniche}
Esistono altre tecniche di ottimizzazione legate al mondo degli 
evolutionary algorithms, che in alcuni casi sfruttano tecniche di ricerca locale, 
ma non necessariamente.

\subsubsection{Tabu search}
L'approccio consiste nel mantenere una coda FIFO nella quale memorizzare 
soluzioni già esplorate in passato.
Alla creazione di un nuovo candidato si eviterà quindi di esplorare soluzioni già viste.
Non sono considerate le mutazioni.

\subsubsection{Algoritmi memetici}
L'idea alla base dell'approccio è quella di combinare i vantaggi degli algoritmi 
basati su popolazioni e quelli di ricerca locale.

Sarà considerata quindi una popolazione iniziale di soluzioni candidate, e, per
velocizzare l'altrimenti lento processo di ottimizzazione, ogni individuo è migliorato 
localmente fino a raggiungere un massimo.

Nonostante velocizzi il processo di ricerca, è comunque molto probabile rimanere bloccati 
in massimi locali, e non riuscire ad esplorare a sufficienza lo spazio di soluzioni.

\subsubsection{Evoluzione differenziale}
Si utilizza un DE-operator, visto come combinazione di mutazione e ricombinazione.
L'operatore guida la ricerca, alla creazione di un nuovo individuo, esso può rimpiazzare
il padre se la qualità è maggiore.

\subsubsection{Scatter search}
Si genera una popolazione iniziale e si cerca di effettuare una ricerca locale attorno ad 
essa, guidando l'esplorazione dell'algoritmo nella direzione dei migliori individui.

La bontà dell'algoritmo risiede nell'esplorazione dello spazio di soluzioni.

\newpage

\section{Elementi base}
Nella sezione precedente sono stati introdotti brevemente i principali elementi 
di un algoritmo evolutivo, a seguire una descrizione più nel dettaglio.

\subsection{Encoding}
La scelta del tipo di codifica è cruciale per il giusto funzionamento
di un algoritmo evolutivo.
Nonostante non si sia una soluzione generale per tale scelta, un encoding deve rispettare 
alcune caratteristiche:
\begin{enumerate}
    \item Fenotipi simili devono essere rappresentati da genotipi simili
    \item Soluzioni codificate in modo simile, devono avere fitness simile
    \item Lo spazio di soluzioni $\Omega$ dovrebbe essere chiuso rispetto agli operatori genetici
\end{enumerate}
\subsubsection{Hamming Cliffs}
La similarità di due genotipi è definita dalla facilità, espressa in numero 
di mutazioni, di passaggio da uno all'altro. 
Il problema delle Hamming Cliffs si incontra quando l'encoding usato non rispetta 
la prima caratteristica introdotta a inizio sezione.

Non rispettando quel punto infatti, due soluzioni simili sarebbero codificate in maniera
molto diversa, finendo per rendere difficile il passaggio da una all'altra, rendendo più difficoltosa 
l'esplorazione dello spazio di soluzioni. Infatti, per passare dalla prima alla seconda, 
pur essendo molto vicine nel dominio iniziale, sarebbero necessarie troppe mutazioni.
\subsubsection{Epistasi}
Il secondo principio da rispettare consiste nell'avere fitness simili per soluzioni 
codificate in modo simile.
Con epistasi si indica l'interazione dei geni in un cromosoma.

Codifiche che presentano alta epistasi sono indesiderabili, poiché andrebbero a 
impedire un miglioramento graduale delle soluzioni.
\subsubsection{Chiusura dello spazio di soluzioni}
Fissato un certo spazio di soluzioni, bisogna fare in modo che gli operatori 
genetici non creino elementi al di fuori di tale spazio.
Elementi fuori da $\Omega$ non sarebbero interpretabili e risulterebbero inutili.

Per ovviare al problema è possibile:
\begin{enumerate}
    \item Scegliere un encoding differente
    \item Scegliere operatori genetici che non creino elementi al di fuori di $\Omega$
    \item Usare meccanismi di riparazione per gli individui generati in modo errato
    \item Introdurre una penalità che riduca il fitness di tali individui
\end{enumerate}

\subsection{Fitness e selezione}
Il principio base della selezione è che individui migliori hanno più probabilità
di procreare. 
\begin{definition}
    Con selective pressure si definisce quanto gli individui migliori 
    siano preferiti per produrre la generazione successiva
\end{definition}
La selective pressure dovrebbe idealmente essere bassa nelle prime generazioni, 
per assecondare l'esplorazione dello spazio di soluzioni, per poi diventare sempre 
piu alta per migliorare localmente le soluzioni individuate.

\subsubsection{Roulette-Wheel Selection}
Una delle tecniche base per implementare la selezione è la Roulette-wheel selection.
Si definisce il fitness relativo di un individuo come 
$$ 
f_{rel}(s) = \frac{f_{abs}(s)}{\sum_{s\prime\in pop(t)} f_{abs}(s\prime)}
$$
Tale valore rappresenta la probabilità con cui l'individuo può essere selezionato.

\subsubsection{Dominance problem}
Una selezione basata sul fitness come quella al punto precedente potrebbe portare a \emph{crowding}, ovvero
il sovraffollamento di una certa regione dello spazio $\Omega$.
Tale effetto non è desiderabile in quanto non favorisce l'esplorazione dello spazio di soluzioni.

Il problema si manifesta quando esistono individui con fitness troppo alto che dominano gli altri
nel passaggio di generazione.

\subsubsection{Vanishing selective pressure}
In contrapposizione all'avere individui con fitness troppo elevato, 
avere valori di fitness troppo simili nella popolazione porta all'azzeramento della selective pressure.

Considerando ad esempio la formula per il calcolo del fitness relativo, 
con valori molto simili di fitness nella popolazione si otterrà $f_{rel}(s) \approx \frac{1}{\mu}$, 
dove $\mu$ è la dimensione della popolazione.
Un comportamento simile porta a non avere di fatto selective pressure.

Una soluzione al problema consiste nello scalare la funzione di fitness, in 
modo da evitare il comportamento appena descritto:
\begin{itemize}
    \item $\sigma scaling$: scaling basato su media e varianza
    \item \emph{linear dynamical scaling}: sottrae minimo della fitness function da tutti gli individui 
\end{itemize}

\subsubsection{Variance problem}
Durante la fase di selezione è possibile computare il fitness relativo e selezionare 
probabilisticamente in base a quello.
La natura probabilistica dell'approccio potrebbe però scartare individui molto validi.

Una prima soluzione consiste nel determinare il numero di discendenti in base 
al fitness, considerando media e deviazione standard $\mu_f(t)$ e $\sigma_f(t)$:
\begin{itemize}
    \item se $f(s) < \mu_t(t) - \sigma_f(t)$: 0 figli
    \item se $\mu_t(t) - \sigma_f(t) \leq f(s) \leq \mu_t(t) + \sigma_f(t)$: 1 figlio
    \item se $f(s) > \mu_t(t) + \sigma_f(t)$: 2 figli
\end{itemize} 

Un'altra soluzione consiste nel considerare una ruota di selezione, 
la cui circonferenza è suddivisa in tanti archi quanti gli individui, e con grandezza
degli archi proporzionale al fitness relativo di ogni individuo.
Vengono fissati poi dei marker e viene fatta girare la ruota.
Verranno poi selezionati gli individui puntati dai marker.

\subsubsection{Rank-Based selection}
Un approccio simile alla roulette-wheel selection è rank selection.

Gli individui vengono inizialmente ordinati per fitness, successivamente viene 
assegnato loro un rank in base all'ordine ottenuto.
La selezione viene poi effettuata probabilisticamente sul rank.

Tale approccio differisce da roulette-wheel selection per il fatto che non vengono 
considerati direttamente i fitness ma solo l'ordine nella popolazione, 
di conseguenza il valore assoluto del fitness viene ignorato, evitando che 
alcuni individui dominino gli altri,

Adattando in base al tempo le probabilità associate ai rank, si può controllare
la selective pressure, mettendo focus sull'esplorazione piuttosto che sul miglioramento
della popolazione attuale.

\subsubsection{Tournament selection}
Un altro approccio per implementare la selezione consiste nello scegliere
un certo numero di individui e farli competere, scegliendo come vincitore
quello con fitness più alto, o decidendo probabilisticamente in base al fitness.

Facendo così anche gli individui meno forti potrebbero passare alla generazione successiva, 
nel caso in cui vengano messi a confronto con individui peggiori.

\subsubsection{Elitism}
Gli approcci probabilistici potrebbero scartare gli individui migliori 
della generazione corrente.

Per risolvere il problema si possono ammettere alla generazione successiva i primi $k$
individui ordinati per fitness, evitando sempre un peggioramento del fitness massimo.

\subsubsection{Sharing}
La tecnica dello sharing è pensata per prevenire crowding di una certa zona 
dello spazio $\Omega$.

Nello specifico, individui che cadono in una zona affollata riceveranno una penalità
nel fitness, andando potenzialmente a ridurre l'affollamento.

\subsection{Operatori genetici}
Gli operatori genetici sono applicati a un sottoinsieme degli individui in 
una generazione per creare mutazioni e combinazioni di soluzioni esistenti.
Nonostante alcune modifiche possano risultare inconvenienti, la speranza è che 
alcune possano ottenere individui migliori.

Lo spazio di soluzioni $\Omega$ può essere o non essere chiuso rispetto 
a un operatore genetico.

\subsubsection{Mutazione}
Per mutazione si intende un piccolo cambiamento nel genotipo di un individuo.
Si possono applicare vari tipi di mutazione, ad esempio:
\begin{itemize}
    \item \emph{Standard}: cambiamento del valore di un gene
    \item \emph{Pair} swap: interscambio dei valori di due geni
    \item \emph{Shift}: shift di una sottosequenza di geni
    \item \emph{Permutazione arbitraria}: permutazione di una certa sottosequenza di geni
    \item \emph{Inversione}: inversione di una sottosequenza di geni
    \item \emph{Binaria}: inversione di alcuni geni con una certa probabilità
    \item \emph{Gaussiana}: aggiunta ad ogni gene di un valore pescato da una distribuzione Gaussiana
\end{itemize}
\subsubsection{Crossover}
Nella tecnica del crossover, si parte da due o più genitori, e si ricombinano 
spezzando le loro sequenze e ricombinando.

Esistono varianti quali:
\begin{itemize}
    \item \emph{One-point}: individuato un punto di spezzamento, si ricombinano i geni dei due genitori
    \item \emph{N-point}: si individuano n punti di spezzamento invece che uno
    \item \emph{Uniform}: si definisce la probabilità di scambio per ogni coppia
    \item \emph{Shuffle}: viene effettuata una permutazione dei geni, si applica one point crossover, 
    e si riporta all'ordine iniziale
    \item \emph{Diagonal}: simile a N-point, ma con un numero più elevato di genitori
\end{itemize}

\subsubsection{Edge recombination}
Il cromosoma è rappresentato come un grafo.
Ogni gene è un vertice che ha archi verso i suoi vicini. Gli archi dei due
grafi vengono mischiati. Si preserva l’informazione relativa alla vicinanza.

\subsubsection{Problemi degli operatori genetici}
Esistono sostanzialmente due problemi negli operatori genetici:
\begin{enumerate}
    \item \emph{Positional bias}: si manifesta quando la probabilità che due geni
    siano ereditati dallo stesso padre varia in base alla loro posizione relativa. Ad esempio
    in one-point crossover, se due geni sono molto vicini, molto probabilmente saranno ereditati
    assieme.
    \item \emph{Distributional bias}: si manifesta se la probabilità che un certo numero di geni 
    di un genitore passino al figlio è differente al variare del numero di geni considerato.
    Ad esempio, in uniform crossover, la probabilità che un certo numero di geni di uno dei due genitori
    passi al figlio è data da una binomiale. 
\end{enumerate}

\newpage
\section{Algoritmi fondamentali}
Introdotti i componenti fondamentali, si discutono ora alcuni tipi e varianti di algoritmi
evolutivi, con alcuni fondamenti teorici legati alla bontà dell'evoluzione simulata.

\subsection{Fondamenti teorici}
Sebbene il funzionamento degli algoritmi evolutivi sia chiaro, 
risulta poco intuitivo il motivo per cui funzionano, ovvero perché riescano a 
migliorare progressivamente le soluzioni individuate.

\subsubsection{Schema theorem}
Un mezzo teorico per determinare il comportamento di un algoritmo evolutivo è dato 
dallo schema theorem.

\begin{definition}
    Uno schema è definito come una sequenza di uni, zeri e placeholder.
    L'encoding di una soluzione matcha uno schema sse gli uni e gli zeri dello schema sono 
    presenti nella rappresentazione binaria dell'encoding considerato.
\end{definition}

Si può definire il fitness di uno schema come il fitness medio delle soluzioni che matchano tale
schema.

Lo schema theorem dimostra che schemi con elevato fitness si riprodurranno esponenzialmente, 
mentre schemi con fitness basso andranno a morire nelle generazioni successive.
Di conseguenza, sempre più individui andranno a matchare schemi con fitness elevato, aumentando
la qualità della popolazione.

\subsubsection{No free lunch theorem}
Il teorema dimostra che gli algoritmi evolutivi, in media, funzionano tanto bene quanto qualsiasi 
altra metaeuristica. 

Si sta di fatto enunciando l'impossibilità di stabilire a priori la qualità di una soluzione
individuata da un algoritmo evolutivo per un particolare problema del quale non si conosce nulla.

\subsection{Swarm e population based optimization}
Swarm e population based optimization sono alcune metaeuristiche utilizzate nel 
mondo degli algoritmi evolutivi.

L'idea di base fa rifermento a specie naturali capaci di interagire e cooperare tra 
loro, in modo da raggiungere un certo obiettivo.

\subsubsection{Particle swarm}
Ispirato al comportamento di uccelli e pesci, le soluzioni candidate operano aggregando 
informazioni creando conoscenze comuni.

L'idea è quella di guidare le soluzioni candidate considerando conoscenze locali e 
globali, in modo da muoverle in gruppo verso una soluzione ottima.

\subsubsection{Ant colony}
Facendo riferimento al comportamento delle formiche, le soluzioni candidate 
operano seguendo un certo decision graph. 

Si parte da soluzioni parziali ed effettuando scelte random per 
renderle complete. Le decisioni prese da un singolo 
sono influenzate da quelle prese dai compagni.

\subsubsection{Population-based incremental learning}
Si parte da una popolazione iniziale ma, invece che memorizzarla, 
si mantengono le statistiche relative ad essa.

Si seguono i seguenti pattern:
\begin{itemize}
    \item gli individui sono generati pescando dalle statistiche memorizzate.
    Si effettua quindi un Uniform crossover implicito alla creazione di nuove 
    soluzioni.
    \item ad ogni iterazione si considerano i migliori individui per aggiornare le statistiche
    della popolazione.
    \item le mutazioni sono in modalità bit-flipping sulle statistiche 
    \item si considera un learning rate che regola l'esplorazione e il miglioramento locale
\end{itemize}
L'approccio presenta alcuni problemi, quali la possibilità di imparare dipendenze tra
i singoli bit che rappresentano le statistiche, oppure la possibile rappresentazione 
di popolazioni differenti tramite le stesse statistiche.


\subsection{Algoritmi genetici}
La programmazione genetica mira a comporre un programma in grado 
di risolvere un problema dato.
Tale programma è espresso attraverso un \emph{parse tree}.

\subsubsection{Processo}
Si definiscono due insiemi: 
\begin{itemize}
    \item $F$: funzioni e operatori
    \item $T$: simboli terminali
\end{itemize}

Una soluzione candidata è rappresentata da un parse tree, che definisce
una certa espressione, componendo elementi di $F$ e $T$.

Il processo è il solito seguito da un algoritmo evolutivo, 
ovvero creazione, selezione e applicazione di operatori genetici.

\subsubsection{Introni}
Visto che si stanno considerando parse tree che rappresentano espressioni, 
bisogna porre attenzione ad eventuali rami inutili nell'albero.

Un esempio, nel contesto delle operazioni matematiche, può essere 
$a + 1 - 1$, dove la seconda parte dell'espressione è chiaramente inutile.
Bisogna quindi tenerne conto in fase di ricombinazione e mutazione, per evitare 
soluzioni candidate con queste caratteristiche.

\subsection{Multi-criteria optimization}
È possibile avere problemi con più criteri da ottimizzare.
Ad esempio, il problema di acquistare un'auto che sia economica, 
ma comoda e veloce.

La strada da seguire per ottimizzare non è chiara, infatti, probabilmente, 
migliorando uno dei tre criteri, gli altri peggioreranno.

\subsubsection{Pareto optimal solution}
Definite le funzioni da ottimizzare, è possibile individuare quei punti nello 
spazio per cui il miglioramento di una funzione porta al peggioramento delle altre.

Tali punti formano il così detto \emph{Pareto front} e concettualmente
è proprio uno di quei punti che ottimizza il problema visto nel complesso.

\subsubsection{Approccio evolutivo}
L'idea è quella di applicare un algoritmo evolutivo al problema, 
in modo da ottenere quante più possibili soluzioni sul Pareto front.

Un approccio semplice consiste nel definire la funzione di fitness come 
somma pesata delle funzioni da ottimizzare. Purtroppo questo favorisce
soluzioni che massimizzano uno dei criteri, sfavorendo soluzioni intermedie, 
ma è possibile risolvere il problema individuando le soluzioni marginali e scartandole
in fase di selezione.

Un secondo problema è quello della convergenza in un punto qualsiasi del fronte, 
si può rimediare applicando tecniche di \emph{power law sharing}, simili a quelle 
per evitare \emph{crowding}. Tali tecniche, tenendo conto delle zone già 
coperte del fronte, cercheranno di coprire punti inesplorati del fronte, 
in modo da garantire una copertura omogenea.

\subsection{Algoritmi evolutivi paralleli}
Rispetto alle altre metaeuristiche si è osservato che gli algoritmi evolutivi spesso
portano a risultati ottimi, ma con il prezzo di un tempo di esecuzione molto lento.

È possibile parallelizzare alcune fasi del processo in modo da 
velocizzarlo o migliorarne il risultato.
Osservando le varie fasi si nota che sono parallelizzabili:
\begin{itemize}
    \item la generazione iniziale, stando attenti ad eventuali duplicati, 
    che comunque non costituiscono grossi problemi
    \item il calcolo del fitness degli individui, con l'accortezza di raccogliere i dati in un unico processore per 
    calcolare il fitness relativo
    \item la selezione se costituita da eventi indipendenti, come ad esempio tournament selection, 
    diventa più complesso gestire l'elitismo
    \item l'applicazione degli operatori genetici
    \item il controllo di raggiungimento del criterio di terminazione
\end{itemize}

\subsubsection{Island model}
È possibile sfruttare la parallelizzazione considerando un modello ad isola.
Ogni isola avrà una popolazione, ed eseguirà il processo evolutivo.

Si può introdurre migrazione degli individui da un isola all'altra in maniera
random o definita da connessione tra le isole.

\subsubsection{Cellular evolution}
I processori sono organizzati in una griglia. Ogni processore è responsabile di un cromosoma.

Per la selezione ogni processore calcola il massimo dei vicini, gli operatori genetici sono applicabili 
solo tra vicini e la mutazione è gestita da ogni singolo processore.



\end{document}