\section{Vector space model}
\label{vect-vect-model}

\paragraph{Text transformation}
The first step to process queries on text is to transform 
it into a model that can be computed by a machine.

Such a model should be suitable for any kind of text, should be able 
to capture the semantic of words and then should support the notion of
\emph{text similarity}.

The main idea of a vector space model is to map documents in a vector, 
so they appear as points in a certain space.

For instance, we can collect some words and index them, then build a vector 
for each document $$d_k = [x_0, \dots, x_n]$$
where the value $x_i$ stores the occurrences of the word indexed with $i$ 
in that document.

\paragraph{Similarity} 
This model somehow implements the notion of text similarity, represented 
by some sort of distance function between two documents in space, perhaps 
the \emph{Euclidean distance} or the difference between the angle of the two 
vectors.
There are a lot of distance measures, and they vary in applications.
Some of them assume normalization in the document vectors.


\paragraph{Document-Term matrix}
If we arrange document vectors in a vector we obtain the 
\emph{document-term matrix}, or the \emph{term-document matrix} if 
we take the opposite.
The number of dimensions in the space is equal to the 
number of words in the vocabulary. \\
That can be a problem without some sort of lemmatisation and stemming, 
as sparsity and dimension of the matrix increases.

\paragraph{Vector length problem}
By considering the length of the vector in a similarity measure, 
we are introducing some problems, let's just consider a text and its 
abstract, the Euclidean distance would be huge, as most of the words do not 
appear that often. \\
To solve the problem we can consider the \emph{Cosine similarity} or 
normalize vector lengths.

The main phases in the generation of a vector space models are:
tokenization, normalization, weight, and indexing.

\subsection{Tokenization}
The goal is to split the text in words, for instance we could split the
document at the spaces.

In practice this can be obtained with \emph{Regex}, \emph{Corpus} based and \emph{Machine Learning} based
methods.
\subsection{Normalization}
\label{normalization}
The idea is to normalize each token, that could mean writing verbs in the normal 
form, or plurals at the singular.

\paragraph{Stemming}
One form of normalization is stemming, that simply truncate words 
to obtain singular or infinite forms.
The problems here could be creation of meaningless words and also 
close words in meaning could be completely different. An example
is the \emph{Porter} stemming algorithm.

\paragraph{Lemmatisation}
Another way to normalize is to lemmatise, 
It's similar to stemming, but the words are replaced with the dictionary 
root, this also has a problem, indeed sometime mapping of same words to 
different lemmas can happen. A way to lemmatise is to use a 
\emph{Dictionary based} lemmatisation, or something more complicated 
such as \emph{Wordnet}.

\paragraph{Wordnet}
Wordnet is a large lexical database, verbs, nouns, adjectives and adverbs are
grouped into cognitive synonyms, called synsets, each expressing
a different concept.
Synsets are connected to each other by conceptual relations and they are also 
connected to a lemma.

\subsection{Term Weighting}
Weighting the words is a crucial point in text processing, 
an easy way could be counting the frequency of the terms in the document.

\paragraph{Term frequency}
An example of weighing is the term frequency, 
which counts the frequency of a term given a document:
\begin{itemize}
    \item \emph{Boolean}: so a 0 or 1 if the word is present or not
    \item \emph{Natural}: $\mathit{tf}_{t,d} = $  count of a specific word
    \item \emph{Log}: $1 + \log(\mathit{tf}_{t,d})$
    \item \emph{Augmented}: $0.5 + \frac{0.5\mathit{tf}_{t,d}}{\max_{t^\prime} \mathit{tf}_{t^\prime,d}}$
    \item \emph{Log average}:  $0.5 + \frac{0.5 \log(\mathit{tf}_{t,d})}{1 + \log(\mathit{avg}_{t^\prime} \mathit{tf}_{t^\prime,d})}$
    \item \emph{Max tf norm}: $k + (1-k)\frac{\mathit{tf}_{t,d}}{\mathit{tf}_{\max}(d)}$
\end{itemize}

\paragraph{Stop words problem}
A problem in this model is the excessive presence of stop words and punctuations.
To compensate, we can remove them in the normalization phase. \\
But sometimes the stop words can be important, for instance with phrasal verbs.
A solution can be to count the number of documents that contains that given word.

\paragraph{Inverse document frequency}
The idea is to count the number of documents that contains a 
given words, and to prefer less frequent words: $$\mathit{idf} = \log\frac{N}{\mathit{df}_t}$$
There are many variants: 
\begin{itemize}
    \item \emph{Smooth}: $\log(\frac{N}{1 + \mathit{df}_t}) + 1$
    \item \emph{Max}: $\log(\frac{\max_{t^\prime \in d} \mathit{df}_t}{1 + \mathit{df}_t}) + 1$
    \item \emph{Probabilistic}: $\log \frac{N - \mathit{df}_t}{\mathit{df}_t}$
\end{itemize}

\paragraph{TfIdf}
By multyplying term frequency and inverse term frequency we obtain this metric.
$$TfIdf(t, d) = \mathit{tf}_{t,d} \cdot \mathit{idf}_t$$
Regarding the value of terms:
\begin{itemize}
    \item the terms with a high \emph{TfIdf} usually appears in a small number of documents
    \item the metric is low when a term appears a few times in a document or when it appears in 
    many documents
\end{itemize}